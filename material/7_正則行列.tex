\documentclass[dvipdfmx,a4paper,11pt]{article}
\usepackage[utf8]{inputenc}
%\usepackage[dvipdfmx]{hyperref} %リンクを有効にする
\usepackage{url} %同上
\usepackage{amsmath,amssymb} %もちろん
\usepackage{amsfonts,amsthm,mathtools} %もちろん
\usepackage{braket,physics} %あると便利なやつ
\usepackage{bm} %ラプラシアンで使った
\usepackage[top=30truemm,bottom=30truemm,left=25truemm,right=25truemm]{geometry} %余白設定
\usepackage{latexsym} %ごくたまに必要になる
\renewcommand{\kanjifamilydefault}{\gtdefault}
\usepackage{otf} %宗教上の理由でmin10が嫌いなので


\usepackage[all]{xy}
\usepackage{amsthm,amsmath,amssymb,comment}
\usepackage{amsmath}    % \UTF{00E6}\UTF{0095}°\UTF{00E5}\UTF{00AD}\UTF{00A6}\UTF{00E7}\UTF{0094}¨
\usepackage{amssymb}  
\usepackage{color}
\usepackage{amscd}
\usepackage{amsthm}  
\usepackage{wrapfig}
\usepackage{comment}	
\usepackage{graphicx}
\usepackage{setspace}
\usepackage{pxrubrica}
\setstretch{1.2}


\newcommand{\R}{\mathbb{R}}
\newcommand{\Z}{\mathbb{Z}}
\newcommand{\Q}{\mathbb{Q}} 
\newcommand{\N}{\mathbb{N}}
\newcommand{\C}{\mathbb{C}} 
\newcommand{\Sin}{\text{Sin}^{-1}} 
\newcommand{\Cos}{\text{Cos}^{-1}} 
\newcommand{\Tan}{\text{Tan}^{-1}} 
\newcommand{\invsin}{\text{Sin}^{-1}} 
\newcommand{\invcos}{\text{Cos}^{-1}} 
\newcommand{\invtan}{\text{Tan}^{-1}} 
\newcommand{\Area}{\text{Area}}
\newcommand{\vol}{\text{Vol}}
\newcommand{\maru}[1]{\raise0.2ex\hbox{\textcircled{\tiny{#1}}}}
%\newcommand{\rank}{{\rm rank}}



   %当然のようにやる.
\allowdisplaybreaks[4]
   %もちろん.
%\title{第1回. 多変数の連続写像 (岩井雅崇, 2020/10/06)}
%\author{岩井雅崇}
%\date{2020/10/06}
%ここまで今回の記事関係ない
\usepackage{tcolorbox}
\tcbuselibrary{breakable, skins, theorems}

\theoremstyle{definition}
\newtheorem{thm}{定理}
\newtheorem{lem}[thm]{補題}
\newtheorem{prop}[thm]{命題}
\newtheorem{cor}[thm]{系}
\newtheorem{claim}[thm]{主張}
\newtheorem{dfn}[thm]{定義}
\newtheorem{rem}[thm]{注意}
\newtheorem{exa}[thm]{例}
\newtheorem{conj}[thm]{予想}
\newtheorem{prob}[thm]{問題}
\newtheorem{rema}[thm]{補足}

\DeclareMathOperator{\Ric}{Ric}
\DeclareMathOperator{\Vol}{Vol}
 \newcommand{\pdrv}[2]{\frac{\partial #1}{\partial #2}}
 \newcommand{\drv}[2]{\frac{d #1}{d#2}}
  \newcommand{\ppdrv}[3]{\frac{\partial #1}{\partial #2 \partial #3}}


%ここから本文.
\begin{document}
%\maketitle
\begin{center}
{\Large 第7回. 正則行列 (三宅先生の本, 2.4の内容)}
\end{center}

\begin{flushright}
 岩井雅崇 2022/06/02
\end{flushright}
\section{正則行列}

\begin{tcolorbox}[
    colback = white,
    colframe = green!35!black,
    fonttitle = \bfseries,
    breakable = true]
    \begin{dfn}
$A$を$n$次正方行列とする.
 ある行列$B$があって
 $$
 AB =BA =E_{n} %\text{(\,\,\,ただし$E_n$は単位行列)}
 $$
 となるとき\underline{$B$を$A$の逆行列}といい$B=A^{-1}$とかく.
 
 行列$A$が逆行列$A^{-1}$を持つとき, $A$は\underline{正則行列}という(\underline{$A$は正則である}ともいう).
  \end{dfn}
 \end{tcolorbox}
 
 \begin{exa}
 $A=
  \begin{pmatrix}
 1& -5  \\
 0& 1  \\
 \end{pmatrix} 
 $
 の逆行列は
  $A^{-1}=
  \begin{pmatrix}
 1& 5  \\
 0& 1  \\
 \end{pmatrix} 
 $
 である. \\ 
 実際
  $
  \begin{pmatrix}
 1& -5  \\
 0& 1  \\
 \end{pmatrix} 
  \begin{pmatrix}
 1& 5  \\
 0& 1  \\
 \end{pmatrix} 
=
  \begin{pmatrix}
 1& 5  \\
 0& 1  \\
 \end{pmatrix} 
   \begin{pmatrix}
 1& -5  \\
 0& 1  \\
 \end{pmatrix} 
 =
   \begin{pmatrix}
 1& 0 \\
 0& 1  \\
 \end{pmatrix} 
 $
 である.
 特に$A$は正則行列である. 
 \end{exa}

 \begin{exa}
2次正方行列
 $A=
  \begin{pmatrix}
 a& b  \\
 c& d  \\
 \end{pmatrix} 
 $
 について
  $ad-bc \neq 0$ならば, $A$は逆行列を持ち
 $$
 A^{-1} =   
 \frac{1}{ad-bc}
 \begin{pmatrix}
 d& -b  \\
 -c& a  \\
 \end{pmatrix} 
 \text{\,\,\,である.}
 $$
  特に$A$は正則行列である. 
 \end{exa}
 
  \begin{exa}
  $
   A=\begin{pmatrix}
 0& 1 \\
 0& 1  \\
 \end{pmatrix} 
 $
 は逆行列を持たない. 特に$A$は正則行列ではない.
  \end{exa}
  
  \begin{tcolorbox}[
    colback = white,
    colframe = green!35!black,
    fonttitle = \bfseries,
    breakable = true]
    \begin{thm}
    $A$を$n$次正方行列とするとき, 以下は同値.
\begin{enumerate}
\item $\rank (A) =n$
\item $A$の簡約化は$E_n$である.
\item 任意の$n$次列ベクトル$\bm{b}$について, $A \bm{x}=\bm{b}$はただ一つの解をもつ.
\item $A \bm{x}=0$の解は$\bm{x}=0$に限る.
\item $A$は正則行列.
\item $A$の行列式$\det(A)$は0ではない(行列式に関しては第9, 10, 11回の講義でやります).
\end{enumerate}
  \end{thm}
 \end{tcolorbox}
 
 \section{掃き出し法を使った逆行列の求め方}
 \begin{tcolorbox}[
    colback = white,
    colframe = green!35!black,
    fonttitle = \bfseries,
    breakable = true]
    \begin{thm}
    $A$を$n$次正方行列とし, $n \times 2n$行列$[A : E_n]$の簡約化が$[E_n : B]$となるとする.
    このとき$A$は正則行列で, $B$は$A$の逆行列である.
  \end{thm}
 \end{tcolorbox}
 この定理により掃き出し法を用いて逆行列を得ることができる.
 
 \begin{exa}
 $
  A=\begin{pmatrix}
 1& 2&1 \\
 2& 3 & 1 \\
 1& 2 &  2 \\
 \end{pmatrix} 
 $
 の逆行列を求めよ.
 
 (解).
 $[A:E_3] = 
 \begin{pmatrix}
 1& 2&1  &1& 0&0 \\
 2& 3 & 1 &0& 1&0 \\
 1& 2 &  2 &0& 0&1 \\
 \end{pmatrix} 
 $
 を(行)基本変形を用いて簡約化すると, \\
 $
 \begin{pmatrix}
 1& 0&0  &-4& 2&1 \\
 0& 1 & 0 &3& -1&-1 \\
 0& 0&  1 &-1& 0&1 \\
 \end{pmatrix} 
 $
 となる. よって$A$の逆行列は
 $
 \begin{pmatrix}
-4& 2&1 \\
3& -1&-1\\
1& 0&1 \\
 \end{pmatrix} 
 $
 である.
 \end{exa}

\section{演習問題}
演習問題の解答は授業動画にあります.

1.
$
\begin{pmatrix}
 2& -1& 0\\
 2& -1 & -1 \\
 1& 0 &  -1 \\
 \end{pmatrix} 
 $
 の逆行列を求めよ.

 

\end{document}
