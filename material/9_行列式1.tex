\documentclass[dvipdfmx,a4paper,11pt]{article}
\usepackage[utf8]{inputenc}
%\usepackage[dvipdfmx]{hyperref} %リンクを有効にする
\usepackage{url} %同上
\usepackage{amsmath,amssymb} %もちろん
\usepackage{amsfonts,amsthm,mathtools} %もちろん
\usepackage{braket,physics} %あると便利なやつ
\usepackage{bm} %ラプラシアンで使った
\usepackage[top=30truemm,bottom=30truemm,left=25truemm,right=25truemm]{geometry} %余白設定
\usepackage{latexsym} %ごくたまに必要になる
\renewcommand{\kanjifamilydefault}{\gtdefault}
\usepackage{otf} %宗教上の理由でmin10が嫌いなので


\usepackage[all]{xy}
\usepackage{amsthm,amsmath,amssymb,comment}
\usepackage{amsmath}    % \UTF{00E6}\UTF{0095}°\UTF{00E5}\UTF{00AD}\UTF{00A6}\UTF{00E7}\UTF{0094}¨
\usepackage{amssymb}  
\usepackage{color}
\usepackage{amscd}
\usepackage{amsthm}  
\usepackage{wrapfig}
\usepackage{comment}	
\usepackage{graphicx}
\usepackage{setspace}
\usepackage{pxrubrica}
\setstretch{1.2}


\newcommand{\R}{\mathbb{R}}
\newcommand{\Z}{\mathbb{Z}}
\newcommand{\Q}{\mathbb{Q}} 
\newcommand{\N}{\mathbb{N}}
\newcommand{\C}{\mathbb{C}} 
\newcommand{\Sin}{\text{Sin}^{-1}} 
\newcommand{\Cos}{\text{Cos}^{-1}} 
\newcommand{\Tan}{\text{Tan}^{-1}} 
\newcommand{\invsin}{\text{Sin}^{-1}} 
\newcommand{\invcos}{\text{Cos}^{-1}} 
\newcommand{\invtan}{\text{Tan}^{-1}} 
\newcommand{\Area}{\text{Area}}
\newcommand{\vol}{\text{Vol}}
\newcommand{\maru}[1]{\raise0.2ex\hbox{\textcircled{\tiny{#1}}}}
\newcommand{\sgn}{{\rm sgn}}
%\newcommand{\rank}{{\rm rank}}



   %当然のようにやる.
\allowdisplaybreaks[4]
   %もちろん.
%\title{第1回. 多変数の連続写像 (岩井雅崇, 2020/10/06)}
%\author{岩井雅崇}
%\date{2020/10/06}
%ここまで今回の記事関係ない
\usepackage{tcolorbox}
\tcbuselibrary{breakable, skins, theorems}

\theoremstyle{definition}
\newtheorem{thm}{定理}
\newtheorem{lem}[thm]{補題}
\newtheorem{prop}[thm]{命題}
\newtheorem{cor}[thm]{系}
\newtheorem{claim}[thm]{主張}
\newtheorem{dfn}[thm]{定義}
\newtheorem{rem}[thm]{注意}
\newtheorem{exa}[thm]{例}
\newtheorem{conj}[thm]{予想}
\newtheorem{prob}[thm]{問題}
\newtheorem{rema}[thm]{補足}

\DeclareMathOperator{\Ric}{Ric}
\DeclareMathOperator{\Vol}{Vol}
 \newcommand{\pdrv}[2]{\frac{\partial #1}{\partial #2}}
 \newcommand{\drv}[2]{\frac{d #1}{d#2}}
  \newcommand{\ppdrv}[3]{\frac{\partial #1}{\partial #2 \partial #3}}


%ここから本文.
\begin{document}
%\maketitle
\begin{center}
{\Large 第9回. 行列式1 -置換- (三宅先生の本, 3.1の内容)}
\end{center}

\begin{flushright}
 岩井雅崇 2022/06/16
\end{flushright}

\section{置換}

\begin{tcolorbox}[
    colback = white,
    colframe = green!35!black,
    fonttitle = \bfseries,
    breakable = true]
    \begin{dfn}
    \text{}
    \begin{itemize}
\item $\{ 1, \ldots, n\}$から$\{ 1, \ldots, n\}$への1対1写像を\underline{置換}と言い$\sigma$で表す.
つまり置換$\sigma$とは$k_1, \ldots, k_n$を1から$n$の並び替えとして, 
1を$k_1$に, 2を$k_2$に, $\cdots$, $n$を$k_n$にと変化させる規則のことである.
\item 上の置換$\sigma$を
$$
\sigma =
  \begin{pmatrix}
 1& 2  &\cdots &n\\
 k_1& k_2  &\cdots &k_n\\
 \end{pmatrix} 
$$
とかき, $\sigma(1) =k_1, \sigma(2) =k_2, \ldots, \sigma(n) =k_n$とする.
    \end{itemize}
  \end{dfn}
 \end{tcolorbox}
 
 \begin{exa}
 置換$\sigma$を
$
\sigma =
  \begin{pmatrix}
 1& 2  &3 &4\\
 3& 1  &4 &2\\
 \end{pmatrix} 
$
とする. 
これは「1を$3$に, 2を$1$に, 3を4に, 4を$2$にと変化させる規則」である.
 $\sigma(1) =3, \sigma(2) =1, \sigma(3) =4,  \sigma(4) =2$である.
 \end{exa}
 
 \begin{exa}
 置換$\sigma$を
$
\sigma =
  \begin{pmatrix}
 1& 2  &3 \\
 2& 1  &3 \\
 \end{pmatrix} 
$
とする. 
これは「1を$2$に, 2を$1$に, 3を3にと変化させる規則」である.
 $\sigma(1) =2, \sigma(2) =1, \sigma(3) =3$である.
 
 この置換は3に関しては何も変化させていないので
 $
\sigma =
  \begin{pmatrix}
 1& 2   \\
 2& 1   \\
 \end{pmatrix} 
$
ともかく.
 \end{exa}

\begin{tcolorbox}[
    colback = white,
    colframe = green!35!black,
    fonttitle = \bfseries,
    breakable = true]
    \begin{dfn}
置換$\sigma, \tau$について, その積$\sigma \tau$を
$\sigma(\tau(i))$で定める.
  \end{dfn}
 \end{tcolorbox}
 
\begin{exa}
 置換$\sigma, \tau$を
$
\sigma =
  \begin{pmatrix}
 1& 2  &3 & 4 \\
 4& 3  &1  &2 \\
 \end{pmatrix} 
\tau=
  \begin{pmatrix}
 1& 2  &3 & 4 \\
 2& 3  &4  &1 \\
 \end{pmatrix} 
 $
とすると, 
$$
  \begin{matrix}
 \sigma (\tau (1)) &= &  \sigma (2)  & = & 3  \\
 \sigma (\tau (2)) &= &  \sigma (3)  & = & 1 \\
 \sigma (\tau (3)) &= &  \sigma (4)  & = & 2  \\
 \sigma (\tau (4)) &= &  \sigma (1)  & = & 4  \\
 \end{matrix} 
 \text{\,\,\,であるので, }
 \sigma \tau
= 
 \begin{pmatrix}
 1& 2  &3 & 4 \\
 3& 1  &2  &4 \\
 \end{pmatrix} 
 \text{である.}
$$

\end{exa}


\begin{tcolorbox}[
    colback = white,
    colframe = green!35!black,
    fonttitle = \bfseries,
    breakable = true]
    \begin{dfn}
\text{}
\begin{itemize}
\item $
\epsilon =
  \begin{pmatrix}
 1& 2  &\cdots &n\\
 1& 2  &\cdots &n\\
 \end{pmatrix} $を\underline{単位置換}という.
 \item  
 $ \sigma =
  \begin{pmatrix}
 1& 2  &\cdots &n\\
 k_1& k_2  &\cdots &k_n\\
 \end{pmatrix} 
$について, 
$
  \begin{pmatrix}
 k_1& k_2  &\cdots &k_n\\
 1& 2  &\cdots &n\\
 \end{pmatrix} 
$を\underline{$\sigma$の逆置換}と言い$\sigma^{-1}$で表す.
\end{itemize}
  \end{dfn}
 \end{tcolorbox}

\begin{exa} 
$\sigma = 
\begin{pmatrix}
 1& 2  &3 & 4 & 5\\
 4& 5  &1  &3 &2\\
 \end{pmatrix} 
$
とするとき
$
\sigma^{-1}
=
\begin{pmatrix}
 4& 5  &1  &3 &2\\
 1& 2  &3 & 4 & 5\\
 \end{pmatrix} 
 =
 \begin{pmatrix}
 1& 2  &3 & 4 & 5\\
 3& 5  &4  &1 &2\\
 \end{pmatrix} 
 \text{である.}
$
\end{exa}

\begin{tcolorbox}[
    colback = white,
    colframe = green!35!black,
    fonttitle = \bfseries,
    breakable = true]
    \begin{dfn}
 $ \sigma =
  \begin{pmatrix}
 k_1& k_2  &\cdots &k_l\\
 k_2& k_3  &\cdots &k_1\\
 \end{pmatrix} 
$となる置換$\sigma$を\underline{巡回置換}と言い
$\sigma =
  \begin{pmatrix}
 k_1& k_2  &\cdots &k_l\\
 \end{pmatrix} 
$と表す.

特に $ \sigma =
  \begin{pmatrix}
 k_1& k_2  \\
 k_2& k_1 \\
 \end{pmatrix} 
$となる巡回置換を\underline{互換}と言い$\sigma =
  \begin{pmatrix}
 k_1& k_2 \\
 \end{pmatrix} 
$と表す.
  \end{dfn}
 \end{tcolorbox}


\begin{tcolorbox}[
    colback = white,
    colframe = green!35!black,
    fonttitle = \bfseries,
    breakable = true]
    \begin{thm}
任意の置換$\sigma$は互換の積$\tau_1 \cdots \tau_{l}$で表わすことができ, $l$の偶奇は$\sigma$によってのみ定まる.
  \end{thm}
 \end{tcolorbox}
 
 \begin{tcolorbox}[
    colback = white,
    colframe = green!35!black,
    fonttitle = \bfseries,
    breakable = true]
    \begin{dfn}
置換$\sigma$が互換の積$\tau_1 \cdots \tau_{l}$で表せられているとする.
\begin{itemize}
\item $\sgn (\sigma) = (-1)^{l}$とし, これを\underline{$\sigma$の符号}と呼ぶ.
\item $\sgn (\sigma) = 1$なる置換$\sigma$を\underline{偶置換}といい, $\sgn (\sigma) = -1$なる置換$\sigma$を\underline{奇置換}という.
\end{itemize}
  \end{dfn}
 \end{tcolorbox}
 
 \begin{exa}
 $
 \sigma 
  =
 \begin{pmatrix}
 1& 2  &3 & 4 & 5 & 6 & 7\\
 4& 1  &6  &2 &7 & 5 & 3\\
 \end{pmatrix} 
 $を互換の積で表し, その符号を求めよ.
 
 (解). 
 $1 \overset{\sigma}{\rightarrow} 4 \overset{\sigma}{\rightarrow}2 \overset{\sigma}{\rightarrow}1 $と変化し,  
  $3 \overset{\sigma}{\rightarrow} 6\overset{\sigma}{\rightarrow}5 \overset{\sigma}{\rightarrow}7 \overset{\sigma}{\rightarrow}3$と変化するので, 
  $$
  \sigma = 
   \begin{pmatrix}
 1& 4 &2 
 \end{pmatrix} 
    \begin{pmatrix}
 3& 6 &5 &7
 \end{pmatrix} 
 \text{である.}
  $$
  さらに
  $   \begin{pmatrix}
 1& 4 &2 
 \end{pmatrix} 
 = 
 \begin{pmatrix}
 1& 4  
 \end{pmatrix} 
 \begin{pmatrix}
 4 &2 
 \end{pmatrix},
\begin{pmatrix}
 3& 6 &5 &7
 \end{pmatrix} 
 =
 \begin{pmatrix}
 3& 6  
 \end{pmatrix} 
  \begin{pmatrix}
 6& 5  
 \end{pmatrix} 
  \begin{pmatrix}
 5& 7  
 \end{pmatrix} 
 $
 であるので, 
 $$
\sigma= 
\begin{pmatrix}
 1& 4  
 \end{pmatrix} 
 \begin{pmatrix}
 4 &2 
 \end{pmatrix}
 \begin{pmatrix}
 3& 6  
 \end{pmatrix} 
  \begin{pmatrix}
 6& 5  
 \end{pmatrix} 
  \begin{pmatrix}
 5& 7  
 \end{pmatrix} 
 $$
 となり, $\sgn(\sigma)= (-1)^{5}=-1$である.
 
 \end{exa}

  \begin{tcolorbox}[
    colback = white,
    colframe = green!35!black,
    fonttitle = \bfseries,
    breakable = true]
    \begin{prop}置換$\sigma, \tau$について, 
    $\sgn(\epsilon) = 1$, $\sgn(\sigma^{-1}) = \sgn(\sigma)$, 
$\sgn(\sigma \tau) = \sgn(\sigma) \sgn(\tau) $が成り立つ(ただし$\epsilon$は単位置換とする).
  \end{prop}
 \end{tcolorbox}
 
 
  \begin{tcolorbox}[
    colback = white,
    colframe = green!35!black,
    fonttitle = \bfseries,
    breakable = true]
    \begin{dfn}
$S_n$を$n$文字置換の集合とし, $A_n$を$n$文字置換の集合とする.
  \end{dfn}
 \end{tcolorbox}
 \footnote{専門用語で$S_n$は対称群と言い, $A_n$は交代群と言います. }

  \begin{tcolorbox}[
    colback = white,
    colframe = green!35!black,
    fonttitle = \bfseries,
    breakable = true]
    \begin{prop}\text{}
    \begin{itemize}
\item $S_n$の個数は$n!$個である.
\item 偶置換と奇置換の個数は同じである.
\item $A_n$の個数は$\frac{n!}{2}$個である.
\item $\sigma, \tau \in A_n$ならば$\sigma \tau \in A_n$
    \end{itemize}
  \end{prop}
 \end{tcolorbox}

 

\end{document}
