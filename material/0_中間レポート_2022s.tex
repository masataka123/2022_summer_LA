\documentclass[dvipdfmx,a4paper,11pt]{article}
\usepackage[utf8]{inputenc}
%\usepackage[dvipdfmx]{hyperref} %リンクを有効にする
\usepackage{url} %同上
\usepackage{amsmath,amssymb} %もちろん
\usepackage{amsfonts,amsthm,mathtools} %もちろん
\usepackage{braket,physics} %あると便利なやつ
\usepackage{bm} %ラプラシアンで使った
\usepackage[top=30truemm,bottom=30truemm,left=25truemm,right=25truemm]{geometry} %余白設定
\usepackage{latexsym} %ごくたまに必要になる
\renewcommand{\kanjifamilydefault}{\gtdefault}
\usepackage{otf} 


\usepackage[all]{xy}
\usepackage{amsthm,amsmath,amssymb,comment}
\usepackage{amsmath}    % \UTF{00E6}\UTF{0095}°\UTF{00E5}\UTF{00AD}\UTF{00A6}\UTF{00E7}\UTF{0094}¨
\usepackage{amssymb}  
\usepackage{color}
\usepackage{amscd}
\usepackage{amsthm}  
\usepackage{wrapfig}
\usepackage{comment}	
\usepackage{graphicx}
\usepackage{setspace}
\setstretch{1.2}


\newcommand{\R}{\mathbb{R}}
\newcommand{\Z}{\mathbb{Z}}
\newcommand{\Q}{\mathbb{Q}} 
\newcommand{\N}{\mathbb{N}}
\newcommand{\C}{\mathbb{C}} 
\newcommand{\Sin}{\text{Sin}^{-1}} 
\newcommand{\Cos}{\text{Cos}^{-1}} 
\newcommand{\Tan}{\text{Tan}^{-1}} 
\newcommand{\invsin}{\text{Sin}^{-1}} 
\newcommand{\invcos}{\text{Cos}^{-1}} 
\newcommand{\invtan}{\text{Tan}^{-1}} 
\newcommand{\Area}{\text{Area}}
\newcommand{\vol}{\text{Vol}}
\newcommand{\maru}[1]{\raise0.2ex\hbox{\textcircled{\tiny{#1}}}}
\newcommand{\sgn}{{\rm sgn}}




   %当然のようにやる.
\allowdisplaybreaks[4]
   %もちろん.
%\title{第1回. 多変数の連続写像 (岩井雅崇, 2020/10/06)}
%\author{岩井雅崇}
%\date{2020/10/06}
%ここまで今回の記事関係ない
\usepackage{tcolorbox}
\tcbuselibrary{breakable, skins, theorems}

\theoremstyle{definition}
\newtheorem{thm}{定理}
\newtheorem{lem}[thm]{補題}
\newtheorem{prop}[thm]{命題}
\newtheorem{cor}[thm]{系}
\newtheorem{claim}[thm]{主張}
\newtheorem{dfn}[thm]{定義}
\newtheorem{rem}[thm]{注意}
\newtheorem{exa}[thm]{例}
\newtheorem{conj}[thm]{予想}
\newtheorem{prob}[thm]{問題}
\newtheorem{rema}[thm]{補足}

\DeclareMathOperator{\Ric}{Ric}
\DeclareMathOperator{\Vol}{Vol}
 \newcommand{\pdrv}[2]{\frac{\partial #1}{\partial #2}}
 \newcommand{\drv}[2]{\frac{d #1}{d#2}}
  \newcommand{\ppdrv}[3]{\frac{\partial #1}{\partial #2 \partial #3}}



%ここから本文.
\begin{document}
%\maketitle
\begin{center}
{ \large 大阪大学 2022年度春夏学期 全学共通教育科目 \\ 木曜2限 線形代数学I (理(生物・生命(化・生)))} \\
\vspace{5pt}

{\LARGE 中間レポート } \\
\vspace{5pt}

{ \Large 提出締め切り 2022年6月16日(木) 23時59分00秒 (日本標準時刻)}
\end{center}

\begin{flushright}
 担当教官: 岩井雅崇(いわいまさたか) 
\end{flushright}

{\Large $\bullet$ 注意事項}
\begin{enumerate}
\item 第1問から第6問まで解くこと. 
\item おまけ問題は全員が解く必要はない(詳しくは第1回目授業のスライドを参照せよ).
\item 用語に関しては授業または教科書(三宅敏恒著 入門線形代数(培風館))に準じます.
\item \underline{提出締め切りを遅れて提出した場合, 大幅に減点する可能性がある.}
\item \underline{名前・学籍番号をきちんと書くこと.}
\item \underline{解答に関して, 答えのみならず, 答えを導出する過程をきちんと記してください.} きちんと記していない場合は大幅に減点する場合がある.
%ただし用語の定義の違いによるミスに関して, 大幅に減点することはない.
\item 字は汚くても構いませんが, \underline{読める字で濃く書いてください.} あまりにも読めない場合は採点をしないかもしれません.%\footnote{私も字が汚い方ですので人のこと言えませんが...自論ですが, 字が汚いと自覚ある人は大きく書けば読みやすくなると思います.}
\item 採点を効率的に行うため, \underline{順番通り解答するようお願いいたします.}
\item 採点を効率的に行うため,  \underline{レポートはpdfファイル形式で提出し,} ファイル名を「lin(学籍番号).pdf」とするようお願いいたします(linは線形(linear)の略です).
例えば学籍番号が「04D99999」の場合はファイル名は「lin04D99999.pdf」となります.
\end{enumerate}

 \begin{tcolorbox}[
    colback = white,
    colframe = black,
    fonttitle = \bfseries,
    breakable = true]
    レポート提出前のチェックリスト
    \begin{itemize}
    \item[] $\Box$ 締め切りを守っているか?
    \item[] $\Box$ レポートに名前・学籍番号を書いたか?
     \item[] $\Box$ 答えを導出する過程をきちんと記したか?
     \item[] $\Box$ 計算ミスしていないか?
    \item[] $\Box$ 他者が読める字で書いたか?
    \item[] $\Box$ 順番通り解答したか?
    \item[] $\Box$レポートはpdfファイル形式で提出したか?
   \item[] $\Box$ ファイル名を「lin(学籍番号).pdf」としたか?
    \end{itemize}

  \end{tcolorbox}
  
%2020年12月15日(火)の10時50分からオンラインによる質疑応答の場を設けます. (出席義務はありません, 来たい人だけ来てください. レポートに関する質問も可とします.) 質疑応答に関してはWebClassを参照してください.
 
\newpage
 \hspace{-11pt}
{\Large  レポートの提出方法について }

\begin{itemize}
\setlength{\parskip}{0cm} % 段落間
  \setlength{\itemsep}{0cm}
  \item \underline{原則的にCLEからの提出しか認めません.}
レポートは余裕を持って提出してください.
\item \underline{レポートはpdfファイルで提出してください.}
またCLEからの提出の際, 提出ファイルを一つにまとめる必要があるとのことですので, 提出ファイルを一つにまとめてください.
\item \underline{採点を効率的に行うため, ファイル名を「lin(学籍番号).pdf」とするようお願いいたします.}
(linは線形(linear)の略です).
例えば学籍番号が「04D99999」の場合はファイル名は「lin04D99999.pdf」となります.
\end{itemize}

 \hspace{-11pt}
{\Large 提出用pdfファイルの作成の仕方について}
\vspace{11pt}

1つ目は「手書きレポートをpdfにする方法」があります.
この方法は時間はあまりかかりませんが, お金がかかる可能性があります.
手書きレポートをpdfにするには以下の方法があると思います.
\begin{itemize}
\setlength{\parskip}{0cm} % 段落間
  \setlength{\itemsep}{0cm}
\item スキャナーを使うかコンビニに行ってスキャンする.
\item スマートフォンやカメラで画像データにしてからpdfにする. 例えばMicrosoft Wordを使えば画像データをpdfにできます. また大阪大学の学生であればMicrosoft Wordを無料でインストールすることができます.
%(見づらくなる可能性あり)
\item その他いろいろ検索して独自の方法を行う.
\end{itemize}

2つ目は「TeXでレポートを作成する方法」があります.
時間はかなりかかります. 見た目はかなり綺麗ですがあまりお勧めしません.
\vspace{11pt}

他にもいろいろと方法はあると思います. 最終的に私が読めるように書いたレポートであれば大丈夫です.
%他者が読める字で書いてあれば問題ありません. (私が読めるようなレポートであれば大丈夫です.)

\vspace{11pt}
 \hspace{-11pt}
{\Large CLEからの提出が不可能な場合}
\vspace{11pt}

提出の期限までに (CLEのシステムトラブル等の理由で)CLEからの提出が不可能な場合のみメール提出を受け付けます.
その場合には以下の項目を厳守してください.
\begin{itemize}
\setlength{\parskip}{0cm} % 段落間
  \setlength{\itemsep}{0cm}
\item 大学のメールアドレスを使って送信すること(なりすまし提出防止のため).
\item 件名を「レポート提出」とすること
\item 講義名, 学籍番号, 氏名 (フルネーム)を書くこと.
\item レポートのファイルを添付すること.
\item CLEでの提出ができなかった事情を説明すること. 提出理由が不十分である場合, 減点となる可能性があります.
\end{itemize}

メール提出の場合はmasataka[at]math.sci.osaka-u.ac.jpにメールするようお願いいたします.

%正当な理由(WebClassのシステムトラブル等)ではない場合, メールでの提出は減点対象となるので注意すること.
\newpage
\begin{center}
{\LARGE 中間レポート問題.} 
\end{center}

 {\Large 第1問} (授業第2-3回の内容).
 
 \vspace{11pt}
次の行列の計算を行え.
 
  \vspace{11pt}
(1).
$
 \begin{pmatrix}
 1 &2 \\
 -4&-1\\
  5&-2\\
 \end{pmatrix}
 + 2
 \begin{pmatrix}
 2 &-1 \\
  0&4\\
  -7&0\\
 \end{pmatrix}
 $
(2).
$
3 \begin{pmatrix}
 2 &-1&4 \\
 0&3&-5\\
 \end{pmatrix}
 - 2
 \left\{
 \begin{pmatrix}
 0 &1&-2 \\
 7&-5&4\\
 \end{pmatrix}
 - 3
  \begin{pmatrix}
 1 &-2&6 \\
 4&-1&5\\
 \end{pmatrix}
\right\}
 $


   \vspace{33pt}
   
{\Large 第2問} (授業第2-3回の内容).

\vspace{11pt}
次の行列$A,B,C,D$のうち, 積が定義される全ての組み合わせを求め, その積を計算せよ.
 $$
  A=\begin{pmatrix} %14
 -1 & 2 &-5  \\
 \end{pmatrix} 
 \text{, \,\,} 
B= \begin{pmatrix} %33
 1& 0 & 2\\
 0 & 3 & 0\\
 4 & 0 & 5 \\
 \end{pmatrix} %%32
 \text{, \,\,} 
 C=
  \begin{pmatrix}
 -2 &5 & 3\\
1 &-3&0  \\
 \end{pmatrix}
 \text{, \,\,} 
 D= \begin{pmatrix} %%41
 -4\\
 3 \\
 1
 \end{pmatrix}
 $$
 
%行列$A$を次で定める. $$ A = \begin{pmatrix}3&-2&10&8&0 \\1&-1&-9&12&9 \\5&-7&-1&2&8 \\\end{pmatrix $$
% 次の問いに答えよ.\begin{enumerate}\item $A$の型をいえ.\item $A$の$(3,2)$成分をいえ.\item $A$の第2行をいえ.\item $A$の第4列をいえ.\item $A$の転置行列${}^{t}A$を求めよ.\end{enumerate}

 \vspace{33pt}
 
   
   {\Large 第3問} (授業第2-3回の内容).
    \vspace{11pt}
    
    $
A = \begin{pmatrix} %%41
2 & 1\\
1 & 2\\
 \end{pmatrix}
 $
 $
P =\frac{1}{\sqrt{2}} 
\begin{pmatrix} %%41
1& 1\\
-1 & 1\\
 \end{pmatrix}
 $
 とおく. 次の問いに答えよ.
     \vspace{11pt}
 
(1). $A^2$と$A^3$をそれぞれ求めよ.
 
(2). $P{}^tP$と${}^t PP$をそれぞれ求めよ.\footnote{$P{}^tP$とは$P$と${}^tP$($P$の転置行列)の積である.}

(3). ${}^tP A P$を求めよ.

(4). $n$を1以上の整数とする. $({}^tP A P)^n$を$n$を用いて表せ.

(5). $n$を1以上の整数とする. $A^n$を$n$を用いて表せ.

     \vspace{33pt} 
     
  \begin{flushright}
 {\LARGE 第4問に続く.}
 \end{flushright}
     
 \newpage
 
 {\Large 第4問} (授業第4-6回の内容).
 
    \vspace{11pt}
 次の行列を簡約化し, その階数を求めよ.
 
 \vspace{11pt}
(1).
$
 \begin{pmatrix}
2&1&-1 \\
1&1& 1 \\
3&1&-3 \\
 \end{pmatrix}
 $
(2).
$
 \begin{pmatrix}
 1& 1& 5  & 0&3\\
 3& 1& 9  & 1&8\\
 2& 0& 4 & 1&5\\
 2& 1& 7 & 1&7\\
 \end{pmatrix}
 $
 (3).
 $
 \begin{pmatrix}
 1& 2& 3  & 4&5\\
 2& 3& 4  & 5&6\\
 3& 4& 5 & 6&7\\
 4& 5& 6 & 7&8\\
 5& 6& 7 & 8&9\\
 \end{pmatrix}
 $
 
\vspace{33pt} 
   
{\Large 第5問} (授業第4-6回の内容).
    \vspace{11pt}

次の連立1次方程式を解け. \\

(1).
 $
 \left\{ 
\begin{matrix}
x_1& + &  2x_2&  +& x_3&  = & 0 \\
2x_1& + & 3x_2&  +& x_3&  = & 0 \\
 x_1& + & 2x_2&  +& 2x_3&  = & 0 \\
\end{matrix}
\right.
 $

(2).
 $
 \left\{ 
\begin{matrix}
x_1& + &  x_2&  +& 5x_3&  && = & 3 \\
2x_1& + &  x_2&  +& 7x_3& + &x_4& = & 7 \\
3x_1& + &  x_2&  +& 9x_3& + &x_4& = & 8 \\
\end{matrix}
\right.
 $
 
(3).
 $
 \left\{ 
\begin{array}{ccccccccccc}
x_1& +& x_2&  -&2x_3	&+&x_4& +&3x_5&=& 1\\
2x_1&-&x_2& + &2x_3&+&2x_4&+&6x_5&= &2 \\
3x_1&+&2x_2& - &4x_3& - &  3x_4  &-&9x_5&= &3\\
\end{array}
\right.
 $
 
\vspace{33pt} 

{\Large 第6問} (授業第4-6回の内容).
    \vspace{11pt}

連立1次方程式
 $$
 \left\{ 
\begin{array}{ccccccccccc}
x_1&-&2x_2&  -&x_3	&+&x_4& &	&=& 0\\
-2x_1&+&5x_2& + &3x_3&-&2x_4&+&x_5&= &-1 \\
x_1&+&x_2& + &2x_3& &    &-&x_5&= &1\\
5x_1& & & + &5x_3& +&3x_4   &+&2x_5&= &a\\
\end{array}
\right.
 $$
の解が存在するような$a$の値を全て求めよ.
 
\vspace{33pt} 

  \begin{flushright}
 {\LARGE 中間レポートおまけ問題に続く.}
 \end{flushright}
 \newpage
 
{\Large 中間レポートおまけ問題} (授業第4-6回の内容).
\vspace{11pt}

全ての成分が0か1である$n$次正方行列について次の操作を考える.

\vspace{5pt}
 \begin{tcolorbox}[
    colback = white,
    colframe = black,
    fonttitle = \bfseries,
    breakable = true]
(操作): $(i,j)$成分を自由に一つ選び, $(i,j)$成分とその上下左右の全ての成分に対して, 0と1を入れ替える.
 \end{tcolorbox}
\vspace{5pt}

例えば
$
A =
 \begin{pmatrix}
1 & 0 & 1\\
1 & 1 & 1\\
0 & 0 & 0 \\
 \end{pmatrix}
 $
 の場合, $(2,2)$成分を選んで上の操作を行うと次のように変化する:
 $$
  \begin{pmatrix}
1 & \colorbox[rgb]{0.8, 1.0, 0.8}{0} & 1\\
\colorbox[rgb]{0.8, 1.0, 0.8}{1} & \colorbox[rgb]{0.8, 1.0, 0.8}{1} & \colorbox[rgb]{0.8, 1.0, 0.8}{1}\\
0 & \colorbox[rgb]{0.8, 1.0, 0.8}{0} & 0 \\
 \end{pmatrix}
 \rightarrow 
  \begin{pmatrix}
1 & 1 & 1\\
0 & 0 & 0\\
0 & 1 & 0 \\
 \end{pmatrix}
 $$
 
上の$A$に対し, $(1,2)$成分を選んで上の操作を行うと次のように変化する:\footnote{$(1,2)$成分に対して, その上の成分は存在しないため, この場合は(1,1), (1,2), (1,3), (2,2)の成分について0と1を入れ替えることになる.}
 $$
  \begin{pmatrix}
\colorbox[rgb]{0.8, 1.0, 0.8}{1} & \colorbox[rgb]{0.8, 1.0, 0.8}{0} & \colorbox[rgb]{0.8, 1.0, 0.8}{1}\\
1 & \colorbox[rgb]{0.8, 1.0, 0.8}{1} & 1\\
0 & 0 & 0 \\
 \end{pmatrix}
 \rightarrow 
  \begin{pmatrix}
0 & 1 & 0\\
1 & 0 & 1\\
0 & 0 & 0 \\
 \end{pmatrix}
 $$

上の$A$に対し, $(3,3)$成分を選んで上の操作を行うと次のように変化する:
 $$
  \begin{pmatrix}
1 & 0 & 1\\
1 & 1 & \colorbox[rgb]{0.8, 1.0, 0.8}{1}\\
0 & \colorbox[rgb]{0.8, 1.0, 0.8}{0} & \colorbox[rgb]{0.8, 1.0, 0.8}{0} \\
 \end{pmatrix}
 \rightarrow 
  \begin{pmatrix}
1 & 0 & 1\\
1 & 1 & 0\\
0 & 1 & 1 \\
 \end{pmatrix}
 $$

次の問いに答えよ.
\vspace{11pt}

(1). 
$
B = 
  \begin{pmatrix}
1 & 1 & 1 & 1\\
1 & 0 & 0 & 0 \\
1 & 0 & 0 & 0 \\
1 & 0 & 0 & 1 \\
 \end{pmatrix}
 $
とする. $B$に上の操作を何回か行なって零行列にできることを示せ.

(2). $B$に上の操作を何回か行なって零行列にするために必要な最小の操作回数を求めよ.

(3). 与えられた$n$次正方行列$C$について, 上の操作を何回か行なって零行列にすることが可能か判定し, 可能ならば零行列にするために必要な最小の操作回数を求めるアルゴリズムを構築せよ.  \footnote{$n$は10程度を想定しています. $n=10$でも処理時間が2秒以内に収まるアルゴリズムを構築してください.}

\vspace{11pt}
中間レポートおまけ問題を解答するに際し, 次の点に注意すること.
\begin{enumerate}
\item[注意1.]  この問題に限りプログラミングや計算機を用いて解答して良い. %\footnote{言い忘れましたが, 中間レポートの\underline{検算}をするためにプログラミングや計算機を用いることは許可しております. むしろ線形代数の理解を深めるためにもプログラミングを用いて検算を行った方が良いと思います. }
\item[注意2.] (3)の解答については「第7回授業の簡約化ができることの証明」のように記述しても良いし, 実際にプログラミングをして提出しても良い. プログラミングを用いて提出した場合はボーナスとして得点を何点か加点する. 
\item[注意3.]プログラミングを用いて提出する場合に際し, プログラミング言語に関しては自由だが, あまりにもマニアックな言語は控えてください.\footnote{Haskellは大丈夫です. 私はc, c++, Pythonぐらいなら読めます.} ただし処理時間があまりにも長い場合は不正解とする. 処理時間の目安は2秒程度とする. 
\item[注意4.] この問題をプログラミングを用いて解答する場合に限り, その提出方法は皆さんにお任せいたします. 例えばgithub等にアップロードしてそのリンクをレポートに貼っても良いし, メールやCLEのダイレクトメッセージで, プログラムのソースファイルを直接私に送るなどでも良いです. プログラムのソースファイルを(スクリーンショット等で)画像にしてその画像をそのままレポートに貼っても良いです. \\
%\item[注意4.]  \underline{この問題(中間レポートおまけ問題(3))に限り, 提出方法は皆さんにお任せいたします.} (ただしプログラミングを用いて提出する場合のみ). github等にアップロードしてそのリンクをレポートに貼っても良いし, プログラムのソースファイルを(スクリーンショット等で)画像にしてその画像をそのままレポートに貼っても良いです(メールやCLEのダイレクトメッセージで, プログラムのソースファイルを直接私に送るなどでも良いです). \\
%目安として処理時間2秒程度とする.
\end{enumerate}

     \vspace{33pt} 
     
 \begin{flushright}
 {\LARGE 以上.}
 \end{flushright}



 

\end{document}
