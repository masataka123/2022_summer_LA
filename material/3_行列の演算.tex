\documentclass[dvipdfmx,a4paper,11pt]{article}
\usepackage[utf8]{inputenc}
%\usepackage[dvipdfmx]{hyperref} %リンクを有効にする
\usepackage{url} %同上
\usepackage{amsmath,amssymb} %もちろん
\usepackage{amsfonts,amsthm,mathtools} %もちろん
\usepackage{braket,physics} %あると便利なやつ
\usepackage{bm} %ラプラシアンで使った
\usepackage[top=30truemm,bottom=30truemm,left=25truemm,right=25truemm]{geometry} %余白設定
\usepackage{latexsym} %ごくたまに必要になる
\renewcommand{\kanjifamilydefault}{\gtdefault}
\usepackage{otf} %宗教上の理由でmin10が嫌いなので


\usepackage[all]{xy}
\usepackage{amsthm,amsmath,amssymb,comment}
\usepackage{amsmath}    % \UTF{00E6}\UTF{0095}°\UTF{00E5}\UTF{00AD}\UTF{00A6}\UTF{00E7}\UTF{0094}¨
\usepackage{amssymb}  
\usepackage{color}
\usepackage{amscd}
\usepackage{amsthm}  
\usepackage{wrapfig}
\usepackage{comment}	
\usepackage{graphicx}
\usepackage{setspace}
\usepackage{pxrubrica}
\setstretch{1.2}


\newcommand{\R}{\mathbb{R}}
\newcommand{\Z}{\mathbb{Z}}
\newcommand{\Q}{\mathbb{Q}} 
\newcommand{\N}{\mathbb{N}}
\newcommand{\C}{\mathbb{C}} 
\newcommand{\Sin}{\text{Sin}^{-1}} 
\newcommand{\Cos}{\text{Cos}^{-1}} 
\newcommand{\Tan}{\text{Tan}^{-1}} 
\newcommand{\invsin}{\text{Sin}^{-1}} 
\newcommand{\invcos}{\text{Cos}^{-1}} 
\newcommand{\invtan}{\text{Tan}^{-1}} 
\newcommand{\Area}{\text{Area}}
\newcommand{\vol}{\text{Vol}}




   %当然のようにやる.
\allowdisplaybreaks[4]
   %もちろん.
%\title{第1回. 多変数の連続写像 (岩井雅崇, 2020/10/06)}
%\author{岩井雅崇}
%\date{2020/10/06}
%ここまで今回の記事関係ない
\usepackage{tcolorbox}
\tcbuselibrary{breakable, skins, theorems}

\theoremstyle{definition}
\newtheorem{thm}{定理}
\newtheorem{lem}[thm]{補題}
\newtheorem{prop}[thm]{命題}
\newtheorem{cor}[thm]{系}
\newtheorem{claim}[thm]{主張}
\newtheorem{dfn}[thm]{定義}
\newtheorem{rem}[thm]{注意}
\newtheorem{exa}[thm]{例}
\newtheorem{conj}[thm]{予想}
\newtheorem{prob}[thm]{問題}
\newtheorem{rema}[thm]{補足}

\DeclareMathOperator{\Ric}{Ric}
\DeclareMathOperator{\Vol}{Vol}
 \newcommand{\pdrv}[2]{\frac{\partial #1}{\partial #2}}
 \newcommand{\drv}[2]{\frac{d #1}{d#2}}
  \newcommand{\ppdrv}[3]{\frac{\partial #1}{\partial #2 \partial #3}}


%ここから本文.
\begin{document}
%\maketitle
\begin{center}
{\Large 第3回. 行列の演算 (三宅先生の本, 1.2と1.3の内容)}
\end{center}

\begin{flushright}
 岩井雅崇 2022/04/28
\end{flushright}



\section{行列の和と差}

 \begin{tcolorbox}[
    colback = white,
    colframe = green!35!black,
    fonttitle = \bfseries,
    breakable = true]
    \begin{dfn}[行列の和と差]
    \text{}
 
$m \times n$行列
$
A=\begin{pmatrix}
a_{11}& a_{12} & \cdots &a_{1n} \\
a_{21}& a_{22} & \cdots &a_{2n} \\
\vdots& \vdots	&	\ddots   &	\vdots \\
a_{m1}& a_{m2} & \cdots &a_{mn} \\
\end{pmatrix}
$, 
$
B=\begin{pmatrix}
b_{11}& b_{12} & \cdots &b_{1n} \\
b_{21}& b_{22} & \cdots &b_{2n} \\
\vdots& \vdots	&	\ddots   &	\vdots \\
b_{m1}& b_{m2} & \cdots &b_{mn} \\
\end{pmatrix}
$
とする.

このとき行列の和$A+B$と差$A-B$を次で定める.
$$
A+B=
\begin{pmatrix}
a_{11}+b_{11}& a_{12}+b_{12}& \cdots &a_{1n} +b_{1n}\\
a_{21}+b_{21}& a_{22}+b_{22}& \cdots &a_{2n}+b_{2n} \\
\vdots& \vdots	&	\ddots   &	\vdots \\
a_{m1}+b_{m1}& a_{m2} +b_{m2}& \cdots &a_{mn} +b_{mn}\\
\end{pmatrix}.
$$
$$
A-B=
\begin{pmatrix}
a_{11}-b_{11}& a_{12}-b_{12}& \cdots &a_{1n} -b_{1n}\\
a_{21}-b_{21}& a_{22}-b_{22}& \cdots &a_{2n}-b_{2n} \\
\vdots& \vdots	&	\ddots   &	\vdots \\
a_{m1}-b_{m1}& a_{m2}-b_{m2}& \cdots &a_{mn}-b_{mn}\\
\end{pmatrix}.
$$
  \end{dfn}
 \end{tcolorbox}
 
 \begin{exa}
 $A = 
 \begin{pmatrix}
 1 &-2&8 \\
 2&5&-1
 \end{pmatrix}
 $, 
 $
 B = 
 \begin{pmatrix}
 -2&5&1 \\
 3&-1&2
 \end{pmatrix}
 $
 とする.
 
 このとき$
 A+B =
 \begin{pmatrix}
 -1 &3&9 \\
 5&4&1
 \end{pmatrix}
 $, 
 $
  A-B =
 \begin{pmatrix}
 3 &-7&7 \\
 -1&6&-3
 \end{pmatrix}
 $である.
 \end{exa}

\begin{exa}
 $A = 
 \begin{pmatrix}
 3&1 \\
 1&4
 \end{pmatrix}
 $, 
 $
 B = 
 \begin{pmatrix}
 2&7\\
 5&8
 \end{pmatrix}
 $
 とする.
 
 このとき$
 A+B =
 \begin{pmatrix}
 5&8 \\
6&12
 \end{pmatrix}
 $, 
 $
  A-B =
 \begin{pmatrix}
 1&-6 \\
 -4&-4
 \end{pmatrix}
 $である.
 \end{exa}
 
 \begin{exa}
 $A = 
 \begin{pmatrix}
 2&1 \\
 1&5
 \end{pmatrix}
 $,
$ 
 B = 
 \begin{pmatrix}
 1&1 &3 \\
 4&6 & 7
 \end{pmatrix}
 $
 とする.このとき$A+B$は型が違うため定義されない. 
 \end{exa}
 
 \begin{tcolorbox}[
    colback = white,
    colframe = green!35!black,
    fonttitle = \bfseries,
    breakable = true]
    \begin{prop}[行列の和と差の性質]
$A, B$を行列とする.
 \begin{itemize}
 \item $A\pm B =B\pm A$.
  \item $A\pm O =A$ (ただし$O$は零行列).
  \item $(A+B)+C =A + (B+C)$.
  \item ${}^{t}(A+B) = {}^{t}A+ {}^{t}B$.
 \end{itemize}
  \end{prop}
 \end{tcolorbox}
 
 \section{行列のスカラー倍}
 
  \begin{tcolorbox}[
    colback = white,
    colframe = green!35!black,
    fonttitle = \bfseries,
    breakable = true]
    \begin{dfn}[行列のスカラー倍]
    \text{}
    
 $m \times n$行列
 $
A=\begin{pmatrix}
a_{11}& a_{12} & \cdots &a_{1n} \\
a_{21}& a_{22} & \cdots &a_{2n} \\
\vdots& \vdots	&	\ddots   &	\vdots \\
a_{m1}& a_{m2} & \cdots &a_{mn} \\
\end{pmatrix}$
とし, $c$を数とする($c$をスカラーとも呼ぶ).

$A$の$c$倍$cA$を次で定める.
$$
cA=
\begin{pmatrix}
ca_{11}&c a_{12} & \cdots &ca_{1n} \\
ca_{21}& ca_{22} & \cdots &ca_{2n} \\
\vdots& \vdots	&	\ddots   &	\vdots \\
ca_{m1}& ca_{m2} & \cdots &ca_{mn} \\
\end{pmatrix}.
$$
  \end{dfn}
 \end{tcolorbox}

\begin{exa}
 $A = 
 \begin{pmatrix}
 1 &-2&8 \\
 2&5&-1
 \end{pmatrix}
 $,
 $
 c =3
 $
 とする.
 このとき$
 cA =
 \begin{pmatrix}
 3 &-6&24 \\
 6&15&-3
 \end{pmatrix}
 $である.
 \end{exa}
 \begin{exa}
 $A = 
 \begin{pmatrix}
 2&1 \\
 4&3
 \end{pmatrix}
 $, 
 $
 c =-1
 $
 とする.
 このとき$
 cA =
 \begin{pmatrix}
 -2 &-1 \\
-4&-3
 \end{pmatrix}
 $である.
 \end{exa}
 
 
 \begin{tcolorbox}[
    colback = white,
    colframe = green!35!black,
    fonttitle = \bfseries,
    breakable = true]
    \begin{prop}[行列のスカラー倍の性質]
$A$を行列, $a,b$を数とする.
 \begin{itemize}
 \item $0A =O$ (ただし$O$は零行列).
  \item $1A=A$. 
  \item $(-1)A$を$-A$と書くことにすると, $A + (-A) =O$. 
  \item $(ab) A = a(bA)$.
 \end{itemize}
  \end{prop}
 \end{tcolorbox}
 
 \section{行列の積}
 
  \begin{tcolorbox}[
    colback = white,
    colframe = green!35!black,
    fonttitle = \bfseries,
    breakable = true]
    \begin{dfn}[行列の積]
    
 $m \times n$行列$A = [a_{ij}]_{m \times n}$と$n \times l$行列$B= [b_{jk}]_{n \times l}$とする.
このとき$A$と$B$の積$AB$は$m \times l$行列で, 次の式で定義される.

$$
AB = [c_{ik}]_{m \times l}\text{としたとき, }
c_{ik} = a_{i1}b_{1k} + a_{i2}b_{2k} + \cdots + a_{in}b_{nk} = \sum_{j=1}^{n} a_{ij}b_{jk}.
$$
  \end{dfn}
 \end{tcolorbox}
 
 \begin{exa}
 $ A=\begin{pmatrix} 1 &2 &3 \end{pmatrix}$, 
 $ 
 B = 
 \begin{pmatrix}
5 \\7\\2
 \end{pmatrix}
 $
 とする. 
 
 $A$は$1\times 3$行列で$B$は$3 \times 1$行列なので, 行列の積$AB$が$1 \times 1$行列として定義でき, 
 $$
 AB = \begin{pmatrix}1 &2&3  \end{pmatrix}
 \begin{pmatrix}
5 \\7\\2
 \end{pmatrix}
 = \begin{pmatrix}1\times 5 + 2 \times 7 + 3 \times 2  \end{pmatrix}= 
  \begin{pmatrix}5+14+6 \end{pmatrix}= \begin{pmatrix}25 \end{pmatrix}.
 $$
 
 \end{exa}
 
  \begin{exa}
 $ A= 
 \begin{pmatrix}
2 & 2\\
4 & 3
 \end{pmatrix}
 $, $
 B = 
 \begin{pmatrix}
5 \\1
 \end{pmatrix}
 $
 とする. 
 
 $A$は$2\times 2$行列で$B$は$2 \times 1$行列なので, 行列の積$AB$が$2 \times 1$行列として定義でき, 
 $$
 AB = 
 \begin{pmatrix}
2 & 2\\
4 & 3
 \end{pmatrix}
  \begin{pmatrix}
5 \\1
 \end{pmatrix}
 =  
 \begin{pmatrix}
2\times 5 + 2\times 1 \\
4 \times 5 + 3 \times 1
 \end{pmatrix}
 = 
  \begin{pmatrix}
12 \\
23
 \end{pmatrix}.
 $$
 
 \end{exa}
 
 \begin{exa}
 $ A= 
 \begin{pmatrix}
2 & 3\\
1 & 4
 \end{pmatrix}
 $, $
 B = 
 \begin{pmatrix}
5 & 2\\
2 & 3
 \end{pmatrix}
 $
 とする. 
 
 $A$は$2\times 2$行列で$B$は$2 \times 2$行列なので, 行列の積$AB$が$2 \times 2$行列として定義でき, 
 $$
 AB = 
 \begin{pmatrix}
2 & 3\\
1 & 4
 \end{pmatrix}
 \begin{pmatrix}
5 & 2\\
2 & 3
 \end{pmatrix}
 =  
 \begin{pmatrix}
2 \times 5 + 3 \times 2& 2 \times 2 + 3 \times 3\\
1 \times 5 + 4 \times 2 & 1\times 2 + 4 \times 3
 \end{pmatrix}
 = 
 \begin{pmatrix}
16 & 13\\
13 & 14
 \end{pmatrix}.
 $$
 
また$B$は$2\times 2$行列で$A$は$2 \times 2$行列なので, 行列の積$BA$が$2 \times 2$行列として定義でき, 
 $$
 BA = 
  \begin{pmatrix}
5 & 2\\
2 & 3
 \end{pmatrix}
  \begin{pmatrix}
2 & 3\\
1 & 4
 \end{pmatrix}
 =
  \begin{pmatrix}
12 & 23\\
7 & 18
 \end{pmatrix}.
 $$

よって\underline{行列の積に関して$AB=BA$とは限らない($AB \neq BA$となることがある).}
 \end{exa}
 
  \begin{exa}
 $ A= 
 \begin{pmatrix}
2 & 1&-3\\
1 & -5 & 2
 \end{pmatrix}
 $, $
 B = 
  \begin{pmatrix}
8 & 7&5 & 2
 \end{pmatrix}
 $
 とする. 
 
 $A$は$2 \times 3$行列で$B$は$1 \times 4$行列であるので, 行列の積$AB$は定義されない.
 \end{exa}
 
 \begin{tcolorbox}[
    colback = white,
    colframe = green!35!black,
    fonttitle = \bfseries,
    breakable = true]
    \begin{prop}[行列の積の性質]
$A,B,C$を行列とする.
 \begin{itemize}
 \item $AO =O = OA$ (ただし$O$は零行列).
  \item $AE_{n}=E_{n}A =A$ (ただし$E_n$は単位行列). 
  \item $(AB)C = A(BC)$. 
  \item ${}^{t}(AB) = {}^{t}B {}^{t}A$
 \end{itemize}
  \end{prop}
 \end{tcolorbox}
\begin{itemize}
\item $A$を$n$次正方行列とするとき$A^{m} = \underbrace{A \cdots A}_{m \text{ 個}}$とする
\item $A^{m}=O$となる行列を\underline{\ruby{冪}{べき}\ruby{零}{ぜろ}行列}という.
\end{itemize}

 \begin{tcolorbox}[
    colback = white,
    colframe = green!35!black,
    fonttitle = \bfseries,
    breakable = true]
    \begin{prop}[行列の演算の性質]
$A,B,C$を行列とし, $a,b$を数とする.
 \begin{itemize}
 \item $a(AB)=(aA)B$. 
  \item $a(A+B)=aA + aB$. 
  \item $(a+b)A = aA + bA$. 
  \item $A(B+C) = AB + AC$.
  \item $(A+B)C = AC + BC$.
 \end{itemize}
  \end{prop}
 \end{tcolorbox}
 
\section{三宅先生の本1.3の内容に関して}
この授業では三宅先生の本1.3の内容「行列の分割」についての説明は割愛する(重要度が低いと思われるため).
ただし証明等で行列の分割の記法を用いるため, 各自で三宅先生の本1.3の内容を読むことをお勧めする.

\section{演習問題}
演習問題の解答は授業動画にあります.

1.  次の行列の計算を行え.
 $$
 \begin{pmatrix}
 2 &3&-1 \\
 0&5&4\\
 -1&0&-2
 \end{pmatrix}
 \left\{
 \begin{pmatrix}
 0 &5&9 \\
 3&-2&8\\
 -1&8&1
 \end{pmatrix}
 - 2
  \begin{pmatrix}
 -1 &0&1 \\
 3&2&3\\
 -4&2&-1
 \end{pmatrix}
\right\}
 $$
  
 2. 次の行列$A,B,C,D$のうち, 積が定義される全ての組み合わせを求め, その積を計算せよ.
 $$
  A=\begin{pmatrix}
 2 \\ 1\\-1
 \end{pmatrix} 
B= \begin{pmatrix}
 3 &2\\
 4&1\\
 0&1
 \end{pmatrix} 
 C=
  \begin{pmatrix}
 2 &0&1 
 \end{pmatrix}
 D= \begin{pmatrix}
 2&3\\
 -1&4
 \end{pmatrix}
 $$


 

\end{document}
