% [beamer を使うために] by J.Goto (Jul.14,2008+Nov.03,2011)
%
% 0. まず, TeXがインストールされていることが前提となります. 
%    TeXがインストールされていない場合は, まずTeX(pLaTeXなど)をインストールして下さい。
% 1. 次に, https://sourceforge.net/projects/latex-beamer/ から必要なファイルをダウンロード&展開します.
%    とりあえず, latex-beamer, pgf, xcolor(フォルダ)をダウンロードして, 
%    適当な場所に適当なファイル解凍ソフトで展開します.
%
%    ## 2011年11月3日現在、上記にあるヴァージョンはやや古いようです。
%    ## 上記のものではなく、下記のダウンロード先を参照してみてください。
%    ## 【beamer本体】http://www.ctan.org/tex-archive/macros/latex/contrib/beamer/
%    ## 【pgf】       http://sourceforge.net/projects/pgf/files/pgf/version%202.00/pgf-2.00.tar.gz/download
%    ## 【xcolor】    http://www.ctan.org/tex-archive/macros/latex/contrib/xcolor/
%
% 2. 展開されたlatex-beamer, pgf, xcolorの3つのフォルダを適当なフォルダ(/texmf/tex/latex/)に置きます. 
%    詳しくは http://sourceforge.net/docman/display_doc.php?docid=19464&group_id=92412 を参照して下さい.
%    ちなみに, 僕の場合は, TeXがインストールされているフォルダを/tex/とすると, 
%    /tex/share/texmf/tex/latex/に3つのフォルダを置きました.
% 3. 後はTeXと同じように.texファイルを作成し, コンパイルして使います. 
%    もちろん, クラスファイルの読み込みや, その入力ルールに従う必要があります.(以下を参考)
%
% -- beamerを初めて使うときは, これより下の部分を, 適当に置き換えながら使ってみましょう --

\documentclass[11pt,dvipdfmx]{beamer}

%%% 使用するテーマ(現在Copenhagenを指定しています。適当に変えてみよう。)%%%
%
\usetheme{Antibes}  
%\usetheme{Default}       % シンプルで機能的なテーマ #何も指定しない場合は自動的にこれ
%\usetheme{Bergen}        % フレームを縦方向に分割
%\usetheme{Boadilla}      % より多くの情報を収容可
%\usetheme{Madrid}        % Boadillaをよりカラフルにしたもの
%\usetheme{Pittsburgh}    % シンプルで機能的、見出しは右寄せ
%\usetheme{Rochester}     % 横方向のヘッダパネルが特徴     % 上部にナビゲーションバーを持つ、明瞭度の高いテーマ
%\usetheme{JuanLesPins}   % Antibesと類似のテーマ
%\usetheme{Montpellier}   % シンプルで色調のおとなしいもの
%\usetheme{Berkeley}      % 横方向のヘッダパネルを持つ機能的なテーマ
%\usetheme{PaloAlto}      % Berkeleyと類似のテーマ
%\usetheme{Goettingen}    % サイドバーは右側で、ヘッダパネルなし
%\usetheme{Marburg}       % Goettingenの色調を強くしたもの
%\usetheme{Hannover}      % サイドバーは左側で見出しは右寄せ
%\usetheme{Berlin}        % 縦方向のナビゲーションバーを上部に持つ強い色調のテーマ
%\usetheme{Ilmenau}       % Berlinと類似のテーマ
%\usetheme{Dresden}       % Ilmenauと類似のテーマ
%\usetheme{Darmstadt}     % 横方向のナビゲーションバーを上部に持つ
%\usetheme{Frankfurt}     % Darmstadtと類似、しかしサブセクション情報は含まない
%\usetheme{Singapore}     % ソフトな色調を持ったテーマ
%\usetheme{Szeged}        % Singaporeと類似、しかし境界線は明確
%\usetheme{Copenhagen}    % セクション/サブセクションテーブルを上部に配置
%\usetheme{Luebeck}       % Copenhagenから丸みを取ったもの
%\usetheme{Malmoe}        % Copenhagenをより質素にしたもの
\usetheme{Warsaw}        % Copenhagenと類似のテーマ

%%% 数式のフォント(TeXっぽいフォントになる)%%%
%
\usefonttheme{professionalfonts}


%%% 隠蔽されている要素の透明度の設定 %%%
%
%\setbeamercovered{transparent=10}

%%% その他のクラス・パッケージの導入 %%%

\usepackage{graphicx}  % includegraphicsコマンドなどで図を表示するためのクラス
\usepackage{amsmath}   % プロ仕様の数学用のフォントI(AMSはアメリカ数学会)
\usepackage{amssymb}   % プロ仕様の数学用のフォントII(AMSはアメリカ数学会)
\usepackage{bm}        % 太字を表現するのに便利なクラス
\usepackage[absolute,overlay]{textpos}


\usepackage[all]{xy}
\usepackage{amsthm,amsmath,amssymb,comment}
\usepackage{float}
\usepackage{graphicx}
%% ゴシック体にする
%\renewcommand{\kanjifamilydefault}{gt}

% フォントはお好みで
%\usepackage{txfonts}
%\mathversion{bold}                             %%% 数式を太字にする
\renewcommand{\familydefault}{\sfdefault}
\renewcommand{\kanjifamilydefault}{\gtdefault} %%% 日本語フォントを太字にする
%\setbeamerfont{title}{size=\large,series=\bfseries}
%\setbeamerfont{frametitle}{size=\large,series=\bfseries}
%\setbeamertemplate{frametitle}[default][center]
%\usefonttheme{professionalfonts} 
%
\newtheorem{proposition}{命題}
\newtheorem{assumption}{仮定}
\newtheorem{cor}{系}
\newtheorem{remark}{Remark}
\newtheorem{exercise}{Exercise}

\newtheorem{thm}{Theorem}[section] 
\newtheorem{theo}[thm]{Theorem}
\newtheorem{corr}[thm]{Corollary}
\newtheorem{prop}[thm]{Proposition}
\newtheorem{conj}[thm]{Conjecture}
\newtheorem*{mainthm}{Theorem}
\newtheorem{deflem}[thm]{Definition-Lemma}
\newtheorem{lem}[thm]{Lemma}
\theoremstyle{definition} 
\newtheorem{defn}[thm]{Definition}
\newtheorem{propdefn}[thm]{Proposition-Definition} 
\newtheorem{lemdefn}[thm]{Lemma-Definition} 
\newtheorem{thmdefn}[thm]{Theorem-Definition} 
\newtheorem{eg}[thm]{Example} 
\newtheorem{ex}[thm]{Example} 
\theoremstyle{remark}
\newtheorem{rem}[thm]{Remark}
\newtheorem{obs}[thm]{Observation}
\newtheorem{ques}[thm]{Question}
%\newtheorem{problem}[thm]{Problem}
\newtheorem{setup}[thm]{Set up}
\newtheorem{notation}[thm]{Notation}
\newtheorem{cl}{Claim}
\newtheorem{claim}{Claim}
\newtheorem{step}{Step}
\newtheorem*{clproof}{Proof of Claim}
\newtheorem{cln}[thm]{Claim}
\newtheorem*{ack}{Acknowledgements} 



%% 自分で定義したマクロ
\newcommand{\Sym}{{\rm Sym}}

\newcommand{\Sigmat}{\mbox{\boldmath\ensuremath{\Sigma}}}
\newcommand{\evec}{\mbox{\boldmath\ensuremath{e}}}
\newcommand{\ovec}{\mbox{\boldmath\ensuremath{0}}}
\newcommand{\xvec}{\mbox{\boldmath\ensuremath{x}}}
\newcommand{\R}{{\rm I\!R}}
\newcommand{\e}{{\rm e}}
\newcommand{\dr}{{\rm d}}
\newcommand{\E}{{\mathbb E}}
\newcommand{\p}{{\mathbb P}}
\newcommand{\V}{{\mathbb V}}


\title[2022年度春夏学期 木曜2限  線形代数学I]{大阪大学 2022年度春夏学期 \\ 全学共通教育科目 木曜2限  \\ 線形代数学I (理(生物・生命(化・生)))}
\subtitle{第1回 授業ガイダンス}
\author[岩井雅崇]{岩井雅崇}
%% 所属の登録 \institute[所属の略称]{所属}
\institute[大阪大学]{大阪大学}
%% 日付
\date{2022年4月14日}  %% <- \today 命令は今日の日付を表示. 任意の日付を入れれば良い:(例)\date{2008年7月14日}

%%% 以下が本体 %%%

\begin{document}

%%%%%%%%%%%%%%%%%%%%%%%%%%%%%%%%%%%%%%%%%%%%%%%%%%%%%%%%%%%%%%%%%%%%%%%%
%%%%%%%%%%%%%%%%%%%%%%%%%%%%%%%%%%%%%%%%%%%%%%%%%%%%%%%%%%%%%%%%%%%%%%%%
%% タイトルページ出力 %%

\begin{frame}  %% <- \begin{frame} から \end{frame}までが1つの頁になると思って下さい.
 \titlepage    %% <- このコマンドで自動的に表紙のページが作成されます.
\end{frame}

% [remark]
% \begin{frame} *** \end{frame} の代わりに
% \frame { *** } としても同じです(次のページの書き方を参考にして下さい).

%%%%%%%%%%%%%%%%%%%%%%%%%%%%%%%%%%%%%%%%%%%%%%%%%%%%%%%%%%%%%%%%%%%%%%%%
%%%%%%%%%%%%%%%%%%%%%%%%%%%%%%%%%%%%%%%%%%%%%%%%%%%%%%%%%%%%%%%%%%%%%%%%
%% 目次ページ

%%\begin{frame}  %% <- この書き方でもOK.
%% \tableofcontents   %% <- このコマンド1つで自動的に目次のページが作成されます.
%%\end{frame}

%%%%%%%%%%%%%%%%%%%%%%%%%%%%%%%%%%%%%%%%%%%%%%%%%%%%%%%%%%%%%%%%%%%%%%%%
%%%%%%%%%%%%%%%%%%%%%%%%%%%%%%%%%%%%%%%%%%%%%%%%%%%%%%%%%%%%%%%%%%%%%%%%

\section{ }
\begin{frame}
\frametitle{この授業について}
 \begin{itemize}

\item みなさんご入学おめでとうございます. 
 \item この授業は"大阪大学 2022年度春夏学期 全学共通教育科目 木曜2限 線形代数学I (理(生物・生命(化・生)))"です.
 \item 担当教官は岩井雅崇(いわいまさたか)です.
 \item この授業でやることは"線形代数"です.
 
 \end{itemize}
\end{frame}

\begin{frame}
\frametitle{線形代数とは?}

   \begin{alertblock}{}
  \begin{center}
 線形代数とは行列を扱った分野.
  \end{center}
 \end{alertblock}
 
 行列とは数を長方形に並べたもの.
 
 $$
  \begin{pmatrix}
 1 &2&5 \\
 3&10&4
 \end{pmatrix}
  \begin{pmatrix}
 13 &2&5&3 \\
 1 &4&2&5 \\
  7&8&6&1 
 \end{pmatrix}
  \begin{pmatrix}
0 &0&0\\
0 &0&0
 \end{pmatrix}
    \begin{pmatrix}
2&0 &0\\
0 &1 &0\\
0&0&5
 \end{pmatrix}
 $$
\end{frame}

\begin{frame}
\frametitle{なぜ線形代数を学ぶのか?}

%\vspace{-20pt}
   \begin{alertblock}{}
  \begin{center}
世の中は行列だらけ!
  \end{center}
 \end{alertblock}
 物体認識カメラとかgoogleのおすすめ検索とか....
 
 \vspace{20pt}
      \begin{alertblock}{}
  \begin{center}
この世の中は行列に支配されつつある.
  \end{center}
 \end{alertblock}
   機械学習, 深層学習, 人工知能, AIなどに行列が使われているから.
  %\begin{itemize}
  %\item 微積分や線型代数とか使って, 多くの理論ができている.
  %\item  何らかのシミュレーションするとき, 偏微分方程式を使うから, 微積分の知識が必要.
 % \item (最近の流行の) 機械学習, 深層学習, 人工知能, AI (etc...)は微積分と行列が多く出てくる.
  %{\tiny (Pythonのnumpyとか行列の記法だし...)}
 
  %\end{itemize}

\end{frame}

%\end{frame}

\begin{frame}
\frametitle{授業の内容 (シラバスとの変更点)}
\begin{enumerate}
\item ガイダンス (Zoomを用いた遠隔授業)
\item 行列の定義
\item 行列の演算
\item 連立一次方程式 1 -基本変形-
\item 連立一次方程式 2 -行列の簡約化-
\item 連立一次方程式 3 -一般的な解法-
\item 正則行列
\item まとめと復習・質疑応答 (Zoomを用いた遠隔授業)
\item 行列式 1 -置換-
\item 行列式 2 -行列式の計算方法-
\item 行列式 3 -行列式の基本性質-
\item 余因子行列と余因子展開
\item クラメルの公式と特殊な行列式
\item 内積と外積
\item まとめと復習・質疑応答(Zoomを用いた遠隔授業)
\end{enumerate}


\end{frame}

\begin{frame}
\frametitle{成績の付け方}
 \begin{itemize}
 \item 中間レポートと期末試験のみで評価する. 
 %どちらも問題は4問(小問有り)ずつである.
 \item 出席点はないので安心してください.
 %\item 合計8問(小問あり). %計120点満点. (ただし1問20点ではない)
% \item %単位が欲しいだけの人も,  全6問を解くことをお勧めします. %(かなり減点があると思うので...)
 \item (奨学金などの申請で)より良い成績がほしい方は, レポートや試験にある"おまけ問題"を解いても良い. \\
 単位が欲しいだけの人は"おまけ問題"を解かなくて良い. \\
% \item 単位が欲しいだけの人は"おまけ問題"を解かなくて良い. 6問は絶対に解くこと! \\
 {\tiny 例えばレポートで79点(良)とった人がおまけ問題を正答してた場合, 成績には80点(優)つける考慮をします.
 レポートで59点(不可)とった人がおまけ問題を正答してても, 成績が60点(可)になることはない. }
{\tiny  おまけ問題を解かなくても90点以上の成績をつけることもあります.}
 \end{itemize}


\end{frame}

\begin{frame}
\frametitle{中間レポートと期末試験の内容(予定)}
    \begin{alertblock}{中間レポートの内容(予定)}
 \begin{itemize}
\item 行列の定義・演算
\item 連立1次方程式
 \end{itemize}
 %おまけ問題は授業に関係ない問題を出す予定.
 \end{alertblock}

\begin{alertblock}{期末試験の内容(予定)}
 \begin{itemize}
\item 行列の演算・連立1次方程式 (少なめに出す)
\item 行列式・逆行列 (こっちがメイン)
 \end{itemize}
 おまけ問題に内積・外積に関する問題を出す予定.
  \end{alertblock}
{\small
これらを出題する理由は, この授業を通して以下を学んでほしいから.
\begin{itemize}
\item 行列の定義と演算(特に掛け算).
\item 行列の基本変形を用いた連立1次方程式の解きかた(掃き出し法).
\item 逆行列・行列式の求め方.
\item 逆行列・行列式の理論.
\end{itemize}
}

 
 \end{frame}

%%%%%%%%%%%%%%%%%%%%%%%%%%%%%%%%
\begin{comment}

\begin{frame}
\frametitle{中間レポートの内容(予定)}
 \begin{itemize}
\item 微分法と初等関数の性質 (第3回)
\item 平均値の定理と関数の極限値計算 (第4回)
\item 高次導関数とテイラーの定理 (第5回)
\item 漸近展開とべき級数展開 (第6回)
 \end{itemize}
 
 おまけ問題は, 実数の定義と性質(第1回), 連続関数(第2回), 高次導関数とテイラーの定理 (第5回)の予定.
   \begin{alertblock}{}
  \begin{center}
予定なので変更の可能性もあります!
  \end{center}
 \end{alertblock}
 
 \end{frame}
 
 \begin{frame}
\frametitle{期末試験の内容(予定)}
 \begin{itemize}
\item 積分の性質 (第9回)
\item 不定積分の計算方法 (第10回)
\item 広義積分 (第11回)
 \end{itemize}
 
 おまけ問題は広義積分 (第11回)の予定.
 
   \begin{alertblock}{}
  \begin{center}
予定なので変更の可能性もあります!
  \end{center}
 \end{alertblock}
 


\end{frame}
\end{comment}
%%%%%%%%%%%%%%%%%%%%%%%%%%%%%%%%%%%%

\begin{frame}
\frametitle{授業の進め方・みなさんの学び方}
この授業はメディア授業です. %私からは授業資料と授業動画を配布いたします.
%まとめと復習のみ遠隔授業をします.(質問タイムみたいな感じです)

%{\tiny (私の家のwifiが微弱でして.....)}
 授業ホームページに授業資料・授業動画・授業の板書をアップロードしていきます(CLEにはアップロードしません).
 \begin{alertblock}{}
  \begin{center}
学び方は皆さんにお任せします.
  \end{center}
 
 \end{alertblock}
 例えば以下の方法など挙げられます.
  \begin{itemize}
    \item 私が作った動画を見て, 私の資料や教科書で復習する.  \\
  \item 教科書を読んで, 私の資料で復習する. \\
  \item 教科書を用いて勉強する. \\ (今回やるのはたった62ページの内容!)
  \item その他, 自己流で勉強する.
  \end{itemize}

  \begin{alertblock}{}
  \begin{center}
最終的にレポートや試験でだす問題を \\ 解けるぐらい理解をすればOKです.
  \end{center}
 \end{alertblock}


\end{frame}

\begin{frame}
\frametitle{最後に}
 \begin{itemize}

 \item 予期しないことで授業の形態が変わる可能性があります.
 授業ホームページはこまめにチェックしてください.
 
 \item  遠隔授業をする際には事前に授業ホームページにてお知らせします.
Zoomのリンク等はCLEでお知らせします.

 \item 中間レポートの提出はCLEを用いて行う予定です. \\
 {\scriptsize (今年着任してきたので, 私はCLEの使いかたをよくわかってないですが...)}
 
\item 質問に関しては, メールやCLEのメッセージでも対応いたします(対面での質問も一応可能です).

  \begin{alertblock}{}
  \begin{center}
無理のないように自分のペースで理解をしていってください. \\
%また体調が悪い場合は無理せずお休みしても構いません. \\
%(出席点はないので無理する必要はございません.)
  \end{center}
 \end{alertblock}


 \end{itemize}

\end{frame}

\end{document}