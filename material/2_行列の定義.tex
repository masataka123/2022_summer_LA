\documentclass[dvipdfmx,a4paper,11pt]{article}
\usepackage[utf8]{inputenc}
%\usepackage[dvipdfmx]{hyperref} %リンクを有効にする
\usepackage{url} %同上
\usepackage{amsmath,amssymb} %もちろん
\usepackage{amsfonts,amsthm,mathtools} %もちろん
\usepackage{braket,physics} %あると便利なやつ
\usepackage{bm} %ラプラシアンで使った
\usepackage[top=30truemm,bottom=30truemm,left=25truemm,right=25truemm]{geometry} %余白設定
\usepackage{latexsym} %ごくたまに必要になる
\renewcommand{\kanjifamilydefault}{\gtdefault}
\usepackage{otf} %宗教上の理由でmin10が嫌いなので


\usepackage[all]{xy}
\usepackage{amsthm,amsmath,amssymb,comment}
\usepackage{amsmath}    % \UTF{00E6}\UTF{0095}°\UTF{00E5}\UTF{00AD}\UTF{00A6}\UTF{00E7}\UTF{0094}¨
\usepackage{amssymb}  
\usepackage{color}
\usepackage{amscd}
\usepackage{amsthm}  
\usepackage{wrapfig}
\usepackage{comment}	
\usepackage{graphicx}
\usepackage{setspace}
\usepackage{pxrubrica}
\setstretch{1.2}


\newcommand{\R}{\mathbb{R}}
\newcommand{\Z}{\mathbb{Z}}
\newcommand{\Q}{\mathbb{Q}} 
\newcommand{\N}{\mathbb{N}}
\newcommand{\C}{\mathbb{C}} 
\newcommand{\Sin}{\text{Sin}^{-1}} 
\newcommand{\Cos}{\text{Cos}^{-1}} 
\newcommand{\Tan}{\text{Tan}^{-1}} 
\newcommand{\invsin}{\text{Sin}^{-1}} 
\newcommand{\invcos}{\text{Cos}^{-1}} 
\newcommand{\invtan}{\text{Tan}^{-1}} 
\newcommand{\Area}{\text{Area}}
\newcommand{\vol}{\text{Vol}}




   %当然のようにやる.
\allowdisplaybreaks[4]
   %もちろん.
%\title{第1回. 多変数の連続写像 (岩井雅崇, 2020/10/06)}
%\author{岩井雅崇}
%\date{2020/10/06}
%ここまで今回の記事関係ない
\usepackage{tcolorbox}
\tcbuselibrary{breakable, skins, theorems}

\theoremstyle{definition}
\newtheorem{thm}{定理}
\newtheorem{lem}[thm]{補題}
\newtheorem{prop}[thm]{命題}
\newtheorem{cor}[thm]{系}
\newtheorem{claim}[thm]{主張}
\newtheorem{dfn}[thm]{定義}
\newtheorem{rem}[thm]{注意}
\newtheorem{exa}[thm]{例}
\newtheorem{conj}[thm]{予想}
\newtheorem{prob}[thm]{問題}
\newtheorem{rema}[thm]{補足}

\DeclareMathOperator{\Ric}{Ric}
\DeclareMathOperator{\Vol}{Vol}
 \newcommand{\pdrv}[2]{\frac{\partial #1}{\partial #2}}
 \newcommand{\drv}[2]{\frac{d #1}{d#2}}
  \newcommand{\ppdrv}[3]{\frac{\partial #1}{\partial #2 \partial #3}}


%ここから本文.
\begin{document}
%\maketitle
\begin{center}
{\Large 第2回. 行列の定義 (三宅先生の本, 1.1の内容)}
\end{center}

\begin{flushright}
 岩井雅崇 2022/04/21
\end{flushright}



\section{行列の定義}

\begin{itemize}
\item $m \times n$個の数(実数または複素数) $a_{ij}$ ($i = 1, \ldots, m$, $j = 1, \ldots, n$)を
$$
\begin{bmatrix}
a_{11}& a_{12} & \cdots &a_{1n} \\
a_{21}& a_{22} & \cdots &a_{2n} \\
\vdots& \vdots	&	\ddots   &	\vdots\\
a_{m1}& a_{m2} & \cdots &a_{mn} \\
\end{bmatrix}
\textit{\,\,\ または\,\,\,}
\begin{pmatrix}
a_{11}& a_{12} & \cdots &a_{1n} \\
a_{21}& a_{22} & \cdots &a_{2n} \\
\vdots& \vdots	&	\ddots   &	\vdots \\
a_{m1}& a_{m2} & \cdots &a_{mn} \\
\end{pmatrix}
$$
のように並べたものを \underline{$m$行$n$列の行列}という.
\underline{ $m \times n$行列}, \underline{ $m \times n$型の行列}, \underline{ $(m, n)$行列}ということもある. 
\item 上の行列を$A$としたとき, $a_{ij}$を行列$A$の$(i,j)$成分という. 行列$A$を\underline{$[a_{ij}]_{m\times n}$や$(a_{ij})$}と略記することもある.
\item $\begin{pmatrix} a_{i1} & \cdots & a_{in}\end{pmatrix}$を\underline{$A$の行}といい, 上から第1行, 第2行, $\cdots$, 第$m$行という.
\item $\begin{pmatrix}a_{1j} \\ \vdots  \\ a_{mj}\end{pmatrix}$を\underline{$A$の列}といい, 上から第1列, 第2列, $\cdots$, 第$n$列という.
\item $1 \times n$行列$(a_{11} \cdots a_{1n})$を\underline{行ベクトル}と呼び, $m \times 1$行列$\begin{pmatrix}a_{11} \\ \vdots  \\ a_{m1}\end{pmatrix}$を\underline{列ベクトル}と呼ぶ(この授業や教科書での用語).
\end{itemize}

 \begin{exa}
 行列$A$を次で定める.
 $$
 A = 
 \begin{pmatrix}
 1 &2&5 \\
 3&10&4
 \end{pmatrix}
 $$
 \begin{itemize}
 \item $A$は2行3列の行列($2 \times 3$行列).
 \item (1,2)成分は2, (2,1)成分は3, (2,3)成分は4である.
 \item 第2行は$\begin{pmatrix}3 & 10 & 4\end{pmatrix}$. 第3列は$\begin{pmatrix}5 \\ 4\end{pmatrix}$である.
 \end{itemize}
 \end{exa}
  \begin{exa}
 行列$A$を次で定める.
 $$
 A = 
 \begin{pmatrix}
 13 &2&5&3 \\
 1 &4&2&5 \\
  7&8&6&1 
 \end{pmatrix}
 $$
 \begin{itemize}
 \item $A$は3行4列の行列($3 \times 4$行列).
 \item (1,1)成分は13, (2,4)成分は5, (3,2)成分は8である.
 \item 第2行は$\begin{pmatrix}1&4&2&5\end{pmatrix}$. 第3列は$\begin{pmatrix}5 \\2\\ 6\end{pmatrix}$である.
 \end{itemize}
 \end{exa}
 
 \begin{exa}行列$A = (2)$とすると, $A$は1行1列の行列($1 \times 1$行列)である.\end{exa}
  
%\begin{exa}行列$A$を次で定める.$$A = (3,5,2)$$$A$は1行3列の行列($1 \times 3$行列)である.\end{exa}

\section{特別な行列}
\begin{itemize}
\item  $
 \begin{pmatrix}
0 &0&0\\
0 &0&0
 \end{pmatrix}, 
  \begin{pmatrix}
0 &0\\
0 &0
 \end{pmatrix}, 
   \begin{pmatrix}
0 
 \end{pmatrix}, 
   \begin{pmatrix}
0 &0\\
0 &0 \\
0&0
 \end{pmatrix}
 $
 のように全ての成分が0の行列を\underline{\ruby{零}{ぜろ}行列}という.
 \item $n \times n$行列のことを\underline{$n$次正方行列}という.
 \item $n$次正方行列
 $$
 A =
 \begin{pmatrix}
a_{11}& a_{12} & \cdots &a_{1n} \\
a_{21}& a_{22} & \cdots &a_{2n} \\
\vdots& \vdots	&	\ddots   &	\vdots \\
a_{n1}& a_{n2} & \cdots &a_{nn} \\
\end{pmatrix}
 $$
 について, $a_{11}, a_{22}, \ldots, a_{nn}$を\underline{$A$の対角成分}という.
 \item 対角成分以外0の行列を\underline{対角行列}という. 例えば以下の行列は対角行列である:
  $$
 \begin{pmatrix}
5&0\\
0 &1
 \end{pmatrix}, 
  \begin{pmatrix}
3
 \end{pmatrix}, 
   \begin{pmatrix}
2&0 &0\\
0 &1 &0\\
0&0&5
 \end{pmatrix}
 $$
 \item 対角成分が全て1な$n$次対角行列を\underline{単位行列}と言い, $E_n$とかく. 例えば以下の行列は単位行列である:
   $$
   E_2 =
 \begin{pmatrix}
1&0\\
0 &1
 \end{pmatrix}, 
 E_1=
  \begin{pmatrix}
1
 \end{pmatrix}, 
 E_3=
   \begin{pmatrix}
1&0 &0\\
0 &1 &0\\
0&0&1
 \end{pmatrix}
 $$
 \item 行列$A$の行と列を入れ替えた行列を\underline{転置行列}と言い${}^{t}A$とかく.
\end{itemize}
\begin{exa}
 $
 A = 
 \begin{pmatrix}
 2 &5&8 \\
 3&2&1
 \end{pmatrix}
 \text{ についてその転置行列は}
{}^{t}A = 
 \begin{pmatrix}
 2 &3 \\
 5&2\\
 8&1
 \end{pmatrix}
  \text{であり, }
 $
 $
{}^{t}({}^{t}A) = 
 \begin{pmatrix}
 2 &5&8 \\
 3&2&1
 \end{pmatrix}
 =A
 $
 である.
 
  $
 A = 
 \begin{pmatrix}
 3 &1 \\
 2&4
 \end{pmatrix}
 \text{ についてその転置行列は}
{}^{t}A = 
 \begin{pmatrix}
 3 &2 \\
 1&4\\
 \end{pmatrix}
  \text{であり, }
 $
  $
{}^{t}({}^{t}A) = 
 \begin{pmatrix}
 3 &1 \\
 2&4
 \end{pmatrix}
 =A
 $
 である.
\end{exa}

 \begin{tcolorbox}[
    colback = white,
    colframe = green!35!black,
    fonttitle = \bfseries,
    breakable = true]
    \begin{prop}[転置行列の性質]
    \text{}
\begin{itemize}
\item $A$が$m\times n$行列なら${}^{t}A $は$n\times m$行列.
\item $A=[a_{ij}]_{m \times n}$とし, ${}^{t}A=[b_{ij}]_{n \times m}$とするとき, $b_{ij} =a_{ji}$.
\item ${}^{t}({}^{t}A) =A$.
\end{itemize}

  \end{prop}
 \end{tcolorbox}
 
 \section{クロネッカーのデルタ}
 $$
 \delta_{ij} = 
 \begin{cases}
1 & \text{$i=j$のとき}\\
0 & \text{$i \neq j$のとき}
\end{cases}
  $$
 \text{を\underline{クロネッカーのデルタ}という.}
 \begin{exa}
 $\delta_{11}=\delta_{22}=1, \delta_{12}=\delta_{21}=0$である. $n$次正方行列$E_n$は$E_n = [\delta_{ij}]_{n\times n}$と略記できる.
 \end{exa}

\section{演習問題}
演習問題の解答は授業動画にあります.

1.  行列$A$を次で定める.
 $$
 A = 
 \begin{pmatrix}
 2 &4&-3&8 \\
 3&-1&2&-5 \\
  18&0&2&12
 \end{pmatrix}
 $$
 次の問いに答えよ.
 \begin{itemize}
 \item $A$の型をいえ.
 \item $A$の$(3,2)$成分をいえ.
  \item $A$の第2行をいえ.
 \item $A$の第3列をいえ.
 \item $A$の転置行列${}^{t}A$を求めよ.
 \end{itemize}
 
 2. ${}^{t}A =-A$となる$n$次正方行列を交代行列という. 交代行列の対角成分は0であることを示せ.



 

\end{document}
