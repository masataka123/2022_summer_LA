\documentclass[dvipdfmx,a4paper,11pt]{article}
\usepackage[utf8]{inputenc}
%\usepackage[dvipdfmx]{hyperref} %リンクを有効にする
\usepackage{url} %同上
\usepackage{amsmath,amssymb} %もちろん
\usepackage{amsfonts,amsthm,mathtools} %もちろん
\usepackage{braket,physics} %あると便利なやつ
\usepackage{bm} %ラプラシアンで使った
\usepackage[top=30truemm,bottom=30truemm,left=25truemm,right=25truemm]{geometry} %余白設定
\usepackage{latexsym} %ごくたまに必要になる
\renewcommand{\kanjifamilydefault}{\gtdefault}
\usepackage{otf} %宗教上の理由でmin10が嫌いなので


\usepackage[all]{xy}
\usepackage{amsthm,amsmath,amssymb,comment}
\usepackage{amsmath}    % \UTF{00E6}\UTF{0095}°\UTF{00E5}\UTF{00AD}\UTF{00A6}\UTF{00E7}\UTF{0094}¨
\usepackage{amssymb}  
\usepackage{color}
\usepackage{amscd}
\usepackage{amsthm}  
\usepackage{wrapfig}
\usepackage{comment}	
\usepackage{graphicx}
\usepackage{setspace}
\usepackage{pxrubrica}
\setstretch{1.2}


\newcommand{\R}{\mathbb{R}}
\newcommand{\Z}{\mathbb{Z}}
\newcommand{\Q}{\mathbb{Q}} 
\newcommand{\N}{\mathbb{N}}
\newcommand{\C}{\mathbb{C}} 
\newcommand{\Sin}{\text{Sin}^{-1}} 
\newcommand{\Cos}{\text{Cos}^{-1}} 
\newcommand{\Tan}{\text{Tan}^{-1}} 
\newcommand{\invsin}{\text{Sin}^{-1}} 
\newcommand{\invcos}{\text{Cos}^{-1}} 
\newcommand{\invtan}{\text{Tan}^{-1}} 
\newcommand{\Area}{\text{Area}}
\newcommand{\vol}{\text{Vol}}
\newcommand{\maru}[1]{\raise0.2ex\hbox{\textcircled{\tiny{#1}}}}
\newcommand{\sgn}{{\rm sgn}}
%\newcommand{\rank}{{\rm rank}}



   %当然のようにやる.
\allowdisplaybreaks[4]
   %もちろん.
%\title{第1回. 多変数の連続写像 (岩井雅崇, 2020/10/06)}
%\author{岩井雅崇}
%\date{2020/10/06}
%ここまで今回の記事関係ない
\usepackage{tcolorbox}
\tcbuselibrary{breakable, skins, theorems}

\theoremstyle{definition}
\newtheorem{thm}{定理}
\newtheorem{lem}[thm]{補題}
\newtheorem{prop}[thm]{命題}
\newtheorem{cor}[thm]{系}
\newtheorem{claim}[thm]{主張}
\newtheorem{dfn}[thm]{定義}
\newtheorem{rem}[thm]{注意}
\newtheorem{exa}[thm]{例}
\newtheorem{conj}[thm]{予想}
\newtheorem{prob}[thm]{問題}
\newtheorem{rema}[thm]{補足}

\DeclareMathOperator{\Ric}{Ric}
\DeclareMathOperator{\Vol}{Vol}
 \newcommand{\pdrv}[2]{\frac{\partial #1}{\partial #2}}
 \newcommand{\drv}[2]{\frac{d #1}{d#2}}
  \newcommand{\ppdrv}[3]{\frac{\partial #1}{\partial #2 \partial #3}}


%ここから本文.
\begin{document}
%\maketitle

\begin{center}
{\Large 第9回. 行列式1 -置換- (三宅先生の本, 3.1の内容)}
\end{center}

\begin{flushright}
 岩井雅崇 2022/06/16
\end{flushright}

\section{置換}

\begin{tcolorbox}[
    colback = white,
    colframe = green!35!black,
    fonttitle = \bfseries,
    breakable = true]
    \begin{dfn}
    \text{}
    \begin{itemize}
\item $\{ 1, \ldots, n\}$から$\{ 1, \ldots, n\}$への1対1写像を\underline{置換}と言い$\sigma$で表す.
つまり置換$\sigma$とは$k_1, \ldots, k_n$を1から$n$の並び替えとして, 
1を$k_1$に, 2を$k_2$に, $\cdots$, $n$を$k_n$にと変化させる規則のことである.
\item 上の置換$\sigma$を
$$
\sigma =
  \begin{pmatrix}
 1& 2  &\cdots &n\\
 k_1& k_2  &\cdots &k_n\\
 \end{pmatrix} 
$$
とかき, $\sigma(1) =k_1, \sigma(2) =k_2, \ldots, \sigma(n) =k_n$とする.
    \end{itemize}
  \end{dfn}
 \end{tcolorbox}
 
 \begin{exa}
 置換$\sigma$を
$
\sigma =
  \begin{pmatrix}
 1& 2  &3 &4\\
 3& 1  &4 &2\\
 \end{pmatrix} 
$
とする. 
これは「1を$3$に, 2を$1$に, 3を4に, 4を$2$にと変化させる規則」である.
 $\sigma(1) =3, \sigma(2) =1, \sigma(3) =4,  \sigma(4) =2$である.
 \end{exa}
 
 \begin{exa}
 置換$\sigma$を
$
\sigma =
  \begin{pmatrix}
 1& 2  &3 \\
 2& 1  &3 \\
 \end{pmatrix} 
$
とする. 
これは「1を$2$に, 2を$1$に, 3を3にと変化させる規則」である.
 $\sigma(1) =2, \sigma(2) =1, \sigma(3) =3$である.
 
 この置換は3に関しては何も変化させていないので
 $
\sigma =
  \begin{pmatrix}
 1& 2   \\
 2& 1   \\
 \end{pmatrix} 
$
ともかく.
 \end{exa}

\begin{tcolorbox}[
    colback = white,
    colframe = green!35!black,
    fonttitle = \bfseries,
    breakable = true]
    \begin{dfn}
置換$\sigma, \tau$について, その積$\sigma \tau$を
$\sigma(\tau(i))$で定める.
  \end{dfn}
 \end{tcolorbox}
 
\begin{exa}
 置換$\sigma, \tau$を
$
\sigma =
  \begin{pmatrix}
 1& 2  &3 & 4 \\
 4& 3  &1  &2 \\
 \end{pmatrix} 
\tau=
  \begin{pmatrix}
 1& 2  &3 & 4 \\
 2& 3  &4  &1 \\
 \end{pmatrix} 
 $
とすると, 
$$
  \begin{matrix}
 \sigma (\tau (1)) &= &  \sigma (2)  & = & 3  \\
 \sigma (\tau (2)) &= &  \sigma (3)  & = & 1 \\
 \sigma (\tau (3)) &= &  \sigma (4)  & = & 2  \\
 \sigma (\tau (4)) &= &  \sigma (1)  & = & 4  \\
 \end{matrix} 
 \text{\,\,\,であるので, }
 \sigma \tau
= 
 \begin{pmatrix}
 1& 2  &3 & 4 \\
 3& 1  &2  &4 \\
 \end{pmatrix} 
 \text{である.}
$$

\end{exa}


\begin{tcolorbox}[
    colback = white,
    colframe = green!35!black,
    fonttitle = \bfseries,
    breakable = true]
    \begin{dfn}
\text{}
\begin{itemize}
\item $
\epsilon =
  \begin{pmatrix}
 1& 2  &\cdots &n\\
 1& 2  &\cdots &n\\
 \end{pmatrix} $を\underline{単位置換}という.
 \item  
 $ \sigma =
  \begin{pmatrix}
 1& 2  &\cdots &n\\
 k_1& k_2  &\cdots &k_n\\
 \end{pmatrix} 
$について, 
$
  \begin{pmatrix}
 k_1& k_2  &\cdots &k_n\\
 1& 2  &\cdots &n\\
 \end{pmatrix} 
$を\underline{$\sigma$の逆置換}と言い$\sigma^{-1}$で表す.
\end{itemize}
  \end{dfn}
 \end{tcolorbox}

\begin{exa} 
$\sigma = 
\begin{pmatrix}
 1& 2  &3 & 4 & 5\\
 4& 5  &1  &3 &2\\
 \end{pmatrix} 
$
とするとき
$
\sigma^{-1}
=
\begin{pmatrix}
 4& 5  &1  &3 &2\\
 1& 2  &3 & 4 & 5\\
 \end{pmatrix} 
 =
 \begin{pmatrix}
 1& 2  &3 & 4 & 5\\
 3& 5  &4  &1 &2\\
 \end{pmatrix} 
 \text{である.}
$
\end{exa}

\begin{tcolorbox}[
    colback = white,
    colframe = green!35!black,
    fonttitle = \bfseries,
    breakable = true]
    \begin{dfn}
 $ \sigma =
  \begin{pmatrix}
 k_1& k_2  &\cdots &k_l\\
 k_2& k_3  &\cdots &k_1\\
 \end{pmatrix} 
$となる置換$\sigma$を\underline{巡回置換}と言い
$\sigma =
  \begin{pmatrix}
 k_1& k_2  &\cdots &k_l\\
 \end{pmatrix} 
$と表す.

特に $ \sigma =
  \begin{pmatrix}
 k_1& k_2  \\
 k_2& k_1 \\
 \end{pmatrix} 
$となる巡回置換を\underline{互換}と言い$\sigma =
  \begin{pmatrix}
 k_1& k_2 \\
 \end{pmatrix} 
$と表す.
  \end{dfn}
 \end{tcolorbox}


\begin{tcolorbox}[
    colback = white,
    colframe = green!35!black,
    fonttitle = \bfseries,
    breakable = true]
    \begin{thm}
任意の置換$\sigma$は互換の積$\tau_1 \cdots \tau_{l}$で表わすことができ, $l$の偶奇は$\sigma$によってのみ定まる.
  \end{thm}
 \end{tcolorbox}
 
 \begin{tcolorbox}[
    colback = white,
    colframe = green!35!black,
    fonttitle = \bfseries,
    breakable = true]
    \begin{dfn}
置換$\sigma$が互換の積$\tau_1 \cdots \tau_{l}$で表せられているとする.
\begin{itemize}
\item $\sgn (\sigma) = (-1)^{l}$とし, これを\underline{$\sigma$の符号}と呼ぶ.
\item $\sgn (\sigma) = 1$なる置換$\sigma$を\underline{偶置換}といい, $\sgn (\sigma) = -1$なる置換$\sigma$を\underline{奇置換}という.
\end{itemize}
  \end{dfn}
 \end{tcolorbox}
 
 \begin{exa}
 $
 \sigma 
  =
 \begin{pmatrix}
 1& 2  &3 & 4 & 5 & 6 & 7\\
 4& 1  &6  &2 &7 & 5 & 3\\
 \end{pmatrix} 
 $を互換の積で表し, その符号を求めよ.
 
 (解). 
 $1 \overset{\sigma}{\rightarrow} 4 \overset{\sigma}{\rightarrow}2 \overset{\sigma}{\rightarrow}1 $と変化し,  
  $3 \overset{\sigma}{\rightarrow} 6\overset{\sigma}{\rightarrow}5 \overset{\sigma}{\rightarrow}7 \overset{\sigma}{\rightarrow}3$と変化するので, 
  $$
  \sigma = 
   \begin{pmatrix}
 1& 4 &2 
 \end{pmatrix} 
    \begin{pmatrix}
 3& 6 &5 &7
 \end{pmatrix} 
 \text{である.}
  $$
  さらに
  $   \begin{pmatrix}
 1& 4 &2 
 \end{pmatrix} 
 = 
 \begin{pmatrix}
 1& 4  
 \end{pmatrix} 
 \begin{pmatrix}
 4 &2 
 \end{pmatrix},
\begin{pmatrix}
 3& 6 &5 &7
 \end{pmatrix} 
 =
 \begin{pmatrix}
 3& 6  
 \end{pmatrix} 
  \begin{pmatrix}
 6& 5  
 \end{pmatrix} 
  \begin{pmatrix}
 5& 7  
 \end{pmatrix} 
 $
 であるので, 
 $$
\sigma= 
\begin{pmatrix}
 1& 4  
 \end{pmatrix} 
 \begin{pmatrix}
 4 &2 
 \end{pmatrix}
 \begin{pmatrix}
 3& 6  
 \end{pmatrix} 
  \begin{pmatrix}
 6& 5  
 \end{pmatrix} 
  \begin{pmatrix}
 5& 7  
 \end{pmatrix} 
 $$
 となり, $\sgn(\sigma)= (-1)^{5}=-1$である.
 
 \end{exa}

  \begin{tcolorbox}[
    colback = white,
    colframe = green!35!black,
    fonttitle = \bfseries,
    breakable = true]
    \begin{prop}置換$\sigma, \tau$について, 
    $\sgn(\epsilon) = 1$, $\sgn(\sigma^{-1}) = \sgn(\sigma)$, 
$\sgn(\sigma \tau) = \sgn(\sigma) \sgn(\tau) $が成り立つ(ただし$\epsilon$は単位置換とする).
  \end{prop}
 \end{tcolorbox}
 
 
  \begin{tcolorbox}[
    colback = white,
    colframe = green!35!black,
    fonttitle = \bfseries,
    breakable = true]
    \begin{dfn}
$S_n$を$n$文字置換の集合とし, $A_n$を$n$文字置換の集合とする.
  \end{dfn}
 \end{tcolorbox}
 \footnote{専門用語で$S_n$は対称群と言い, $A_n$は交代群と言います. }

  \begin{tcolorbox}[
    colback = white,
    colframe = green!35!black,
    fonttitle = \bfseries,
    breakable = true]
    \begin{prop}\text{}
    \begin{itemize}
\item $S_n$の個数は$n!$個である.
\item 偶置換と奇置換の個数は同じである.
\item $A_n$の個数は$\frac{n!}{2}$個である.
\item $\sigma, \tau \in A_n$ならば$\sigma \tau \in A_n$
    \end{itemize}
  \end{prop}
 \end{tcolorbox}

\newpage

\begin{center}
{\Large 第10回. 行列式2 -行列式の計算方法- (三宅先生の本, 3.2, 3.3の内容)} 
\end{center}

\begin{flushright}
 岩井雅崇 2022/06/23
\end{flushright}

%この資料では第10回授業と第11回授業の内容を取り扱います.

\section{行列式}

\begin{tcolorbox}[
    colback = white,
    colframe = green!35!black,
    fonttitle = \bfseries,
    breakable = true]
    \begin{dfn}
$n$次正方行列$A = (a_{ij})$について
$$
\det(A) =  \sum_{\sigma \in S_n}\sgn(\sigma) 
a_{1 \sigma(1)} a_{2 \sigma(2)} \cdots a_{n \sigma(n)} 
\text{を\underline{$A$の行列式}と言う.}
$$
 $A$の行列式は$\det(A)$, $|A|$, 
$
\begin{vmatrix}
a_{11}& a_{12} & \cdots &a_{1n} \\
a_{21}& a_{22} & \cdots &a_{2n} \\
\vdots& \vdots	&	\ddots   &	\vdots \\
a_{n1}& a_{n2} & \cdots &a_{nn} \\
\end{vmatrix}
$
ともかく.
  \end{dfn}
 \end{tcolorbox}

\begin{exa}
\label{2jidet}
$A = 
  \begin{pmatrix}
a_{11}& a_{12}\\
a_{21}& a_{22}\\
 \end{pmatrix} 
$
とすると$\det(A) = a_{11}a_{22} - a_{12}a_{21}$である.

(証).
$S_2 = \left\{   \begin{pmatrix}
1& 2\\
1& 2\\
 \end{pmatrix} , 
   \begin{pmatrix}
1&2\\
2& 1\\
 \end{pmatrix} 
  \right\}$
  であるので, $A$の行列式は
  \begin{align*}
  \det(A) &= 
  \sgn \begin{pmatrix}
1& 2\\
1& 2\\
 \end{pmatrix} a_{11}a_{22}
 +
  \sgn \begin{pmatrix}
1& 2\\
2& 1\\
 \end{pmatrix} a_{12}a_{21}
 =
 a_{11}a_{22} - a_{12}a_{21}.
  \end{align*}

\end{exa}

\begin{exa}
$A = 
  \begin{pmatrix}
a_{11}& a_{12} & a_{13}\\
a_{21}& a_{22} & a_{23}\\
a_{31}& a_{32} & a_{33}\\
 \end{pmatrix} 
$
の行列式を求める.

$S_3 = \left\{   
\begin{pmatrix}
1& 2 &3\\
1& 2 &3\\
 \end{pmatrix} , 
\begin{pmatrix}
1& 2 &3\\
2& 1 &3\\
 \end{pmatrix} , 
\begin{pmatrix}
1& 2 &3\\
1& 3 &2\\
 \end{pmatrix} , 
 \begin{pmatrix}
1& 2 &3\\
3& 2 &1\\
 \end{pmatrix} , 
 \begin{pmatrix}
1& 2 &3\\
2& 3 &1\\
 \end{pmatrix} , 
 \begin{pmatrix}
1& 2 &3\\
3& 1 &2\\
 \end{pmatrix}
  \right\}$
  であるので, $A$の行列式は
  
  \begin{align*}
  \det(A) 
  &= 
  \sgn \begin{pmatrix}
1& 2 &3\\
1& 2 &3\\
 \end{pmatrix}
 a_{11}a_{22}a_{33}
 +
   \sgn \begin{pmatrix}
1& 2 &3\\
2& 1 &3\\
 \end{pmatrix} 
 a_{12}a_{21}a_{33}
 +
\sgn \begin{pmatrix}
1& 2 &3\\
1& 3 &2\\
 \end{pmatrix} 
 a_{11}a_{23}a_{32} 
  \\%%%
  &+
\sgn  \begin{pmatrix}
1& 2 &3\\
3& 2 &1\\
 \end{pmatrix}
 a_{13}a_{22}a_{31} 
 +
\sgn  \begin{pmatrix}
1& 2 &3\\
2& 3 &1\\
 \end{pmatrix}
 a_{12}a_{23}a_{31} 
  +
\sgn   \begin{pmatrix}
1& 2 &3\\
3& 1 &2\\
 \end{pmatrix}
 a_{13}a_{21}a_{32} 
 \\%%
   &= a_{11}a_{22}a_{33}- a_{12}a_{21}a_{33}- a_{11}a_{23}a_{32} 
   - a_{13}a_{22}a_{31}  + a_{12}a_{23}a_{31}  +  a_{13}a_{21}a_{32} 
  \end{align*}
  以上より
  $
 \det(A)= 
 a_{11}a_{22}a_{33}+ a_{12}a_{23}a_{31}  +  a_{13}a_{21}a_{32} 
- a_{11}a_{23}a_{32}     - a_{13}a_{22}a_{31}  - a_{12}a_{21}a_{33}
  $
  である.
\end{exa}
\begin{rema}
2次正方行列や3次正方行列の行列式は視覚的に綺麗に表わすことができる(サラスの公式と呼ばれる).
\end{rema}

\section{行列式の基本性質}

\begin{tcolorbox}[
    colback = white,
    colframe = green!35!black,
    fonttitle = \bfseries,
    breakable = true]
    \begin{thm}
   \label{determinant}
$A,B$を$n$次正方行列とする.
\begin{enumerate}
\item $\det({}^{t} A) =\det(A)$.
\item $\det(AB)=(\det(A))(\det(B)) = \det(BA)$.
\item $\det(A) \neq 0$であることと$A$が正則であることは同値.
\item 
$
\begin{vmatrix}
a_{11}& a_{12} & \cdots &a_{1n} \\
0 	   & a_{22} & \cdots &a_{2n} \\
\vdots& \vdots	&	\ddots   &	\vdots \\
0	& a_{n2} & \cdots &a_{nn} \\
\end{vmatrix}
=a_{11}
\begin{vmatrix}
 a_{22} & \cdots &a_{2n} \\
 \vdots	&	\ddots   &	\vdots \\
 a_{n2} & \cdots &a_{nn} \\
\end{vmatrix}
$.
\item 
$
\begin{vmatrix}
a_{11}& a_{12} & a_{13} &\cdots &a_{1n-1}&a_{1n} \\
0 	   & a_{22} & a_{23} &\cdots&a_{2n-1} &a_{2n} \\
0 	   & 0  		& a_{33} &\cdots &a_{3n-1}&a_{3n} \\
\vdots& \vdots	&  \ddots &\ddots&	\vdots  &	\vdots \\
0	& 0      		&      	&\ddots	&a_{n-1n-1} &a_{n-1n} \\
0	& 0      		&     \cdots	&	\cdots&0		&a_{nn} \\
\end{vmatrix}
=a_{11}a_{22}\cdots a_{nn}
$.
%以下行ベクトル$a_1, \cdots, a_n$を用いて$A = \begin{pmatrix}a_1 \\a_2 \\\vdots\\a_n\end{pmatrix}$とかけているとする.
\item  1つの行を$c$倍すると行列式は$c$倍される:
$
\begin{vmatrix}
a_{11}&  \cdots &a_{1n} \\
\vdots&	 	  &	\vdots \\
ca_{i1} & \cdots &ca_{in} \\
\vdots& 		   &	\vdots \\
a_{n1}	& \cdots &a_{nn} \\
\end{vmatrix}
=
c
\begin{vmatrix}
a_{11}&  \cdots &a_{1n} \\
\vdots&	 	  &	\vdots \\
a_{i1} & \cdots & a_{in} \\
\vdots& 		   &	\vdots \\
a_{n1}	& \cdots &a_{nn} \\
\end{vmatrix}
$.
\item  
$
\begin{vmatrix}
a_{11}&  \cdots &a_{1n} \\
\vdots&	 	  &	\vdots \\
b_{i1} + c_{i1} & \cdots &b_{in} + c_{in} \\
\vdots& 		   &	\vdots \\
a_{n1}	& \cdots &a_{nn} \\
\end{vmatrix}
=
\begin{vmatrix}
a_{11}&  \cdots &a_{1n} \\
\vdots&	 	  &	\vdots \\
b_{i1}  & \cdots &b_{in}  \\
\vdots& 		   &	\vdots \\
a_{n1}	& \cdots &a_{nn} \\
\end{vmatrix}
+
\begin{vmatrix}
a_{11}&  \cdots &a_{1n} \\
\vdots&	 	  &	\vdots \\
c_{i1} & \cdots & c_{in} \\
\vdots& 		   &	\vdots \\
a_{n1}	& \cdots &a_{nn} \\
\end{vmatrix}
$.
\item 2つの行を入れ替えたら, 行列式は$-1$倍される:
$$
\begin{vmatrix}
a_{11}&  \cdots &a_{1n} \\
\vdots& 		   &	\vdots \\
a_{j1} & \cdots & a_{jn} \\
\vdots&	 	  &	\vdots \\
a_{i1} & \cdots & a_{in} \\
\vdots& 		   &	\vdots \\
a_{n1}	& \cdots &a_{nn} \\
\end{vmatrix}
= (-1)
\begin{vmatrix}
a_{11}&  \cdots &a_{1n} \\
\vdots&	 	  &	\vdots \\
a_{i1} & \cdots & a_{in} \\
\vdots& 		   &	\vdots \\
a_{j1} & \cdots & a_{jn} \\
\vdots& 		   &	\vdots \\
a_{n1}	& \cdots &a_{nn} \\
\end{vmatrix}.
$$
\item 第$i$行の$c$倍を第$j$行に加えても行列式は変わらない:
$$
\begin{vmatrix}
a_{11}&  \cdots &a_{1n} \\
\vdots& 		   &	\vdots \\
a_{j1} +ca_{i1}& \cdots & a_{jn} +ca_{in}\\
\vdots& 		   &	\vdots \\
a_{n1}	& \cdots &a_{nn} \\
\end{vmatrix}
= 
\begin{vmatrix}
a_{11}&  \cdots &a_{1n} \\
\vdots& 		   &	\vdots \\
a_{j1} & \cdots & a_{jn} \\
\vdots& 		   &	\vdots \\
a_{n1}	& \cdots &a_{nn} \\
\end{vmatrix}.
$$
\item 列ベクトルに関して上の6から9と同様のことが成り立つ.
\end{enumerate}
  \end{thm}
 \end{tcolorbox}
 
 \begin{tcolorbox}[
    colback = white,
    colframe = green!35!black,
    fonttitle = \bfseries,
    breakable = true]
    \begin{cor}
$A,B$を$n$次正方行列とする. $AB=E_n$ならば, $A$は正則で$B$は$A$の逆行列.
  \end{cor}
 \end{tcolorbox}
 
 \section{行列式の計算方法}
 定理\ref{determinant}を用いると行列式を比較的簡単に計算できる.
 
  \begin{exa}
 $
 \begin{pmatrix}
 1&3&4\\
 -2&-5&7\\
 -3&2&-1\\
 \end{pmatrix}
$
の行列式を定理\ref{determinant}を用いて計算すると次の通りになる.

\begin{align*}
 &\begin{vmatrix}
 1&3&4\\
 -2&-5&7\\
 -3&2&-1\\
 \end{vmatrix}
 \overset{\text{定理\ref{determinant}.(9)} } {=}
 \begin{vmatrix}
 1&3&4\\
 0&1&15\\
 0&11&11\\
 \end{vmatrix}
 \overset{\text{定理\ref{determinant}.(4)} } {=}
 1
 \begin{vmatrix}
1&15\\
11&11\\
 \end{vmatrix}
  \overset{\text{定理\ref{determinant}.(6)} } {=}
 11
 \begin{vmatrix}
1&15\\
1&1\\
 \end{vmatrix}\overset{\text{例 \ref{2jidet}} } {=}
 11 \left\{ 1 \times 1 - 15 \times 1 \right\} 
 =
 -154.
 \end{align*}
 
\end{exa}



 \begin{exa}
 $
 \begin{pmatrix}
 2&-4&-5&3\\
 -6&13&14&1\\
 1&-2&-2&-8\\
 2&-5&0&5\\
 \end{pmatrix}
$
の行列式を定理\ref{determinant}を用いて計算すると次の通りになる.
\begin{align*}
 &\begin{vmatrix}
 2&-4&-5&3\\
 -6&13&14&1\\
 1&-2&-2&-8\\
 2&-5&0&5\\
 \end{vmatrix}
 \overset{\text{定理\ref{determinant}.(8)} } {=}
 (-1)
  \begin{vmatrix}
   1&-2&-2&-8\\
 -6&13&14&1\\
 2&-4&-5&3\\
 2&-5&0&5\\
 \end{vmatrix}
  \overset{\text{定理\ref{determinant}.(9)} }  {=}
 (-1)
  \begin{vmatrix}
   1&-2&-2&-8\\
 0&1 &2  &-47\\
 0& 0&-1&19\\
 0&-1&4&21\\
 \end{vmatrix}
\\ %%%
& \overset{\text{定理\ref{determinant}.(4)} } {=}
 (-1)
  \begin{vmatrix}
1 &2  &-47\\
 0&-1&19\\
-1&4&21\\
 \end{vmatrix}
  \overset{\text{定理\ref{determinant}.(9)} } {=}
   (-1)
  \begin{vmatrix}
1 &2  &-47\\
 0&-1&19\\
 0&6&-26\\
 \end{vmatrix}
 \overset{\text{定理\ref{determinant}.(4)} } {=}
  (-1)
    \begin{vmatrix}
-1&19\\
6&-26\\
 \end{vmatrix}
 \\ %%
 & \overset{\text{例 \ref{2jidet}} } {=}
 (-1)\left\{(-1)\times (-26) - 6\times 19\right\} = 88.
\end{align*}

 \end{exa}

\section{演習問題}
演習問題の解答は授業動画にあります.

1. 行列式
$
\begin{vmatrix}
 0& -3& -6 &15 \\
 -2& 5& 14 &4 \\
 1& -3& -2 &5 \\
 15 & 10& 10 &-5 \\
 \end{vmatrix} 
 $
 を計算せよ.


\newpage

\begin{center}
{\Large 第11回. 行列式3 -行列式の基本性質- (三宅先生の本, 3.2, 3.3の内容)} 
\end{center}

\begin{flushright}
 岩井雅崇 2022/06/30
\end{flushright}

一部の内容について, 齋藤正彦著 線型代数学 (東京図書)の第3章を参考にした.

%\section{行列式の基本性質}

\begin{tcolorbox}[
    colback = white,
    colframe = green!35!black,
    fonttitle = \bfseries,
    breakable = true]
    \begin{prop}
$a_1, \ldots, a_{n}$を行ベクトルとし, $n$次正方行列
$A = 
\begin{pmatrix}
a_1 \\ \vdots \\ a_{n}
\end{pmatrix}
$とする.
\begin{enumerate}
\item $\tau$を$n$次の置換とすると
$$
\det \begin{pmatrix}
a_{\tau(1)} \\ \vdots \\ a_{\tau(n)} 
\end{pmatrix}
= 
\sgn(\tau) \det \begin{pmatrix}
a_1 \\ \vdots \\ a_{n}
\end{pmatrix}
= \sgn(\tau) \det(A).
\text{\,\,\,(交代性)}
$$
\item $b_i, c_i$を行ベクトルとし, $\alpha, \beta$を数とすると,
$$
\det \begin{pmatrix}
a_1 \\ \vdots \\ \alpha b_i + \beta c_i \\ \vdots  \\ a_{n}
\end{pmatrix}
= 
\alpha\det \begin{pmatrix}
a_1 \\ \vdots \\  b_i \\ \vdots  \\ a_{n}
\end{pmatrix}
+
\beta
\det \begin{pmatrix}
a_1 \\ \vdots \\ c_i \\ \vdots  \\ a_{n}
\end{pmatrix}
\text{\,\,\,(多重線型性)}
$$
\end{enumerate}

  \end{prop}
 \end{tcolorbox}
 

\begin{tcolorbox}[
    colback = white,
    colframe = green!35!black,
    fonttitle = \bfseries,
    breakable = true]
    \begin{thm}
行ベクトル$x_1, \ldots, x_n$について, 数
$F\begin{pmatrix}
x_1 \\ \vdots \\ x_{n}
\end{pmatrix}$
を対応させる関数$F$を考える.
この$F$が交代性と多重線型性を満たすとき, 
$$
F\begin{pmatrix}
x_1 \\ \vdots \\ x_{n}
\end{pmatrix}
=
F\begin{pmatrix}
f_1 \\ \vdots \\ f_{n}
\end{pmatrix}
\det
\begin{pmatrix}
x_1 \\ \vdots \\ x_{n}
\end{pmatrix}
\text{となる.}
$$
ここで$f_i =
\begin{pmatrix}
0 & \cdots&\overset{i}{\hat{1}}&\cdots &0
\end{pmatrix}
$
という行ベクトルとする.

特に行列$A$に対して数$F(A)$を対応させる関数が, 行に関して交代性と多重線型性を満たすとき
$F(A) =F(E_n) \det(A)$となる.
  \end{thm}
 \end{tcolorbox}
 
\newpage

\begin{center}
{\Large 第12回. 余因子行列と余因子展開 (三宅先生の本, 3.4の内容)} 
\end{center}

\begin{flushright}
 岩井雅崇 2022/07/07
\end{flushright}


\section{余因子行列}

\begin{tcolorbox}[
    colback = white,
    colframe = green!35!black,
    fonttitle = \bfseries,
    breakable = true]
    \begin{dfn}
    $n$次正方行列$A=(a_{ij})$の$i$行と$j$列を取り除いた$n-1$次正方行列を$\tilde{A}_{ij}$とかく(この授業だけの記法). つまり
  $$
  \tilde{A}_{ij}
  =
    \begin{pmatrix}
a_{11}&   \cdots &a_{1j-1}&a_{1j+1}&\cdots&a_{1n} \\
\vdots&   		& \vdots &\vdots &   		&\vdots  \\
a_{i-11}&   \cdots &a_{i-1j-1}&a_{i-1j+1}&\cdots&a_{i-1n} \\
a_{i+11}&   \cdots &a_{i+1j-1}&a_{i+1j+1}&\cdots&a_{i+1n} \\
\vdots&   		& \vdots &\vdots &   		&\vdots  \\
a_{n1}&   \cdots &a_{nj-1}&a_{nj+1}&\cdots&a_{nn} \\
\end{pmatrix}
\text{とする.}
$$
    \end{dfn}
 \end{tcolorbox}
\begin{exa}
$A=
\begin{pmatrix}
a_{11} & a_{12} \\
a_{21} & a_{22}
\end{pmatrix}
$
のとき, 
$  \tilde{A}_{11} =(a_{22})$, $  \tilde{A}_{12} =(a_{21})$, $  \tilde{A}_{21} =(a_{12})$, $  \tilde{A}_{22} =(a_{11})$.
\end{exa}
\begin{exa}
$
A=
\begin{pmatrix}
a_{11} & a_{12}&a_{13} \\
a_{21} & a_{22}&a_{23} \\
a_{31} & a_{32}&a_{33} \\
\end{pmatrix}
$
のとき, 
$  \tilde{A}_{12} =
\begin{pmatrix}
a_{21} & a_{23} \\
a_{31} & a_{33}
\end{pmatrix}
$, 
$  \tilde{A}_{22} =
\begin{pmatrix}
a_{11} & a_{13} \\
a_{31} & a_{33}
\end{pmatrix}
$, 
$  \tilde{A}_{31} =
\begin{pmatrix}
a_{12} & a_{13} \\
a_{22} & a_{23}
\end{pmatrix}
$.
\end{exa}

\begin{tcolorbox}[
    colback = white,
    colframe = green!35!black,
    fonttitle = \bfseries,
    breakable = true]
    \begin{dfn}
    $n$次正方行列$A=(a_{ij})$について, $\tilde{A} =(b_{ij})$を
    \underline{$
    b_{ij} = (-1)^{i+j} \det(\tilde{A}_{ji})$}
    で定める.
 \underline{$\tilde{A}$を$A$の余因子行列}という.
    \end{dfn}
 \end{tcolorbox}
\begin{exa}
\label{inverse_2}
$
\begin{pmatrix}
a_{11} & a_{12} \\
a_{21} & a_{22}
\end{pmatrix}
$
のときの余因子行列$\tilde{A} $を求める.
$  \tilde{A}_{11} =(a_{22})$, $  \tilde{A}_{12} =(a_{21})$, $  \tilde{A}_{21} =(a_{12})$, $  \tilde{A}_{22} =(a_{11})$より次が成り立つ.
\begin{itemize}
\item $\tilde{A} $の$(1,1)$成分は$(-1)^{1+1}\det( \tilde{A}_{11}) = a_{22}$.
\item $\tilde{A} $の$(1,2)$成分は$(-1)^{1+2}\det( \tilde{A}_{21}) = -a_{12}$.
\item $\tilde{A} $の$(2,1)$成分は$(-1)^{2+1}\det( \tilde{A}_{12}) = -a_{21}$.
\item $\tilde{A} $の$(2,2)$成分は$(-1)^{2+2}\det( \tilde{A}_{22}) = a_{11}$.
\end{itemize}
以上より余因子行列$\tilde{A} = 
\begin{pmatrix}
a_{22} &- a_{12} \\
-a_{21} & a_{11}
\end{pmatrix}$となる.
\end{exa}


\begin{tcolorbox}[
    colback = white,
    colframe = green!35!black,
    fonttitle = \bfseries,
    breakable = true]
    \begin{thm}
 $A$を$n$次正方行列とする.
 \begin{enumerate}
\item %$i=1, \ldots, n, j=1, \ldots, n$について
任意の$1 \leqq i \leqq n, 1 \leqq j\leqq n$なる$i,j$について, 次が成り立つ.
 \begin{align*}
 \det(A) & =(-1)^{1+j}a_{1j}\det(\tilde{A}_{1j}) + \cdots +(-1)^{n+j}a_{nj}\det(\tilde{A}_{nj}) 
 \\
 &=(-1)^{i+1}a_{i1}\det(\tilde{A}_{i1}) + \cdots +(-1)^{i+n}a_{in}\det(\tilde{A}_{in}).
  \end{align*}
  これを\underline{余因子展開}という.
 \item $A\tilde{A} = \tilde{A}A =(\det A)E_n$. 特に$\det(A)\neq0$ならば$A^{-1} = \frac{1}{\det A} \tilde{A}$.
 \end{enumerate}
     \end{thm}
 \end{tcolorbox}

\begin{exa}行列
$A=
\begin{pmatrix}
2 & 7&13 & 5\\
5 & 3&8 & 2\\
0 & 0 & 9  & 4\\
0 & 0&-2 & 1\\
\end{pmatrix}
$の行列式$\det(A)$を余因子展開で求める.

\begin{align*}
\det(A) 
&= 
(-1)^{1+1}a_{11}\det(\tilde{A}_{11}) + (-1)^{2+1}a_{21}\det(\tilde{A}_{21}) + (-1)^{3+1}a_{31}\det(\tilde{A}_{31}) + (-1)^{4+1}a_{41}\det(\tilde{A}_{41}) 
\\ %%
&=
2 
\begin{vmatrix}
 3&8 & 2\\
0 & 9  & 4\\
0&-2 & 1\\
\end{vmatrix}
- 5 
\begin{vmatrix}
 7&13 & 5\\
0 & 9  & 4\\
0&-2 & 1\\
\end{vmatrix}
+0
\begin{vmatrix}
 7&13 & 5\\
 3&8 & 2\\
0&-2 & 1\\
\end{vmatrix}
-0
\begin{vmatrix}
 7&13 & 5\\
 3&8 & 2\\
0 & 9  & 4\\
\end{vmatrix}
\\ %%
&=2 
\begin{vmatrix}
 3&8 & 2\\
0 & 9  & 4\\
0&-2 & 1\\
\end{vmatrix}
- 5 
\begin{vmatrix}
 7&13 & 5\\
0 & 9  & 4\\
0&-2 & 1\\
\end{vmatrix}
\\%%
&=2 \times 3
\begin{vmatrix}
9  & 4\\
-2 & 1\\
\end{vmatrix}
-5 \times 7
\begin{vmatrix}
9  & 4\\
-2 & 1\\
\end{vmatrix}
=(2 \times 3 - 5 \times 7) \times (9 \times 1 - 4 \times (-2)) = -493.
\end{align*}
\end{exa}


\begin{exa}
2次正方行列
$
A = 
\begin{pmatrix}
a & b \\
c & d
\end{pmatrix}
$について, $\det A =ad-bc \neq0$ならば$A$は正則であり, 
例\ref{inverse_2}から
$$
A^{-1} = \frac{1}{\det A} \tilde{A}
=\frac{1}{ad-bc}
\begin{pmatrix}
d & -b \\
-c & a
\end{pmatrix}.
$$
\end{exa}

\section{演習問題}
演習問題の解答は授業動画にあります.

1. 行列式
$
\begin{vmatrix}
3 & 5&1 & 2&-1\\
2 & 6&0 & 9&1\\
0 & 0& 7& 1&2\\
0 & 0& 3& 2&5\\
0 & 0& 0& 0&-6\\
\end{vmatrix}
$を計算せよ.


\newpage

\begin{center}
{\Large 第13回. クラメルの公式と特殊な行列式 (三宅先生の本, 3.4, 3.5の内容)} 
\end{center}

\begin{flushright}
 岩井雅崇 2022/07/14
\end{flushright}

この授業で行う内容は理解しなくても構いません(結構マニアックな話題を扱います). 
また覚える必要もございません. 
\section{クラメルの公式}

\begin{tcolorbox}[
    colback = white,
    colframe = green!35!black,
    fonttitle = \bfseries,
    breakable = true]
    \begin{thm}
$A$を正則な$n$次正方行列とし, 列ベクトル$a_1, \ldots, a_{n}$を用いて
$
A = 
\begin{pmatrix}
a_1 & \cdots & a_{n}
\end{pmatrix}
$
と表されているとする.
このとき連立1次方程式$A \bm{x} =\bm{b}$の解は次のようになる.
$$
\bm{x}= \begin{pmatrix}
x_1 \\ \vdots \\ x_{n}
\end{pmatrix}, 
x_i = \frac{\det
\begin{pmatrix}
a_1 & \cdots& \bm{b}&\cdots & a_{n}
\end{pmatrix}
}{\det A}.
$$
    \end{thm}
 \end{tcolorbox}
\begin{exa}
$
A = 
\begin{pmatrix}
5 &1\\
3&2 \\ 
\end{pmatrix}
$, $
\bm{b} = 
\begin{pmatrix}
3\\
2 \\ 
\end{pmatrix}
$
とする. 
連立1次方程式$A \bm{x} =\bm{b}$の解を
$
\bm{x}= \begin{pmatrix}
x_1 \\x_2
\end{pmatrix} 
$
とすると,
$$
x_1 = \frac{\det
\begin{pmatrix}
 \bm{b}& a_{2}
\end{pmatrix}
}{\det A}
= 
\frac{ 
\begin{vmatrix}
3&1\\
2&2 \\
\end{vmatrix}
}
{
\begin{vmatrix}
5&1\\
3&2 \\
\end{vmatrix}
}
=\frac{4}{7} 
\text{, }
x_2 = \frac{\det
\begin{pmatrix}
a_{1}& \bm{b}
\end{pmatrix}
}{\det A}
= 
\frac{ 
\begin{vmatrix}
5&3\\
3&2 \\
\end{vmatrix}
}
{
\begin{vmatrix}
5&1\\
3&2 \\
\end{vmatrix}
}
=\frac{1}{7}
\text{となる.}
$$
\end{exa}

\section{特殊な行列式}
\begin{tcolorbox}[
    colback = white,
    colframe = green!35!black,
    fonttitle = \bfseries,
    breakable = true]
    \begin{thm}
  \begin{enumerate}
\item (ヴァンデルモンドの行列式)
\begin{align*}
 &
 \begin{vmatrix}
1   &   1     & \cdots &  1  \\
x_1& x_2 & \cdots &x_n \\
{x_1}^{2}& {x_2}^{2} & \cdots &{x_n}^{2} \\
\vdots & \vdots    &  &\vdots \\
{x_1}^{n-1}& {x_2}^{n-1} & \cdots &{x_n}^{n-1} \\
\end{vmatrix}
= \prod_{1 \leqq i < j \leqq n} (x_j - x_i).
\\%%
%&=(x_2 - x_1)  (x_3 - x_1)  \cdots  (x_n - x_1)  (x_3- x_2)    \cdots  (x_n - x_2)  (x_4 - x_3)\cdots (x_n - x_{n-1})
%\begin{matrix}=(x_2 - x_1)  (x_3 - x_1)  \cdots  (x_n - x_1) \\\times (x_3- x_2)  (x_4 - x_2)  \cdots  (x_n - x_2) \\\times \cdots \\\times (x_n - x_{n-1}) \\ \end{matrix}
%\begin{matrix}(x_2 - x_1) & \times&  (x_3 - x_1) & \times&  \cdots &\times & (x_n - x_1) \\\end{matrix}
\end{align*}
\item (ヴァンデルモンドの行列式の応用) $b_1, \ldots, b_n, c_1, \ldots, c_n$を実数とし, $b_1, \ldots, b_n$は相異なると仮定する.
 このとき実数係数の$n$次式$f(x) = x^{n} +  a_1 x^{n-1} +  \cdots +  a_{n-1} x + a_{n}$があって, 任意の$i = 1, \ldots, n$について$f(b_i) =c_i $となる.

  \end{enumerate}
    \end{thm}
 \end{tcolorbox}
$\prod_{1 \leqq i < j \leqq n}$は積の記号で, $\prod_{1 \leqq i < j \leqq n} (x_j - x_i)$は「$1 \leqq i < j \leqq n$を満たす$(i,j)$について$ (x_j - x_i)$を全てかけた数」を表している.



\begin{tcolorbox}[
    colback = white,
    colframe = green!35!black,
    fonttitle = \bfseries,
    breakable = true]
    \begin{thm}
$$
 \begin{vmatrix}
a_0   &   -1   &0 &0 & \cdots &  0 \\
a_1   &   x    &-1  &0 & \cdots &  0 \\
a_2   &   0    & x   &-1 & \cdots &  0 \\
\vdots  &    \vdots   &   &\ddots & \ddots &  0 \\
a_{n-1}&   0    & 0   &                  & x&  -1\\
a_n   &   0   & 0   &0 & \cdots &  x\\
\end{vmatrix}
=a_0 x^{n} +  a_1 x^{n-1} +  \cdots +  a_{n-1} x + a_{n}.
$$
    \end{thm}
 \end{tcolorbox}
 
\section{終結式と判別式}
以下の内容は「永田雅宜著 理系のための線型代数の基礎 (紀伊國屋書店)」の第3章に基づく.

\begin{tcolorbox}[
    colback = white,
    colframe = green!35!black,
    fonttitle = \bfseries,
    breakable = true]
    \begin{dfn}
$a_0, a_1, \ldots, a_n$を複素数とし, $f(x) = a_0 x^{n} +  a_1 x^{n-1} +  \cdots +  a_{n-1} x + a_{n}$とする(ただし$a_0 \neq 0$とする). $f(x)=0$の解を$\alpha_{1}, \ldots, \alpha_{n}$とするとき, 
$$
D = a_{0}^{2n -2} \prod_{1 \leqq i < j \leqq n} (\alpha_j - \alpha_i)^2
\text{ を\underline{$f(x)$の判別式}という.}
$$
    \end{dfn}
 \end{tcolorbox}
 簡単にわかることとして, 「$D \neq 0$ $\Leftrightarrow$ $f(x)=0$の解が相異なる」である.
\begin{exa}
$f(x) = a_0 x^2 + a_1 x + a_2$の判別式$D$を求める(ただし$a_0 \neq 0$とする). 
$\alpha_1, \alpha_2$を$f(x)=0$の解とすると, 解と係数の関係から
$$
a_0 x^2 + a_1 x + a_2 = a_0 (x - \alpha_1) (x- \alpha_2)
$$
であるので, $-a_ 1 = a_0 (\alpha_1 + \alpha_2), a_2 = a_0\alpha_1 \alpha_2$となる. よって
$$
D = a_{0}^{2} (\alpha_2 - \alpha_1)^2 = a_{0}^{2}\{ (\alpha_1 + \alpha_2)^2 - 4 \alpha_1 \alpha_2 \} = a_{1}^{2} - 4 a_0 a_{2}.
 $$
\end{exa}

\begin{tcolorbox}[
    colback = white,
    colframe = green!35!black,
    fonttitle = \bfseries,
    breakable = true]
    \begin{dfn}
複素係数多項式$f(x) = a_0 x^{n} +  a_1 x^{n-1} +  \cdots +  a_{n-1} x + a_{n}$, 
$g(x) = b_0 x^{m} +  b_1 x^{m-1} +  \cdots +  b_{m-1} x + b_{m}$
(ただし$a_0 \neq 0, b_0 \neq 0$)について, $m+n$次正方行列を次で定める.
$$
 \begin{pmatrix}
 a_0 	& a_1& \cdots &a_n 	&0		&0 	     & \cdots&0 \\
0   		& a_0 & a_1		& \cdots &a_n 	&0 	      & \cdots&0 \\
 0 		& 0      & a_0 		&    a_1	& \cdots &a_n  & \cdots&0 \\
\vdots  & \vdots  &      \ddots 	&     \ddots &  \ddots     &   & \ddots&\vdots \\
   0		 & 0         &	 \cdots	&    0	& a_0 &a_1 & \cdots &a_n \\
 b_0 	& b_1& b_2 &  \cdots  	& \cdots		&b_m	     & \cdots&0 \\
%0		& b_0 	& b_1& b_2 &  \cdots  	&b_m 	    &      &0 \\
\vdots 	&    \ddots 	&   \ddots    &   \ddots     & 		&   	    & \ddots&\vdots \\
0		&     \cdots 	&      b_0 & b_1 &  b_2  	&  \cdots& \cdots&b_m \\
 \end{pmatrix}
$$
この行列の行列式を\underline{$f,g$の終結式と言い, $R(f,g)$と表す.}
    \end{dfn}
 \end{tcolorbox}


\begin{tcolorbox}[
    colback = white,
    colframe = green!35!black,
    fonttitle = \bfseries,
    breakable = true]
    \begin{thm}
    \begin{enumerate}
\item $f(x) =0$の解を$\alpha_1, \ldots, \alpha_n$とし, $g(x) =0$の解を$\beta_1, \ldots, \beta_m$とすると
$$
R(f,g) = a_{0}^{m}b_{0}^{n} \prod_{1 \leqq i \leqq n, 1\leqq j \leqq m} (\alpha_i - \beta_j)
=a_{0}^{m} \prod_{1 \leqq i \leqq n} g(\alpha_{i})\text{である.}
$$
特に$R(f,g)=0$は$f(x)=g(x)=0$が共通解を持つことと同値である.
\item $f'$を$f$の微分とすると, 
$$
R(f, f') = (-1)^{\frac{n(n-1)}{2}} a_0 D.
$$
特に$f(x)$の判別式$D$は$a_0, \ldots, a_n$の式でかける. また$R(f,f')=0$は$f(x)=0$が重根を持つことと同値である.
    \end{enumerate}

    \end{thm}
 \end{tcolorbox}
 
 \newpage


\begin{center}
{\Large 第14回. 内積と外積} 
\end{center}

\begin{flushright}
 岩井雅崇 2022/07/21
\end{flushright}

以下の内容は「基礎数学研究会 新版基礎線形代数 (東海大学出版会)」の第8章を参考にした. 
これも覚える必要はない(\underline{ただしベクトル解析などで役に立つ内容である}).

\section{内積}

$\R$を実数の集合とし, $n \geqq1$なる自然数について
$$
\R^n  = \{ (x_1, \ldots, x_n) | x_1, \ldots, x_n \in \R\} \text{とする.}
$$
\begin{exa}
$\R^2$は平面をあらわし, $\R^3$は空間を表す.
\end{exa}

\begin{tcolorbox}[
    colback = white,
    colframe = green!35!black,
    fonttitle = \bfseries,
    breakable = true]
    \begin{dfn}
$\bm{a}=(a_1, \ldots, a_n), \bm{b}=(b_1, \ldots, b_n)\in \R^n$, $\alpha \in \R$について和, 差, スカラー倍, 内積, 長さ(ノルム)を次で定める.
\begin{itemize}
\item 和 $\bm{a} + \bm{b} = (a_1 + b_1, \ldots, a_n + b_n)$.
\item 差 $\bm{a} - \bm{b} = (a_1 - b_1, \ldots, a_n - b_n)$.
\item スカラー倍 $\alpha \bm{a} = (\alpha a_1, \ldots, \alpha a_n)$.
\item 内積 $\bm{a} \cdot\bm{b} = a_1 b_1 + \cdots + a_n b_n $.
\item 長さ(ノルム) $||\bm{a}||= \sqrt{\bm{a} \cdot\bm{a}} = \sqrt{a_{1}^{2}+ \cdots + a_{n}^{2}}$.
\end{itemize}
    \end{dfn}
 \end{tcolorbox}

\begin{exa}
$\bm{a}=(3,5), \bm{b} = (6,1), \alpha=2$とすると
$\bm{a} + \bm{b} =(9,6)$, $\bm{a} - \bm{b} =(-3,4)$, $\alpha \bm{a}= (6,10)$, 
$\bm{a} \cdot\bm{b} = 3 \times 6 + 5 \times 1 =23$, $||\bm{a}||=\sqrt{3^2 + 5^2}= \sqrt{34}$となる.
\end{exa}

\begin{tcolorbox}[
    colback = white,
    colframe = green!35!black,
    fonttitle = \bfseries,
    breakable = true]
    \begin{prop}
$\bm{a}, \bm{b} \in \R^n$とする.
\begin{enumerate}
\item (中線定理) $||\bm{a} + \bm{b}||^2 + ||\bm{a} - \bm{b}||^2 = 2(||\bm{a}||^2 + ||\bm{b}||^2)$.
\item $\bm{a} \cdot\bm{b} = \frac{1}{4}(||\bm{a} + \bm{b}||^2 - ||\bm{a} - \bm{b}||^2)
= \frac{1}{2}(||\bm{a} + \bm{b}||^2 - ||\bm{a} ||^2- || \bm{b}||^2)
= \frac{1}{2}(||\bm{a} ||^2 + || \bm{b}||^2 - ||\bm{a} - \bm{b}||^2)$.
\item (Cauchy-Schwarzの不等式) $(\bm{a} \cdot\bm{b})^2 \leqq ||\bm{a} ||^2 ||\bm{b} ||^2 $.
\item (三角不等式) $ ||\bm{a} + \bm{b} ||   \leqq ||\bm{a} || +  ||\bm{b} ||  $.
\item $n=3$とし$\bm{a} =(a_1, a_2, a_3), \bm{b}=(b_1, b_2, b_3)$とする.
$\R^3$上の点Pを$(a_1, a_2, a_3)$, $\R^3$上の点Qを$(b_1, b_2, b_3)$, $\R^3$上の原点を点Oとする.
このとき線分OPとOQがなす角を$\theta$とすると
$$
\bm{a} \cdot\bm{b} = ||\bm{a} || || \bm{b}|| \cos \theta \text{となる.}
$$
特に$||\bm{a} ||  \neq 0$かつ$|| \bm{b}|| \neq0$のとき, $\bm{a} \cdot\bm{b} =0$は直線OPとOQが直交していることと同値である.
\end{enumerate}
    \end{prop}
 \end{tcolorbox}
\begin{exa}
$\bm{a} = (a_1, a_2, a_3)$に直交し点$\bm{c} = (c_1, c_2, c_3)$を通る平面$S$を求めよ.

(解). $\bm{x}=(x_1, x_2, x_3)$が平面$S$の点であるとき, $\bm{x} - \bm{c}$と$\bm{a}$は直交する.
よって
$(\bm{x} - \bm{c}) \cdot \bm{a} =0$である.
$$
(\bm{x} - \bm{c}) \cdot \bm{a} = 
a_1(x_1 - c_1) +a_2(x_2 - c_2) +a_3(x_3 - c_3)
$$
であるので, $S = \{ (x_1, x_2, x_3) \in \R^3 | a_1(x_1 - c_1) +a_2(x_2 - c_2) +a_3(x_3 - c_3)=0\} $となる.
\end{exa}

\section{外積}
\begin{tcolorbox}[
    colback = white,
    colframe = green!35!black,
    fonttitle = \bfseries,
    breakable = true]
    \begin{dfn}
$\bm{a}=(a_1, a_2, a_3), \bm{b}=(b_1, b_2, b_3)\in \R^3$について, 外積$\bm{a} \times \bm{b}$を次で定める. 
\begin{align*}
\bm{a} \times \bm{b} 
&=\left(\begin{vmatrix}
a_2&a_3\\
b_2&b_3 \\
\end{vmatrix},
\begin{vmatrix}
a_3&a_1\\
b_3&b_1 \\
\end{vmatrix},
\begin{vmatrix}
a_1&a_2\\
b_1&b_2 \\
\end{vmatrix}
\right)
\\
&=
( a_2b_3 - a_3b_2, a_3b_1-a_1b_3, a_1b_2-a_2b_1)  
\end{align*}
\end{dfn}
 \end{tcolorbox}
 \begin{exa}
 $\bm{a}=(3, 5, 0), \bm{b}=(6, 1, 0)$とすると
 $$
 \bm{a} \times \bm{b}
 =\left(\begin{vmatrix}
5&0\\
1&0 \\
\end{vmatrix},
\begin{vmatrix}
0&3\\
0&6 \\
\end{vmatrix},
\begin{vmatrix}
3&5\\
6&1 \\
\end{vmatrix}
\right)
= (0,0,-27)
\text{, } 
\bm{b} \times  \bm{a} 
 =\left(\begin{vmatrix}
1&0 \\
5&0\\
\end{vmatrix},
\begin{vmatrix}
0&6 \\
0&3\\
\end{vmatrix},
\begin{vmatrix}
6&1 \\
3&5\\
\end{vmatrix}
\right)
= (0,0,27).
 $$
 \end{exa}
 
 \begin{tcolorbox}[
    colback = white,
    colframe = green!35!black,
    fonttitle = \bfseries,
    breakable = true]
    \begin{prop}
$\bm{a}, \bm{b} \in \R^3$とする.
\begin{enumerate}
\item $\bm{b} \times  \bm{a}  = - \bm{a} \times  \bm{b}$. 特に$\bm{a} \times  \bm{a} =0$.
\item $\bm{a} \times  \bm{b}$は$\bm{a} $や$\bm{b}$に直交する.
\item $\bm{a} \times  \bm{b} =0$であることは$\bm{a} $と$\bm{b}$が平行であることと同値.
\item $|| \bm{a} \times  \bm{b}||$は$\bm{a} $と$\bm{b}$を2辺とする平行四辺形の面積に等しい.
\end{enumerate}
    \end{prop}
 \end{tcolorbox}
 
 
\begin{exa}
$a_1, a_2, b_1,b_2$を実数とする. 
このとき$\begin{vmatrix}
a_1&a_2 \\
b_1&b_2\\
\end{vmatrix}$
の行列式の絶対値$|a_1b_2 - a_2b_1|$は$(a_1, a_2)$と$(b_1, b_2)$を2辺とする平行四辺形の面積に等しい.

\end{exa}
\section{3次の行列式と内積外積}

\begin{tcolorbox}[
    colback = white,
    colframe = green!35!black,
    fonttitle = \bfseries,
    breakable = true]
    \begin{thm}
$\bm{a}=(a_1, a_2, a_3), \bm{b}=(b_1, b_2, b_3), \bm{c}=(c_1, c_2, c_3)\in \R^3$について, 
$$
\det
\begin{pmatrix}
a_1& a_2 & a_3\\
b_1& b_2 & b_3\\
c_1& c_2 & c_3\\
\end{pmatrix}
=\bm{a} \cdot (\bm{b} \times \bm{c}).
$$
特に
$\bm{a} \cdot (\bm{b} \times \bm{c}) = \bm{b} \cdot (\bm{c} \times \bm{a})=\bm{c} \cdot (\bm{a} \times \bm{b})$である(スカラー3重積とも呼ばれる).
\end{thm}
 \end{tcolorbox}
 
 \begin{tcolorbox}[
    colback = white,
    colframe = green!35!black,
    fonttitle = \bfseries,
    breakable = true]
    \begin{thm}
$\bm{a}=(a_1, a_2, a_3), \bm{b}=(b_1, b_2, b_3), \bm{c}=(c_1, c_2, c_3)\in \R^3$とすると次の値は等しい.
\begin{itemize}
\item $\det
\begin{pmatrix}
a_1& a_2 & a_3\\
b_1& b_2 & b_3\\
c_1& c_2 & c_3\\
\end{pmatrix}$の絶対値.
\item $\bm{a} \cdot (\bm{b} \times \bm{c})$の絶対値.
\item $\bm{a}, \bm{b}, \bm{c}$によって生成される平行6面体の体積.
\end{itemize}
\end{thm}
 \end{tcolorbox}
 
 \end{document}
