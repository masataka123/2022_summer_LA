\documentclass[dvipdfmx,a4paper,11pt]{article}
\usepackage[utf8]{inputenc}
%\usepackage[dvipdfmx]{hyperref} %リンクを有効にする
\usepackage{url} %同上
\usepackage{amsmath,amssymb} %もちろん
\usepackage{amsfonts,amsthm,mathtools} %もちろん
\usepackage{braket,physics} %あると便利なやつ
\usepackage{bm} %ラプラシアンで使った
\usepackage[top=30truemm,bottom=30truemm,left=25truemm,right=25truemm]{geometry} %余白設定
\usepackage{latexsym} %ごくたまに必要になる
\renewcommand{\kanjifamilydefault}{\gtdefault}
\usepackage{otf} %宗教上の理由でmin10が嫌いなので


\usepackage[all]{xy}
\usepackage{amsthm,amsmath,amssymb,comment}
\usepackage{amsmath}    % \UTF{00E6}\UTF{0095}°\UTF{00E5}\UTF{00AD}\UTF{00A6}\UTF{00E7}\UTF{0094}¨
\usepackage{amssymb}  
\usepackage{color}
\usepackage{amscd}
\usepackage{amsthm}  
\usepackage{wrapfig}
\usepackage{comment}	
\usepackage{graphicx}
\usepackage{setspace}
\usepackage{pxrubrica}
\setstretch{1.2}


\newcommand{\R}{\mathbb{R}}
\newcommand{\Z}{\mathbb{Z}}
\newcommand{\Q}{\mathbb{Q}} 
\newcommand{\N}{\mathbb{N}}
\newcommand{\C}{\mathbb{C}} 
\newcommand{\Sin}{\text{Sin}^{-1}} 
\newcommand{\Cos}{\text{Cos}^{-1}} 
\newcommand{\Tan}{\text{Tan}^{-1}} 
\newcommand{\invsin}{\text{Sin}^{-1}} 
\newcommand{\invcos}{\text{Cos}^{-1}} 
\newcommand{\invtan}{\text{Tan}^{-1}} 
\newcommand{\Area}{\text{Area}}
\newcommand{\vol}{\text{Vol}}
\newcommand{\maru}[1]{\raise0.2ex\hbox{\textcircled{\tiny{#1}}}}



   %当然のようにやる.
\allowdisplaybreaks[4]
   %もちろん.
%\title{第1回. 多変数の連続写像 (岩井雅崇, 2020/10/06)}
%\author{岩井雅崇}
%\date{2020/10/06}
%ここまで今回の記事関係ない
\usepackage{tcolorbox}
\tcbuselibrary{breakable, skins, theorems}

\theoremstyle{definition}
\newtheorem{thm}{定理}
\newtheorem{lem}[thm]{補題}
\newtheorem{prop}[thm]{命題}
\newtheorem{cor}[thm]{系}
\newtheorem{claim}[thm]{主張}
\newtheorem{dfn}[thm]{定義}
\newtheorem{rem}[thm]{注意}
\newtheorem{exa}[thm]{例}
\newtheorem{conj}[thm]{予想}
\newtheorem{prob}[thm]{問題}
\newtheorem{rema}[thm]{補足}

\DeclareMathOperator{\Ric}{Ric}
\DeclareMathOperator{\Vol}{Vol}
 \newcommand{\pdrv}[2]{\frac{\partial #1}{\partial #2}}
 \newcommand{\drv}[2]{\frac{d #1}{d#2}}
  \newcommand{\ppdrv}[3]{\frac{\partial #1}{\partial #2 \partial #3}}


%ここから本文.
\begin{document}
%\maketitle
\begin{center}
{\Large 第6回. 連立1次方程式3-一般的な解法- (三宅先生の本, 2.3の内容)}
\end{center}

\begin{flushright}
 岩井雅崇 2022/05/26
\end{flushright}

\section{簡約化を用いた連立1次方程式の解法}
\begin{tcolorbox}[
    colback = white,
    colframe = green!35!black,
    fonttitle = \bfseries,
    breakable = true]
    \begin{thm}
連立1次方程式$A\bm{x} =\bm{b}$が解を持つ
$\Leftrightarrow$ ${\rm rank}([A:\bm{b}]) = {\rm rank}(A)$.
  \end{thm}
 \end{tcolorbox}
%連立1次方程式$A\bm{x} =\bm{b}$を解くためには, 拡大係数行列$[A:\bm{b}]$を簡約化してあげれば良い(この解き方を掃き出し法・ガウスの消去法とも言います). 

\begin{tcolorbox}[
    colback = white,
    colframe = green!35!black,
    fonttitle = \bfseries,
    breakable = true]
連立1次方程式$A\bm{x} =\bm{b}$の解きかた(掃き出し法・ガウスの消去法).
 \begin{enumerate}
 \item[手順1.] 連立方程式$A\bm{x} =\bm{b}$から拡大係数行列$[A:\bm{b}]$を作る.
 \item[手順2.] 拡大係数行列$[A:\bm{b}]$を(行)基本変形で簡約化する.
 \item[手順3.] その簡約化された行列のデータから連立方程式を書き下し, 一般解を求める.
 \end{enumerate}
 \end{tcolorbox}
 
\begin{exa}
連立1次方程式
 $
 \left\{ 
\begin{matrix}
x_1&+&2x_2& = &2 \\
x_1&+&4x_2& = &4\\
\end{matrix}
\right.
 $
 を解け.
 
 (解). 連立方程式の拡大係数行列は
 $[A:\bm{b}]=
  \begin{pmatrix}
 1& 2& 2  \\
 2& 4& 4  \\
 \end{pmatrix}
 $
 である. これを簡約化すると
 $
  \begin{pmatrix}
 1& 2& 2  \\
 0& 0& 0  \\
 \end{pmatrix} 
 $
 となる. よってこれより
 $
  \left\{ 
\begin{matrix}
x_1&+&2x_2& = &2 \\
0x_1&+&0x_2& = &0\\
\end{matrix}
\right.
$
である. 

以上より解は
$
 \left\{ 
\begin{matrix}
x_1&=& 2 -2c_2\\
x_2 &=& c_2\\
\end{matrix}
\text{\,\, ($c_2$は任意定数)}
\right.
$
となる. 

解の書き方として
$
\begin{pmatrix}
x_1\\
x_2 \\
\end{pmatrix}
=
\begin{pmatrix}
2\\
0 \\
\end{pmatrix}
+t 
\begin{pmatrix}
-2\\
1 \\
\end{pmatrix}
(t \in \R)
$
と書くこともある.
\end{exa}

\begin{exa}
連立1次方程式
 $
 \left\{ 
\begin{matrix}
x_1&+&2x_2& = &2 \\
x_1&+&4x_2& = &5\\
\end{matrix}
\right.
 $
 を解け.
 
 (解). 連立方程式の拡大係数行列は
 $[A:\bm{b}]=
  \begin{pmatrix}
 1& 2& 2  \\
 2& 4& 5  \\
 \end{pmatrix}
 $
 である. 
 これを簡約化すると
 $
  \begin{pmatrix}
 1& 2& 0  \\
 0& 0& 1 \\
 \end{pmatrix} 
 $
 となる. よってこれより
 $
  \left\{ 
\begin{matrix}
x_1&+&2x_2& = &0 \\
0x_1&+&0x_2& = &1\\
\end{matrix}
\right.
$
である. 以上より解は存在しない.
\end{exa}

\begin{exa}
連立1次方程式
 $
 \left\{ 
\begin{matrix}
x_1&-&2x_2&   &		&+&3x_4& &	&= 2 \\
x_1&-&2x_2& + &x_3&+&2x_4&+&x_5&= 2 \\
2x_1&-&4x_2& + &x_3&+&5x_4&+&2x_5&= 5 \\
\end{matrix}
\right.
 $
 を解け.
 
 (解). 拡大係数行列は 
 $[A:\bm{b}]=
  \begin{pmatrix}
 1& -2& 0 & 3& 0& 2   \\
  1& -2& 1& 2& 1& 2   \\
 2& -4& 1 & 5& 2& 5   \\
 \end{pmatrix}
 $
 である. これを基本変形で簡約化すると
 $
  \begin{pmatrix}
 1& -2& 0 & 3& 0& 2   \\
 0& 0& 1& -1& 0& 1   \\
 0& 0& 0 & 0& 1& 1   \\
 \end{pmatrix}
 $
 となる.
 これをもう一回式に書き下すと
 $$
\left \{
 \begin{matrix}
x_1&-&2x_2&   &		&+&3x_4& &	&= 2 \\
      & &		&   &x_3       &- & x_4& &       &= -1 \\
      & & &   &    & &		& & x_5&= 1 \\
\end{matrix}
\right.
\text{である.}
 $$
 
以上より解は
$
 \left\{ 
\begin{matrix}
x_1&=& 2 +2c_2 -3c_4\\
x_2&=&c_2 \\
x_3&=& -1 + c_4\\
x_4&=&c_4 \\
x_5&=& 1\\
\end{matrix}
\text{\,\, ($c_2, c_4$は任意定数)}
\right.
$
となる. 
 
解の書き方として
$
\begin{pmatrix}
x_1\\
x_2 \\
x_3 \\
x_4 \\
x_5 \\
\end{pmatrix}
=
\begin{pmatrix}
2\\
0 \\
-1 \\
0\\
1 \\
\end{pmatrix}
+ s
\begin{pmatrix}
2\\
1\\
0\\
0\\
0 \\
\end{pmatrix}
+ t
\begin{pmatrix}
-3\\
0\\
1\\
1\\
0 \\
\end{pmatrix}
(s, t \in \R)
$
と書くこともある.

 \end{exa}

\begin{rema}
実際に連立1次方程式をプログラミングで解くときも, 掃き出し法・ガウスの消去法によって解きます. 実際にc++で書いたソースコードを以下のホームページで見ることができます.\footnote{第7回授業で簡約化の証明をする際にもこのホームページを参考にさせていただきました.}
\begin{itemize}
\item Gauss-Jordan の掃き出し法と、連立一次方程式の解き方 \\
 \texttt{https://drken1215.hatenablog.com/entry/2019/03/20/202800}
\end{itemize}
\end{rema}

\section{演習問題}
演習問題の解答は授業動画にあります.

1.
連立1次方程式
 $
 \left\{ 
\begin{matrix}
x_1& &         &  +& 2x_3&- &x_4&+ & 2x_5&= 3 \\
2x_1&+&x_2& + &3x_3&-&x_4&-&x_5&= -1 \\
-x_1&+&3x_2& - &5x_3&+&4x_4&+&x_5&= -6 \\
\end{matrix}
\right.
 $
 を解け.

 

\end{document}
