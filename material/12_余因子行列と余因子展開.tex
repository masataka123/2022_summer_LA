\documentclass[dvipdfmx,a4paper,11pt]{article}
\usepackage[utf8]{inputenc}
%\usepackage[dvipdfmx]{hyperref} %リンクを有効にする
\usepackage{url} %同上
\usepackage{amsmath,amssymb} %もちろん
\usepackage{amsfonts,amsthm,mathtools} %もちろん
\usepackage{braket,physics} %あると便利なやつ
\usepackage{bm} %ラプラシアンで使った
\usepackage[top=30truemm,bottom=30truemm,left=25truemm,right=25truemm]{geometry} %余白設定
\usepackage{latexsym} %ごくたまに必要になる
\renewcommand{\kanjifamilydefault}{\gtdefault}
\usepackage{otf} %宗教上の理由でmin10が嫌いなので


\usepackage[all]{xy}
\usepackage{amsthm,amsmath,amssymb,comment}
\usepackage{amsmath}    % \UTF{00E6}\UTF{0095}°\UTF{00E5}\UTF{00AD}\UTF{00A6}\UTF{00E7}\UTF{0094}¨
\usepackage{amssymb}  
\usepackage{color}
\usepackage{amscd}
\usepackage{amsthm}  
\usepackage{wrapfig}
\usepackage{comment}	
\usepackage{graphicx}
\usepackage{setspace}
\usepackage{pxrubrica}
\setstretch{1.2}


\newcommand{\R}{\mathbb{R}}
\newcommand{\Z}{\mathbb{Z}}
\newcommand{\Q}{\mathbb{Q}} 
\newcommand{\N}{\mathbb{N}}
\newcommand{\C}{\mathbb{C}} 
\newcommand{\Sin}{\text{Sin}^{-1}} 
\newcommand{\Cos}{\text{Cos}^{-1}} 
\newcommand{\Tan}{\text{Tan}^{-1}} 
\newcommand{\invsin}{\text{Sin}^{-1}} 
\newcommand{\invcos}{\text{Cos}^{-1}} 
\newcommand{\invtan}{\text{Tan}^{-1}} 
\newcommand{\Area}{\text{Area}}
\newcommand{\vol}{\text{Vol}}
\newcommand{\maru}[1]{\raise0.2ex\hbox{\textcircled{\tiny{#1}}}}
\newcommand{\sgn}{{\rm sgn}}
%\newcommand{\rank}{{\rm rank}}



   %当然のようにやる.
\allowdisplaybreaks[4]
   %もちろん.
%\title{第1回. 多変数の連続写像 (岩井雅崇, 2020/10/06)}
%\author{岩井雅崇}
%\date{2020/10/06}
%ここまで今回の記事関係ない
\usepackage{tcolorbox}
\tcbuselibrary{breakable, skins, theorems}

\theoremstyle{definition}
\newtheorem{thm}{定理}
\newtheorem{lem}[thm]{補題}
\newtheorem{prop}[thm]{命題}
\newtheorem{cor}[thm]{系}
\newtheorem{claim}[thm]{主張}
\newtheorem{dfn}[thm]{定義}
\newtheorem{rem}[thm]{注意}
\newtheorem{exa}[thm]{例}
\newtheorem{conj}[thm]{予想}
\newtheorem{prob}[thm]{問題}
\newtheorem{rema}[thm]{補足}

\DeclareMathOperator{\Ric}{Ric}
\DeclareMathOperator{\Vol}{Vol}
 \newcommand{\pdrv}[2]{\frac{\partial #1}{\partial #2}}
 \newcommand{\drv}[2]{\frac{d #1}{d#2}}
  \newcommand{\ppdrv}[3]{\frac{\partial #1}{\partial #2 \partial #3}}


%ここから本文.
\begin{document}
%\maketitle
\begin{center}
{\Large 第12回. 余因子行列と余因子展開 (三宅先生の本, 3.4の内容)} 
\end{center}

\begin{flushright}
 岩井雅崇 2022/07/07
\end{flushright}


\section{余因子行列}

\begin{tcolorbox}[
    colback = white,
    colframe = green!35!black,
    fonttitle = \bfseries,
    breakable = true]
    \begin{dfn}
    $n$次正方行列$A=(a_{ij})$の$i$行と$j$列を取り除いた$n-1$次正方行列を$\tilde{A}_{ij}$とかく(この授業だけの記法). つまり
  $$
  \tilde{A}_{ij}
  =
    \begin{pmatrix}
a_{11}&   \cdots &a_{1j-1}&a_{1j+1}&\cdots&a_{1n} \\
\vdots&   		& \vdots &\vdots &   		&\vdots  \\
a_{i-11}&   \cdots &a_{i-1j-1}&a_{i-1j+1}&\cdots&a_{i-1n} \\
a_{i+11}&   \cdots &a_{i+1j-1}&a_{i+1j+1}&\cdots&a_{i+1n} \\
\vdots&   		& \vdots &\vdots &   		&\vdots  \\
a_{n1}&   \cdots &a_{nj-1}&a_{nj+1}&\cdots&a_{nn} \\
\end{pmatrix}
\text{とする.}
$$
    \end{dfn}
 \end{tcolorbox}
\begin{exa}
$A=
\begin{pmatrix}
a_{11} & a_{12} \\
a_{21} & a_{22}
\end{pmatrix}
$
のとき, 
$  \tilde{A}_{11} =(a_{22})$, $  \tilde{A}_{12} =(a_{21})$, $  \tilde{A}_{21} =(a_{12})$, $  \tilde{A}_{22} =(a_{11})$.
\end{exa}
\begin{exa}
$
A=
\begin{pmatrix}
a_{11} & a_{12}&a_{13} \\
a_{21} & a_{22}&a_{23} \\
a_{31} & a_{32}&a_{33} \\
\end{pmatrix}
$
のとき, 
$  \tilde{A}_{12} =
\begin{pmatrix}
a_{21} & a_{23} \\
a_{31} & a_{33}
\end{pmatrix}
$, 
$  \tilde{A}_{22} =
\begin{pmatrix}
a_{11} & a_{13} \\
a_{31} & a_{33}
\end{pmatrix}
$, 
$  \tilde{A}_{31} =
\begin{pmatrix}
a_{12} & a_{13} \\
a_{22} & a_{23}
\end{pmatrix}
$.
\end{exa}

\begin{tcolorbox}[
    colback = white,
    colframe = green!35!black,
    fonttitle = \bfseries,
    breakable = true]
    \begin{dfn}
    $n$次正方行列$A=(a_{ij})$について, $\tilde{A} =(b_{ij})$を
    \underline{$
    b_{ij} = (-1)^{i+j} \det(\tilde{A}_{ji})$}
    で定める.
 \underline{$\tilde{A}$を$A$の余因子行列}という.
    \end{dfn}
 \end{tcolorbox}
\begin{exa}
\label{inverse_2}
$
\begin{pmatrix}
a_{11} & a_{12} \\
a_{21} & a_{22}
\end{pmatrix}
$
のときの余因子行列$\tilde{A} $を求める.
$  \tilde{A}_{11} =(a_{22})$, $  \tilde{A}_{12} =(a_{21})$, $  \tilde{A}_{21} =(a_{12})$, $  \tilde{A}_{22} =(a_{11})$より次が成り立つ.
\begin{itemize}
\item $\tilde{A} $の$(1,1)$成分は$(-1)^{1+1}\det( \tilde{A}_{11}) = a_{22}$.
\item $\tilde{A} $の$(1,2)$成分は$(-1)^{1+2}\det( \tilde{A}_{21}) = -a_{12}$.
\item $\tilde{A} $の$(2,1)$成分は$(-1)^{2+1}\det( \tilde{A}_{12}) = -a_{21}$.
\item $\tilde{A} $の$(2,2)$成分は$(-1)^{2+2}\det( \tilde{A}_{22}) = a_{11}$.
\end{itemize}
以上より余因子行列$\tilde{A} = 
\begin{pmatrix}
a_{22} &- a_{12} \\
-a_{21} & a_{11}
\end{pmatrix}$となる.
\end{exa}


\begin{tcolorbox}[
    colback = white,
    colframe = green!35!black,
    fonttitle = \bfseries,
    breakable = true]
    \begin{thm}
 $A$を$n$次正方行列とする.
 \begin{enumerate}
\item %$i=1, \ldots, n, j=1, \ldots, n$について
任意の$1 \leqq i \leqq n, 1 \leqq j\leqq n$なる$i,j$について, 次が成り立つ.
 \begin{align*}
 \det(A) & =(-1)^{1+j}a_{1j}\det(\tilde{A}_{1j}) + \cdots +(-1)^{n+j}a_{nj}\det(\tilde{A}_{nj}) 
 \\
 &=(-1)^{i+1}a_{i1}\det(\tilde{A}_{i1}) + \cdots +(-1)^{i+n}a_{in}\det(\tilde{A}_{in}).
  \end{align*}
  これを\underline{余因子展開}という.
 \item $A\tilde{A} = \tilde{A}A =(\det A)E_n$. 特に$\det(A)\neq0$ならば$A^{-1} = \frac{1}{\det A} \tilde{A}$.
 \end{enumerate}
     \end{thm}
 \end{tcolorbox}

\begin{exa}行列
$A=
\begin{pmatrix}
2 & 7&13 & 5\\
5 & 3&8 & 2\\
0 & 0 & 9  & 4\\
0 & 0&-2 & 1\\
\end{pmatrix}
$の行列式$\det(A)$を余因子展開で求める.

\begin{align*}
\det(A) 
&= 
(-1)^{1+1}a_{11}\det(\tilde{A}_{11}) + (-1)^{2+1}a_{21}\det(\tilde{A}_{21}) + (-1)^{3+1}a_{31}\det(\tilde{A}_{31}) + (-1)^{4+1}a_{41}\det(\tilde{A}_{41}) 
\\ %%
&=
2 
\begin{vmatrix}
 3&8 & 2\\
0 & 9  & 4\\
0&-2 & 1\\
\end{vmatrix}
- 5 
\begin{vmatrix}
 7&13 & 5\\
0 & 9  & 4\\
0&-2 & 1\\
\end{vmatrix}
+0
\begin{vmatrix}
 7&13 & 5\\
 3&8 & 2\\
0&-2 & 1\\
\end{vmatrix}
-0
\begin{vmatrix}
 7&13 & 5\\
 3&8 & 2\\
0 & 9  & 4\\
\end{vmatrix}
\\ %%
&=2 
\begin{vmatrix}
 3&8 & 2\\
0 & 9  & 4\\
0&-2 & 1\\
\end{vmatrix}
- 5 
\begin{vmatrix}
 7&13 & 5\\
0 & 9  & 4\\
0&-2 & 1\\
\end{vmatrix}
\\%%
&=2 \times 3
\begin{vmatrix}
9  & 4\\
-2 & 1\\
\end{vmatrix}
-5 \times 7
\begin{vmatrix}
9  & 4\\
-2 & 1\\
\end{vmatrix}
=(2 \times 3 - 5 \times 7) \times (9 \times 1 - 4 \times (-2)) = -493.
\end{align*}
\end{exa}


\begin{exa}
2次正方行列
$
A = 
\begin{pmatrix}
a & b \\
c & d
\end{pmatrix}
$について, $\det A =ad-bc \neq0$ならば$A$は正則であり, 
例\ref{inverse_2}から
$$
A^{-1} = \frac{1}{\det A} \tilde{A}
=\frac{1}{ad-bc}
\begin{pmatrix}
d & -b \\
-c & a
\end{pmatrix}.
$$
\end{exa}

\section{演習問題}
演習問題の解答は授業動画にあります.

1. 行列式
$
\begin{vmatrix}
3 & 5&1 & 2&-1\\
2 & 6&0 & 9&1\\
0 & 0& 7& 1&2\\
0 & 0& 3& 2&5\\
0 & 0& 0& 0&-6\\
\end{vmatrix}
$を計算せよ.


 \end{document}
