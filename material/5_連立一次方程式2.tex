\documentclass[dvipdfmx,a4paper,11pt]{article}
\usepackage[utf8]{inputenc}
%\usepackage[dvipdfmx]{hyperref} %リンクを有効にする
\usepackage{url} %同上
\usepackage{amsmath,amssymb} %もちろん
\usepackage{amsfonts,amsthm,mathtools} %もちろん
\usepackage{braket,physics} %あると便利なやつ
\usepackage{bm} %ラプラシアンで使った
\usepackage[top=30truemm,bottom=30truemm,left=25truemm,right=25truemm]{geometry} %余白設定
\usepackage{latexsym} %ごくたまに必要になる
\renewcommand{\kanjifamilydefault}{\gtdefault}
\usepackage{otf} %宗教上の理由でmin10が嫌いなので


\usepackage[all]{xy}
\usepackage{amsthm,amsmath,amssymb,comment}
\usepackage{amsmath}    % \UTF{00E6}\UTF{0095}°\UTF{00E5}\UTF{00AD}\UTF{00A6}\UTF{00E7}\UTF{0094}¨
\usepackage{amssymb}  
\usepackage{color}
\usepackage{amscd}
\usepackage{amsthm}  
\usepackage{wrapfig}
\usepackage{comment}	
\usepackage{graphicx}
\usepackage{setspace}
\usepackage{pxrubrica}
\setstretch{1.2}


\newcommand{\R}{\mathbb{R}}
\newcommand{\Z}{\mathbb{Z}}
\newcommand{\Q}{\mathbb{Q}} 
\newcommand{\N}{\mathbb{N}}
\newcommand{\C}{\mathbb{C}} 
\newcommand{\Sin}{\text{Sin}^{-1}} 
\newcommand{\Cos}{\text{Cos}^{-1}} 
\newcommand{\Tan}{\text{Tan}^{-1}} 
\newcommand{\invsin}{\text{Sin}^{-1}} 
\newcommand{\invcos}{\text{Cos}^{-1}} 
\newcommand{\invtan}{\text{Tan}^{-1}} 
\newcommand{\Area}{\text{Area}}
\newcommand{\vol}{\text{Vol}}
\newcommand{\maru}[1]{\raise0.2ex\hbox{\textcircled{\tiny{#1}}}}



   %当然のようにやる.
\allowdisplaybreaks[4]
   %もちろん.
%\title{第1回. 多変数の連続写像 (岩井雅崇, 2020/10/06)}
%\author{岩井雅崇}
%\date{2020/10/06}
%ここまで今回の記事関係ない
\usepackage{tcolorbox}
\tcbuselibrary{breakable, skins, theorems}

\theoremstyle{definition}
\newtheorem{thm}{定理}
\newtheorem{lem}[thm]{補題}
\newtheorem{prop}[thm]{命題}
\newtheorem{cor}[thm]{系}
\newtheorem{claim}[thm]{主張}
\newtheorem{dfn}[thm]{定義}
\newtheorem{rem}[thm]{注意}
\newtheorem{exa}[thm]{例}
\newtheorem{conj}[thm]{予想}
\newtheorem{prob}[thm]{問題}
\newtheorem{rema}[thm]{補足}

\DeclareMathOperator{\Ric}{Ric}
\DeclareMathOperator{\Vol}{Vol}
 \newcommand{\pdrv}[2]{\frac{\partial #1}{\partial #2}}
 \newcommand{\drv}[2]{\frac{d #1}{d#2}}
  \newcommand{\ppdrv}[3]{\frac{\partial #1}{\partial #2 \partial #3}}


%ここから本文.
\begin{document}
%\maketitle
\begin{center}
{\Large 第5回. 連立1次方程式2-行列の簡約化- (三宅先生の本, 2.2の内容)}
\end{center}

\begin{flushright}
 岩井雅崇 2022/05/19
\end{flushright}

\section{簡約な行列}

\begin{tcolorbox}[
    colback = white,
    colframe = green!35!black,
    fonttitle = \bfseries,
    breakable = true]
    \begin{dfn}[簡約な行列]
  行列$A$が次の4つの条件を満たすとき, $A$を\underline{簡約な行列}という.
  \begin{enumerate}
\item 行ベクトルのうちに零ベクトル(全ての成分が0である行)があれば, それは零ベクトルでないものよりも下にある.
\item 零ベクトルでない行ベクトルの主成分は1である.
\item 第$i$行の主成分を$a_{ij_{i}}$とすると, $j_1<j_2<j_3<\cdots$となる. すなわち各行の主成分は下の行ほど右にある.
\item 各行の主成分を含む列の他の成分は全て0である. すなわち第$i$行の主成分が$a_{ij_{i}}$であるならば, 第$j_i$列の$a_{ij_{i}}$以外の成分は全て0である.
  \end{enumerate}
  \end{dfn}
 \end{tcolorbox}
 \begin{exa}
以下の行列は全て簡約な行列である.
$$
 \begin{pmatrix}
 0& 1& 3  & 0&2\\
 0& 0& 0  & 1&1\\
 0& 0& 0 & 0&0\\
 \end{pmatrix}
  \begin{pmatrix}
 1& 0& 1  & 4&0&-1\\
 0& 1& 7 & -4&0&1\\
 0& 0& 0 & 0&1&3\\
 \end{pmatrix}
   \begin{pmatrix}
 0& 0& 0  & 1&6&0&3&0\\
 0& 0& 0 & 0&0&1&2&0\\
 0& 0& 0 & 0&0&0&0&0 \\
 \end{pmatrix}
$$
\end{exa}
 \begin{exa}
 次に簡約ではない行列の例を理由とともに挙げる.
 \begin{itemize}
\item 
$ 
\begin{pmatrix}
 1& 0& 1  & 1&0\\
 0& 0& 0  & 0&0\\
 0& 0& 0 & 0&1\\
 \end{pmatrix} 
 $
 は1番目の条件が満たされていないので簡約ではない.
 \item 
$ 
\begin{pmatrix}
 1& 0& 1  & 1&0\\
 0& 0& 0  & 0&3\\
 \end{pmatrix} 
 $
 は2番目の条件が満たされていないので簡約ではない.
 \item 
$ 
\begin{pmatrix}
 0& 0& 1  & 0&0\\
 1& 0& 0  & 0&0\\
 \end{pmatrix} 
 $
 は3番目の条件が満たされていないので簡約ではない.
 \item 
$ 
\begin{pmatrix}
 1& 0& 1  & 1&0\\
 1& 0& 0  & 0&1\\
 \end{pmatrix} 
 $
 は4番目の条件が満たされていないので簡約ではない.
 \end{itemize}
\end{exa}

\begin{tcolorbox}[
    colback = white,
    colframe = green!35!black,
    fonttitle = \bfseries,
    breakable = true]
    \begin{dfn}[簡約化]
  行列$A$に(行)基本変形
 \begin{enumerate}
 \item 1つの行を何倍か($\neq 0$倍)する.
 \item 2つの行を入れ替える.
 \item1つの行に他の行の何倍かを加える.
 \end{enumerate}
 を繰り返して簡約な行列$B$を得ることを\underline{$A$を簡約化する}といい, \underline{$B$を$A$の簡約化}という.
   \end{dfn}
 \end{tcolorbox}
 
 \begin{tcolorbox}[
    colback = white,
    colframe = green!35!black,
    fonttitle = \bfseries,
    breakable = true]
    \begin{thm}
    任意の行列は基本変形を繰り返すことによって簡約化することができ, その簡約化は一意に定まる.
   \end{thm}
 \end{tcolorbox}
 
 \begin{tcolorbox}[
    colback = white,
    colframe = green!35!black,
    fonttitle = \bfseries,
    breakable = true]
    \begin{dfn}[階数(ランク)]
$A$を行列とし, $B$を$A$の簡約化とする.
${\rm rank}(A)$を$B$の零ベクトルでない行の個数とし\underline{$A$の階数(ランク)}と呼ぶ.
   \end{dfn}
 \end{tcolorbox}
${\rm rank}(A)$は簡約化の仕方によらずに定まる数である.
また$A$を$m\times n$行列とすると${\rm rank}(A) \le \min(m,n)$である.
\begin{exa}
$A=
 \begin{pmatrix}
 0& 1& 3  & 0&2\\
 0& 0& 0  & 1&1\\
 0& 0& 0 & 0&0\\
 \end{pmatrix}
 $
 とすると, これは簡約な行列であり零ベクトルでない行の個数は2個である. よって${\rm rank}(A)=2$.
 
 $B= \begin{pmatrix}
 1& 0& 1  & 4&0&-1\\
 0& 1& 7 & -4&0&1\\
 0& 0& 0 & 0&1&3\\
 \end{pmatrix}
 $ とすると, これは簡約な行列であり零ベクトルでない行の個数は3個である. よって${\rm rank}(B)=3$.
\end{exa}

\begin{exa}
$
 \begin{pmatrix}
 1& 2& -3  \\
 1& 1& 1  \\
 \end{pmatrix}
 $
 を基本変形で簡約化すると次のとおりである.\footnote{第4回授業資料と同じで「$\maru{2} + \maru{1}\times(-1)$」は「行列の2行目に1行目の(-1)倍を加える」を意味している.}
 \begin{align*}
  \begin{pmatrix}
 1& 2& -3  \\
 1& 1& 1  \\
 \end{pmatrix}
 \overset{\text{$\maru{2} + \maru{1}\times(-1)$}}{\longrightarrow} 
   \begin{pmatrix}
 1& 2& -3  \\
 0& -1& 4  \\
 \end{pmatrix}
 \overset{\text{$\maru{2} \times(-1)$}}{\longrightarrow} 
   \begin{pmatrix}
 1& 2& -3  \\
 0& 1& -4  \\
 \end{pmatrix}
  \overset{\maru{1} + \maru{2}\times(-1)}{\longrightarrow} 
   \begin{pmatrix}
 1& 0& 5  \\
 0& 1& -4  \\
 \end{pmatrix}.
  \end{align*}
  よってこの行列の階数(ランク)は2である.
\end{exa}

\begin{exa}
$
 \begin{pmatrix}
 1& 0& 2  &1\\
 2& 1& 1  &0\\
 0& 1& 1  &0\\
 \end{pmatrix}
 $
 を基本変形で簡約化すると次のとおりである.
 
 \begin{align*}
 &\begin{pmatrix}
 1& 0& 2  &1\\
 2& 1& 1  &0\\
 0& 1& 1  &0\\
 \end{pmatrix}
 \overset{\maru{2} + \maru{1}\times(-2)}{\longrightarrow} 
\begin{pmatrix}
 1& 0& 2  &1\\
 0& 1& -3 &-2\\
 0& 1& 1  &0\\
 \end{pmatrix}
\overset{\maru{3} + \maru{2}\times(-1)}{\longrightarrow} 
\begin{pmatrix}
 1& 0& 2  &1\\
 0& 1& -3 &-2\\
 0& 0& 4  &2\\
 \end{pmatrix}
 \\ %%
 & \overset{\maru{3}\times \frac{1}{2}}{\longrightarrow} 
\begin{pmatrix}
 1& 0& 2  &1\\
 0& 1& -3 &-2\\
 0& 0& 2  &1\\
 \end{pmatrix} 
 \overset{\maru{1} + \maru{3}\times (-1)}{\underset{\maru{2} + \maru{3}\times \frac{3}{2}}{\longrightarrow}}
 \begin{pmatrix}
 1& 0& 0  &0\\
 0& 1& 0 &-\frac{1}{2}\\
 0& 0& 2  &1\\
 \end{pmatrix} 
 \overset{\maru{3}\times \frac{1}{2}}{\longrightarrow} 
  \begin{pmatrix}
 1& 0& 0  &0\\
 0& 1& 0 &-\frac{1}{2}\\
 0& 0& 1 &\frac{1}{2}\\
 \end{pmatrix}.
 \end{align*}
   よってこの行列の階数(ランク)は3である.
 \end{exa}
 
\section{演習問題}
演習問題の解答は授業動画にあります.

1.
$
 \begin{pmatrix}
 1& 0& -1  & 0&-2\\
 0& 1& 1  & 0&1\\
 -1& 0& 1 & 1&1\\
 2& 1& -1 & 0&3\\
 \end{pmatrix}
 $
 を簡約化し, その階数を求めよ.

2.
$
 \begin{pmatrix}
 1& 0& -1  & 0&-2\\
 0& 1& 1  & 0&1\\
  -1& 0& 1 & 1&1\\
 2& 1& -1 & 0&-3\\
 \end{pmatrix}
 $
 を簡約化し, その階数を求めよ.


 

\end{document}
