\documentclass[dvipdfmx,a4paper,11pt]{article}
\usepackage[utf8]{inputenc}
%\usepackage[dvipdfmx]{hyperref} %リンクを有効にする
\usepackage{url} %同上
\usepackage{amsmath,amssymb} %もちろん
\usepackage{amsfonts,amsthm,mathtools} %もちろん
\usepackage{braket,physics} %あると便利なやつ
\usepackage{bm} %ラプラシアンで使った
\usepackage[top=30truemm,bottom=30truemm,left=25truemm,right=25truemm]{geometry} %余白設定
\usepackage{latexsym} %ごくたまに必要になる
\renewcommand{\kanjifamilydefault}{\gtdefault}
\usepackage{otf} %宗教上の理由でmin10が嫌いなので


\usepackage[all]{xy}
\usepackage{amsthm,amsmath,amssymb,comment}
\usepackage{amsmath}    % \UTF{00E6}\UTF{0095}°\UTF{00E5}\UTF{00AD}\UTF{00A6}\UTF{00E7}\UTF{0094}¨
\usepackage{amssymb}  
\usepackage{color}
\usepackage{amscd}
\usepackage{amsthm}  
\usepackage{wrapfig}
\usepackage{comment}	
\usepackage{graphicx}
\usepackage{setspace}
\usepackage{pxrubrica}
\setstretch{1.2}


\newcommand{\R}{\mathbb{R}}
\newcommand{\Z}{\mathbb{Z}}
\newcommand{\Q}{\mathbb{Q}} 
\newcommand{\N}{\mathbb{N}}
\newcommand{\C}{\mathbb{C}} 
\newcommand{\Sin}{\text{Sin}^{-1}} 
\newcommand{\Cos}{\text{Cos}^{-1}} 
\newcommand{\Tan}{\text{Tan}^{-1}} 
\newcommand{\invsin}{\text{Sin}^{-1}} 
\newcommand{\invcos}{\text{Cos}^{-1}} 
\newcommand{\invtan}{\text{Tan}^{-1}} 
\newcommand{\Area}{\text{Area}}
\newcommand{\vol}{\text{Vol}}
\newcommand{\maru}[1]{\raise0.2ex\hbox{\textcircled{\tiny{#1}}}}
\newcommand{\sgn}{{\rm sgn}}
%\newcommand{\rank}{{\rm rank}}



   %当然のようにやる.
\allowdisplaybreaks[4]
   %もちろん.
%\title{第1回. 多変数の連続写像 (岩井雅崇, 2020/10/06)}
%\author{岩井雅崇}
%\date{2020/10/06}
%ここまで今回の記事関係ない
\usepackage{tcolorbox}
\tcbuselibrary{breakable, skins, theorems}

\theoremstyle{definition}
\newtheorem{thm}{定理}
\newtheorem{lem}[thm]{補題}
\newtheorem{prop}[thm]{命題}
\newtheorem{cor}[thm]{系}
\newtheorem{claim}[thm]{主張}
\newtheorem{dfn}[thm]{定義}
\newtheorem{rem}[thm]{注意}
\newtheorem{exa}[thm]{例}
\newtheorem{conj}[thm]{予想}
\newtheorem{prob}[thm]{問題}
\newtheorem{rema}[thm]{補足}

\DeclareMathOperator{\Ric}{Ric}
\DeclareMathOperator{\Vol}{Vol}
 \newcommand{\pdrv}[2]{\frac{\partial #1}{\partial #2}}
 \newcommand{\drv}[2]{\frac{d #1}{d#2}}
  \newcommand{\ppdrv}[3]{\frac{\partial #1}{\partial #2 \partial #3}}


%ここから本文.
\begin{document}
%\maketitle
\begin{center}
{\Large 第14回. 内積と外積} 
\end{center}

\begin{flushright}
 岩井雅崇 2022/07/21
\end{flushright}

以下の内容は「基礎数学研究会 新版基礎線形代数 (東海大学出版会)」の第8章を参考にした. 
これも覚える必要はない(\underline{ただしベクトル解析などで役に立つ内容である}).

\section{内積}

$\R$を実数の集合とし, $n \geqq1$なる自然数について
$$
\R^n  = \{ (x_1, \ldots, x_n) | x_1, \ldots, x_n \in \R\} \text{とする.}
$$
\begin{exa}
$\R^2$は平面をあらわし, $\R^3$は空間を表す.
\end{exa}

\begin{tcolorbox}[
    colback = white,
    colframe = green!35!black,
    fonttitle = \bfseries,
    breakable = true]
    \begin{dfn}
$\bm{a}=(a_1, \ldots, a_n), \bm{b}=(b_1, \ldots, b_n)\in \R^n$, $\alpha \in \R$について和, 差, スカラー倍, 内積, 長さ(ノルム)を次で定める.
\begin{itemize}
\item 和 $\bm{a} + \bm{b} = (a_1 + b_1, \ldots, a_n + b_n)$.
\item 差 $\bm{a} - \bm{b} = (a_1 - b_1, \ldots, a_n - b_n)$.
\item スカラー倍 $\alpha \bm{a} = (\alpha a_1, \ldots, \alpha a_n)$.
\item 内積 $\bm{a} \cdot\bm{b} = a_1 b_1 + \cdots + a_n b_n $.
\item 長さ(ノルム) $||\bm{a}||= \sqrt{\bm{a} \cdot\bm{a}} = \sqrt{a_{1}^{2}+ \cdots + a_{n}^{2}}$.
\end{itemize}
    \end{dfn}
 \end{tcolorbox}

\begin{exa}
$\bm{a}=(3,5), \bm{b} = (6,1), \alpha=2$とすると
$\bm{a} + \bm{b} =(9,6)$, $\bm{a} - \bm{b} =(-3,4)$, $\alpha \bm{a}= (6,10)$, 
$\bm{a} \cdot\bm{b} = 3 \times 6 + 5 \times 1 =23$, $||\bm{a}||=\sqrt{3^2 + 5^2}= \sqrt{34}$となる.
\end{exa}

\begin{tcolorbox}[
    colback = white,
    colframe = green!35!black,
    fonttitle = \bfseries,
    breakable = true]
    \begin{prop}
$\bm{a}, \bm{b} \in \R^n$とする.
\begin{enumerate}
\item (中線定理) $||\bm{a} + \bm{b}||^2 + ||\bm{a} - \bm{b}||^2 = 2(||\bm{a}||^2 + ||\bm{b}||^2)$.
\item $\bm{a} \cdot\bm{b} = \frac{1}{4}(||\bm{a} + \bm{b}||^2 - ||\bm{a} - \bm{b}||^2)
= \frac{1}{2}(||\bm{a} + \bm{b}||^2 - ||\bm{a} ||^2- || \bm{b}||^2)
= \frac{1}{2}(||\bm{a} ||^2 + || \bm{b}||^2 - ||\bm{a} - \bm{b}||^2)$.
\item (Cauchy-Schwarzの不等式) $(\bm{a} \cdot\bm{b})^2 \leqq ||\bm{a} ||^2 ||\bm{b} ||^2 $.
\item (三角不等式) $ ||\bm{a} + \bm{b} ||   \leqq ||\bm{a} || +  ||\bm{b} ||  $.
\item $n=3$とし$\bm{a} =(a_1, a_2, a_3), \bm{b}=(b_1, b_2, b_3)$とする.
$\R^3$上の点Pを$(a_1, a_2, a_3)$, $\R^3$上の点Qを$(b_1, b_2, b_3)$, $\R^3$上の原点を点Oとする.
このとき線分OPとOQがなす角を$\theta$とすると
$$
\bm{a} \cdot\bm{b} = ||\bm{a} || || \bm{b}|| \cos \theta \text{となる.}
$$
特に$||\bm{a} ||  \neq 0$かつ$|| \bm{b}|| \neq0$のとき, $\bm{a} \cdot\bm{b} =0$は直線OPとOQが直交していることと同値である.
\end{enumerate}
    \end{prop}
 \end{tcolorbox}
\begin{exa}
$\bm{a} = (a_1, a_2, a_3)$に直交し点$\bm{c} = (c_1, c_2, c_3)$を通る平面$S$を求めよ.

(解). $\bm{x}=(x_1, x_2, x_3)$が平面$S$の点であるとき, $\bm{x} - \bm{c}$と$\bm{a}$は直交する.
よって
$(\bm{x} - \bm{c}) \cdot \bm{a} =0$である.
$$
(\bm{x} - \bm{c}) \cdot \bm{a} = 
a_1(x_1 - c_1) +a_2(x_2 - c_2) +a_3(x_3 - c_3)
$$
であるので, $S = \{ (x_1, x_2, x_3) \in \R^3 | a_1(x_1 - c_1) +a_2(x_2 - c_2) +a_3(x_3 - c_3)=0\} $となる.
\end{exa}

\section{外積}
\begin{tcolorbox}[
    colback = white,
    colframe = green!35!black,
    fonttitle = \bfseries,
    breakable = true]
    \begin{dfn}
$\bm{a}=(a_1, a_2, a_3), \bm{b}=(b_1, b_2, b_3)\in \R^3$について, 外積$\bm{a} \times \bm{b}$を次で定める. 
\begin{align*}
\bm{a} \times \bm{b} 
&=\left(\begin{vmatrix}
a_2&a_3\\
b_2&b_3 \\
\end{vmatrix},
\begin{vmatrix}
a_3&a_1\\
b_3&b_1 \\
\end{vmatrix},
\begin{vmatrix}
a_1&a_2\\
b_1&b_2 \\
\end{vmatrix}
\right)
\\
&=
( a_2b_3 - a_3b_2, a_3b_1-a_1b_3, a_1b_2-a_2b_1)  
\end{align*}
\end{dfn}
 \end{tcolorbox}
 \begin{exa}
 $\bm{a}=(3, 5, 0), \bm{b}=(6, 1, 0)$とすると
 $$
 \bm{a} \times \bm{b}
 =\left(\begin{vmatrix}
5&0\\
1&0 \\
\end{vmatrix},
\begin{vmatrix}
0&3\\
0&6 \\
\end{vmatrix},
\begin{vmatrix}
3&5\\
6&1 \\
\end{vmatrix}
\right)
= (0,0,-27)
\text{, } 
\bm{b} \times  \bm{a} 
 =\left(\begin{vmatrix}
1&0 \\
5&0\\
\end{vmatrix},
\begin{vmatrix}
0&6 \\
0&3\\
\end{vmatrix},
\begin{vmatrix}
6&1 \\
3&5\\
\end{vmatrix}
\right)
= (0,0,27).
 $$
 \end{exa}
 
 \begin{tcolorbox}[
    colback = white,
    colframe = green!35!black,
    fonttitle = \bfseries,
    breakable = true]
    \begin{prop}
$\bm{a}, \bm{b} \in \R^3$とする.
\begin{enumerate}
\item $\bm{b} \times  \bm{a}  = - \bm{a} \times  \bm{b}$. 特に$\bm{a} \times  \bm{a} =0$.
\item $\bm{a} \times  \bm{b}$は$\bm{a} $や$\bm{b}$に直交する.
\item $\bm{a} \times  \bm{b} =0$であることは$\bm{a} $と$\bm{b}$が平行であることと同値.
\item $|| \bm{a} \times  \bm{b}||$は$\bm{a} $と$\bm{b}$を2辺とする平行四辺形の面積に等しい.
\end{enumerate}
    \end{prop}
 \end{tcolorbox}
 
 
\begin{exa}
$a_1, a_2, b_1,b_2$を実数とする. 
このとき$\begin{vmatrix}
a_1&a_2 \\
b_1&b_2\\
\end{vmatrix}$
の行列式の絶対値$|a_1b_2 - a_2b_1|$は$(a_1, a_2)$と$(b_1, b_2)$を2辺とする平行四辺形の面積に等しい.

\end{exa}
\section{3次の行列式と内積外積}

\begin{tcolorbox}[
    colback = white,
    colframe = green!35!black,
    fonttitle = \bfseries,
    breakable = true]
    \begin{thm}
$\bm{a}=(a_1, a_2, a_3), \bm{b}=(b_1, b_2, b_3), \bm{c}=(c_1, c_2, c_3)\in \R^3$について, 
$$
\det
\begin{pmatrix}
a_1& a_2 & a_3\\
b_1& b_2 & b_3\\
c_1& c_2 & c_3\\
\end{pmatrix}
=\bm{a} \cdot (\bm{b} \times \bm{c}).
$$
特に
$\bm{a} \cdot (\bm{b} \times \bm{c}) = \bm{b} \cdot (\bm{c} \times \bm{a})=\bm{c} \cdot (\bm{a} \times \bm{b})$である(スカラー3重積とも呼ばれる).
\end{thm}
 \end{tcolorbox}
 
 \begin{tcolorbox}[
    colback = white,
    colframe = green!35!black,
    fonttitle = \bfseries,
    breakable = true]
    \begin{thm}
$\bm{a}=(a_1, a_2, a_3), \bm{b}=(b_1, b_2, b_3), \bm{c}=(c_1, c_2, c_3)\in \R^3$とすると次の値は等しい.
\begin{itemize}
\item $\det
\begin{pmatrix}
a_1& a_2 & a_3\\
b_1& b_2 & b_3\\
c_1& c_2 & c_3\\
\end{pmatrix}$の絶対値.
\item $\bm{a} \cdot (\bm{b} \times \bm{c})$の絶対値.
\item $\bm{a}, \bm{b}, \bm{c}$によって生成される平行6面体の体積.
\end{itemize}
\end{thm}
 \end{tcolorbox}
 
 \end{document}
