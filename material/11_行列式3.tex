\documentclass[dvipdfmx,a4paper,11pt]{article}
\usepackage[utf8]{inputenc}
%\usepackage[dvipdfmx]{hyperref} %リンクを有効にする
\usepackage{url} %同上
\usepackage{amsmath,amssymb} %もちろん
\usepackage{amsfonts,amsthm,mathtools} %もちろん
\usepackage{braket,physics} %あると便利なやつ
\usepackage{bm} %ラプラシアンで使った
\usepackage[top=30truemm,bottom=30truemm,left=25truemm,right=25truemm]{geometry} %余白設定
\usepackage{latexsym} %ごくたまに必要になる
\renewcommand{\kanjifamilydefault}{\gtdefault}
\usepackage{otf} %宗教上の理由でmin10が嫌いなので


\usepackage[all]{xy}
\usepackage{amsthm,amsmath,amssymb,comment}
\usepackage{amsmath}    % \UTF{00E6}\UTF{0095}°\UTF{00E5}\UTF{00AD}\UTF{00A6}\UTF{00E7}\UTF{0094}¨
\usepackage{amssymb}  
\usepackage{color}
\usepackage{amscd}
\usepackage{amsthm}  
\usepackage{wrapfig}
\usepackage{comment}	
\usepackage{graphicx}
\usepackage{setspace}
\usepackage{pxrubrica}
\setstretch{1.2}


\newcommand{\R}{\mathbb{R}}
\newcommand{\Z}{\mathbb{Z}}
\newcommand{\Q}{\mathbb{Q}} 
\newcommand{\N}{\mathbb{N}}
\newcommand{\C}{\mathbb{C}} 
\newcommand{\Sin}{\text{Sin}^{-1}} 
\newcommand{\Cos}{\text{Cos}^{-1}} 
\newcommand{\Tan}{\text{Tan}^{-1}} 
\newcommand{\invsin}{\text{Sin}^{-1}} 
\newcommand{\invcos}{\text{Cos}^{-1}} 
\newcommand{\invtan}{\text{Tan}^{-1}} 
\newcommand{\Area}{\text{Area}}
\newcommand{\vol}{\text{Vol}}
\newcommand{\maru}[1]{\raise0.2ex\hbox{\textcircled{\tiny{#1}}}}
\newcommand{\sgn}{{\rm sgn}}
%\newcommand{\rank}{{\rm rank}}



   %当然のようにやる.
\allowdisplaybreaks[4]
   %もちろん.
%\title{第1回. 多変数の連続写像 (岩井雅崇, 2020/10/06)}
%\author{岩井雅崇}
%\date{2020/10/06}
%ここまで今回の記事関係ない
\usepackage{tcolorbox}
\tcbuselibrary{breakable, skins, theorems}

\theoremstyle{definition}
\newtheorem{thm}{定理}
\newtheorem{lem}[thm]{補題}
\newtheorem{prop}[thm]{命題}
\newtheorem{cor}[thm]{系}
\newtheorem{claim}[thm]{主張}
\newtheorem{dfn}[thm]{定義}
\newtheorem{rem}[thm]{注意}
\newtheorem{exa}[thm]{例}
\newtheorem{conj}[thm]{予想}
\newtheorem{prob}[thm]{問題}
\newtheorem{rema}[thm]{補足}

\DeclareMathOperator{\Ric}{Ric}
\DeclareMathOperator{\Vol}{Vol}
 \newcommand{\pdrv}[2]{\frac{\partial #1}{\partial #2}}
 \newcommand{\drv}[2]{\frac{d #1}{d#2}}
  \newcommand{\ppdrv}[3]{\frac{\partial #1}{\partial #2 \partial #3}}


%ここから本文.
\begin{document}
%\maketitle
\begin{center}
{\Large 第11回. 行列式3 -行列式の基本性質- (三宅先生の本, 3.2, 3.3の内容)} 
\end{center}

\begin{flushright}
 岩井雅崇 2022/06/30
\end{flushright}

一部の内容について, 齋藤正彦著 線型代数学 (東京図書)の第3章を参考にした.

%\section{行列式の基本性質}

\begin{tcolorbox}[
    colback = white,
    colframe = green!35!black,
    fonttitle = \bfseries,
    breakable = true]
    \begin{prop}
$a_1, \ldots, a_{n}$を行ベクトルとし, $n$次正方行列
$A = 
\begin{pmatrix}
a_1 \\ \vdots \\ a_{n}
\end{pmatrix}
$とする.
\begin{enumerate}
\item $\tau$を$n$次の置換とすると
$$
\det \begin{pmatrix}
a_{\tau(1)} \\ \vdots \\ a_{\tau(n)} 
\end{pmatrix}
= 
\sgn(\tau) \det \begin{pmatrix}
a_1 \\ \vdots \\ a_{n}
\end{pmatrix}
= \sgn(\tau) \det(A).
\text{\,\,\,(交代性)}
$$
\item $b_i, c_i$を行ベクトルとし, $\alpha, \beta$を数とすると,
$$
\det \begin{pmatrix}
a_1 \\ \vdots \\ \alpha b_i + \beta c_i \\ \vdots  \\ a_{n}
\end{pmatrix}
= 
\alpha\det \begin{pmatrix}
a_1 \\ \vdots \\  b_i \\ \vdots  \\ a_{n}
\end{pmatrix}
+
\beta
\det \begin{pmatrix}
a_1 \\ \vdots \\ c_i \\ \vdots  \\ a_{n}
\end{pmatrix}
\text{\,\,\,(多重線型性)}
$$
\end{enumerate}

  \end{prop}
 \end{tcolorbox}
 

\begin{tcolorbox}[
    colback = white,
    colframe = green!35!black,
    fonttitle = \bfseries,
    breakable = true]
    \begin{thm}
行ベクトル$x_1, \ldots, x_n$について, 数
$F\begin{pmatrix}
x_1 \\ \vdots \\ x_{n}
\end{pmatrix}$
を対応させる関数$F$を考える.
この$F$が交代性と多重線型性を満たすとき, 
$$
F\begin{pmatrix}
x_1 \\ \vdots \\ x_{n}
\end{pmatrix}
=
F\begin{pmatrix}
f_1 \\ \vdots \\ f_{n}
\end{pmatrix}
\det
\begin{pmatrix}
x_1 \\ \vdots \\ x_{n}
\end{pmatrix}
\text{となる.}
$$
ここで$f_i =
\begin{pmatrix}
0 & \cdots&\overset{i}{\hat{1}}&\cdots &0
\end{pmatrix}
$
という行ベクトルとする.

特に行列$A$に対して数$F(A)$を対応させる関数が, 行に関して交代性と多重線型性を満たすとき
$F(A) =F(E_n) \det(A)$となる.
  \end{thm}
 \end{tcolorbox}
 

\end{document}
