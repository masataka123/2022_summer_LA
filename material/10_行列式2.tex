\documentclass[dvipdfmx,a4paper,11pt]{article}
\usepackage[utf8]{inputenc}
%\usepackage[dvipdfmx]{hyperref} %リンクを有効にする
\usepackage{url} %同上
\usepackage{amsmath,amssymb} %もちろん
\usepackage{amsfonts,amsthm,mathtools} %もちろん
\usepackage{braket,physics} %あると便利なやつ
\usepackage{bm} %ラプラシアンで使った
\usepackage[top=30truemm,bottom=30truemm,left=25truemm,right=25truemm]{geometry} %余白設定
\usepackage{latexsym} %ごくたまに必要になる
\renewcommand{\kanjifamilydefault}{\gtdefault}
\usepackage{otf} %宗教上の理由でmin10が嫌いなので


\usepackage[all]{xy}
\usepackage{amsthm,amsmath,amssymb,comment}
\usepackage{amsmath}    % \UTF{00E6}\UTF{0095}°\UTF{00E5}\UTF{00AD}\UTF{00A6}\UTF{00E7}\UTF{0094}¨
\usepackage{amssymb}  
\usepackage{color}
\usepackage{amscd}
\usepackage{amsthm}  
\usepackage{wrapfig}
\usepackage{comment}	
\usepackage{graphicx}
\usepackage{setspace}
\usepackage{pxrubrica}
\setstretch{1.2}


\newcommand{\R}{\mathbb{R}}
\newcommand{\Z}{\mathbb{Z}}
\newcommand{\Q}{\mathbb{Q}} 
\newcommand{\N}{\mathbb{N}}
\newcommand{\C}{\mathbb{C}} 
\newcommand{\Sin}{\text{Sin}^{-1}} 
\newcommand{\Cos}{\text{Cos}^{-1}} 
\newcommand{\Tan}{\text{Tan}^{-1}} 
\newcommand{\invsin}{\text{Sin}^{-1}} 
\newcommand{\invcos}{\text{Cos}^{-1}} 
\newcommand{\invtan}{\text{Tan}^{-1}} 
\newcommand{\Area}{\text{Area}}
\newcommand{\vol}{\text{Vol}}
\newcommand{\maru}[1]{\raise0.2ex\hbox{\textcircled{\tiny{#1}}}}
\newcommand{\sgn}{{\rm sgn}}
%\newcommand{\rank}{{\rm rank}}



   %当然のようにやる.
\allowdisplaybreaks[4]
   %もちろん.
%\title{第1回. 多変数の連続写像 (岩井雅崇, 2020/10/06)}
%\author{岩井雅崇}
%\date{2020/10/06}
%ここまで今回の記事関係ない
\usepackage{tcolorbox}
\tcbuselibrary{breakable, skins, theorems}

\theoremstyle{definition}
\newtheorem{thm}{定理}
\newtheorem{lem}[thm]{補題}
\newtheorem{prop}[thm]{命題}
\newtheorem{cor}[thm]{系}
\newtheorem{claim}[thm]{主張}
\newtheorem{dfn}[thm]{定義}
\newtheorem{rem}[thm]{注意}
\newtheorem{exa}[thm]{例}
\newtheorem{conj}[thm]{予想}
\newtheorem{prob}[thm]{問題}
\newtheorem{rema}[thm]{補足}

\DeclareMathOperator{\Ric}{Ric}
\DeclareMathOperator{\Vol}{Vol}
 \newcommand{\pdrv}[2]{\frac{\partial #1}{\partial #2}}
 \newcommand{\drv}[2]{\frac{d #1}{d#2}}
  \newcommand{\ppdrv}[3]{\frac{\partial #1}{\partial #2 \partial #3}}


%ここから本文.
\begin{document}
%\maketitle
\begin{center}
{\Large 第10回. 行列式2 -行列式の計算方法- (三宅先生の本, 3.2, 3.3の内容)} 
\end{center}

\begin{flushright}
 岩井雅崇 2022/06/23
\end{flushright}

%この資料では第10回授業と第11回授業の内容を取り扱います.

\section{行列式}

\begin{tcolorbox}[
    colback = white,
    colframe = green!35!black,
    fonttitle = \bfseries,
    breakable = true]
    \begin{dfn}
$n$次正方行列$A = (a_{ij})$について
$$
\det(A) =  \sum_{\sigma \in S_n}\sgn(\sigma) 
a_{1 \sigma(1)} a_{2 \sigma(2)} \cdots a_{n \sigma(n)} 
\text{を\underline{$A$の行列式}と言う.}
$$
 $A$の行列式は$\det(A)$, $|A|$, 
$
\begin{vmatrix}
a_{11}& a_{12} & \cdots &a_{1n} \\
a_{21}& a_{22} & \cdots &a_{2n} \\
\vdots& \vdots	&	\ddots   &	\vdots \\
a_{n1}& a_{n2} & \cdots &a_{nn} \\
\end{vmatrix}
$
ともかく.
  \end{dfn}
 \end{tcolorbox}

\begin{exa}
\label{2jidet}
$A = 
  \begin{pmatrix}
a_{11}& a_{12}\\
a_{21}& a_{22}\\
 \end{pmatrix} 
$
とすると$\det(A) = a_{11}a_{22} - a_{12}a_{21}$である.

(証).
$S_2 = \left\{   \begin{pmatrix}
1& 2\\
1& 2\\
 \end{pmatrix} , 
   \begin{pmatrix}
1&2\\
2& 1\\
 \end{pmatrix} 
  \right\}$
  であるので, $A$の行列式は
  \begin{align*}
  \det(A) &= 
  \sgn \begin{pmatrix}
1& 2\\
1& 2\\
 \end{pmatrix} a_{11}a_{22}
 +
  \sgn \begin{pmatrix}
1& 2\\
2& 1\\
 \end{pmatrix} a_{12}a_{21}
 =
 a_{11}a_{22} - a_{12}a_{21}.
  \end{align*}

\end{exa}

\begin{exa}
$A = 
  \begin{pmatrix}
a_{11}& a_{12} & a_{13}\\
a_{21}& a_{22} & a_{23}\\
a_{31}& a_{32} & a_{33}\\
 \end{pmatrix} 
$
の行列式を求める.

$S_3 = \left\{   
\begin{pmatrix}
1& 2 &3\\
1& 2 &3\\
 \end{pmatrix} , 
\begin{pmatrix}
1& 2 &3\\
2& 1 &3\\
 \end{pmatrix} , 
\begin{pmatrix}
1& 2 &3\\
1& 3 &2\\
 \end{pmatrix} , 
 \begin{pmatrix}
1& 2 &3\\
3& 2 &1\\
 \end{pmatrix} , 
 \begin{pmatrix}
1& 2 &3\\
2& 3 &1\\
 \end{pmatrix} , 
 \begin{pmatrix}
1& 2 &3\\
3& 1 &2\\
 \end{pmatrix}
  \right\}$
  であるので, $A$の行列式は
  
  \begin{align*}
  \det(A) 
  &= 
  \sgn \begin{pmatrix}
1& 2 &3\\
1& 2 &3\\
 \end{pmatrix}
 a_{11}a_{22}a_{33}
 +
   \sgn \begin{pmatrix}
1& 2 &3\\
2& 1 &3\\
 \end{pmatrix} 
 a_{12}a_{21}a_{33}
 +
\sgn \begin{pmatrix}
1& 2 &3\\
1& 3 &2\\
 \end{pmatrix} 
 a_{11}a_{23}a_{32} 
  \\%%%
  &+
\sgn  \begin{pmatrix}
1& 2 &3\\
3& 2 &1\\
 \end{pmatrix}
 a_{13}a_{22}a_{31} 
 +
\sgn  \begin{pmatrix}
1& 2 &3\\
2& 3 &1\\
 \end{pmatrix}
 a_{12}a_{23}a_{31} 
  +
\sgn   \begin{pmatrix}
1& 2 &3\\
3& 1 &2\\
 \end{pmatrix}
 a_{13}a_{21}a_{32} 
 \\%%
   &= a_{11}a_{22}a_{33}- a_{12}a_{21}a_{33}- a_{11}a_{23}a_{32} 
   - a_{13}a_{22}a_{31}  + a_{12}a_{23}a_{31}  +  a_{13}a_{21}a_{32} 
  \end{align*}
  以上より
  $
 \det(A)= 
 a_{11}a_{22}a_{33}+ a_{12}a_{23}a_{31}  +  a_{13}a_{21}a_{32} 
- a_{11}a_{23}a_{32}     - a_{13}a_{22}a_{31}  - a_{12}a_{21}a_{33}
  $
  である.
\end{exa}
\begin{rema}
2次正方行列や3次正方行列の行列式は視覚的に綺麗に表わすことができる(サラスの公式と呼ばれる).
\end{rema}

\section{行列式の基本性質}

\begin{tcolorbox}[
    colback = white,
    colframe = green!35!black,
    fonttitle = \bfseries,
    breakable = true]
    \begin{thm}
   \label{determinant}
$A,B$を$n$次正方行列とする.
\begin{enumerate}
\item $\det({}^{t} A) =\det(A)$.
\item $\det(AB)=(\det(A))(\det(B)) = \det(BA)$.
\item $\det(A) \neq 0$であることと$A$が正則であることは同値.
\item 
$
\begin{vmatrix}
a_{11}& a_{12} & \cdots &a_{1n} \\
0 	   & a_{22} & \cdots &a_{2n} \\
\vdots& \vdots	&	\ddots   &	\vdots \\
0	& a_{n2} & \cdots &a_{nn} \\
\end{vmatrix}
=a_{11}
\begin{vmatrix}
 a_{22} & \cdots &a_{2n} \\
 \vdots	&	\ddots   &	\vdots \\
 a_{n2} & \cdots &a_{nn} \\
\end{vmatrix}
$.
\item 
$
\begin{vmatrix}
a_{11}& a_{12} & a_{13} &\cdots &a_{1n-1}&a_{1n} \\
0 	   & a_{22} & a_{23} &\cdots&a_{2n-1} &a_{2n} \\
0 	   & 0  		& a_{33} &\cdots &a_{3n-1}&a_{3n} \\
\vdots& \vdots	&  \ddots &\ddots&	\vdots  &	\vdots \\
0	& 0      		&      	&\ddots	&a_{n-1n-1} &a_{n-1n} \\
0	& 0      		&     \cdots	&	\cdots&0		&a_{nn} \\
\end{vmatrix}
=a_{11}a_{22}\cdots a_{nn}
$.
%以下行ベクトル$a_1, \cdots, a_n$を用いて$A = \begin{pmatrix}a_1 \\a_2 \\\vdots\\a_n\end{pmatrix}$とかけているとする.
\item  1つの行を$c$倍すると行列式は$c$倍される:
$
\begin{vmatrix}
a_{11}&  \cdots &a_{1n} \\
\vdots&	 	  &	\vdots \\
ca_{i1} & \cdots &ca_{in} \\
\vdots& 		   &	\vdots \\
a_{n1}	& \cdots &a_{nn} \\
\end{vmatrix}
=
c
\begin{vmatrix}
a_{11}&  \cdots &a_{1n} \\
\vdots&	 	  &	\vdots \\
a_{i1} & \cdots & a_{in} \\
\vdots& 		   &	\vdots \\
a_{n1}	& \cdots &a_{nn} \\
\end{vmatrix}
$.
\item  
$
\begin{vmatrix}
a_{11}&  \cdots &a_{1n} \\
\vdots&	 	  &	\vdots \\
b_{i1} + c_{i1} & \cdots &b_{in} + c_{in} \\
\vdots& 		   &	\vdots \\
a_{n1}	& \cdots &a_{nn} \\
\end{vmatrix}
=
\begin{vmatrix}
a_{11}&  \cdots &a_{1n} \\
\vdots&	 	  &	\vdots \\
b_{i1}  & \cdots &b_{in}  \\
\vdots& 		   &	\vdots \\
a_{n1}	& \cdots &a_{nn} \\
\end{vmatrix}
+
\begin{vmatrix}
a_{11}&  \cdots &a_{1n} \\
\vdots&	 	  &	\vdots \\
c_{i1} & \cdots & c_{in} \\
\vdots& 		   &	\vdots \\
a_{n1}	& \cdots &a_{nn} \\
\end{vmatrix}
$.
\item 2つの行を入れ替えたら, 行列式は$-1$倍される:
$$
\begin{vmatrix}
a_{11}&  \cdots &a_{1n} \\
\vdots& 		   &	\vdots \\
a_{j1} & \cdots & a_{jn} \\
\vdots&	 	  &	\vdots \\
a_{i1} & \cdots & a_{in} \\
\vdots& 		   &	\vdots \\
a_{n1}	& \cdots &a_{nn} \\
\end{vmatrix}
= (-1)
\begin{vmatrix}
a_{11}&  \cdots &a_{1n} \\
\vdots&	 	  &	\vdots \\
a_{i1} & \cdots & a_{in} \\
\vdots& 		   &	\vdots \\
a_{j1} & \cdots & a_{jn} \\
\vdots& 		   &	\vdots \\
a_{n1}	& \cdots &a_{nn} \\
\end{vmatrix}.
$$
\item 第$i$行の$c$倍を第$j$行に加えても行列式は変わらない:
$$
\begin{vmatrix}
a_{11}&  \cdots &a_{1n} \\
\vdots& 		   &	\vdots \\
a_{j1} +ca_{i1}& \cdots & a_{jn} +ca_{in}\\
\vdots& 		   &	\vdots \\
a_{n1}	& \cdots &a_{nn} \\
\end{vmatrix}
= 
\begin{vmatrix}
a_{11}&  \cdots &a_{1n} \\
\vdots& 		   &	\vdots \\
a_{j1} & \cdots & a_{jn} \\
\vdots& 		   &	\vdots \\
a_{n1}	& \cdots &a_{nn} \\
\end{vmatrix}.
$$
\item 列ベクトルに関して上の6から9と同様のことが成り立つ.
\end{enumerate}
  \end{thm}
 \end{tcolorbox}
 
 \begin{tcolorbox}[
    colback = white,
    colframe = green!35!black,
    fonttitle = \bfseries,
    breakable = true]
    \begin{cor}
$A,B$を$n$次正方行列とする. $AB=E_n$ならば, $A$は正則で$B$は$A$の逆行列.
  \end{cor}
 \end{tcolorbox}
 
 \section{行列式の計算方法}
 定理\ref{determinant}を用いると行列式を比較的簡単に計算できる.
 
  \begin{exa}
 $
 \begin{pmatrix}
 1&3&4\\
 -2&-5&7\\
 -3&2&-1\\
 \end{pmatrix}
$
の行列式を定理\ref{determinant}を用いて計算すると次の通りになる.

\begin{align*}
 &\begin{vmatrix}
 1&3&4\\
 -2&-5&7\\
 -3&2&-1\\
 \end{vmatrix}
 \overset{\text{定理\ref{determinant}.(9)} } {=}
 \begin{vmatrix}
 1&3&4\\
 0&1&15\\
 0&11&11\\
 \end{vmatrix}
 \overset{\text{定理\ref{determinant}.(4)} } {=}
 1
 \begin{vmatrix}
1&15\\
11&11\\
 \end{vmatrix}
  \overset{\text{定理\ref{determinant}.(6)} } {=}
 11
 \begin{vmatrix}
1&15\\
1&1\\
 \end{vmatrix}\overset{\text{例 \ref{2jidet}} } {=}
 11 \left\{ 1 \times 1 - 15 \times 1 \right\} 
 =
 -154.
 \end{align*}
 
\end{exa}



 \begin{exa}
 $
 \begin{pmatrix}
 2&-4&-5&3\\
 -6&13&14&1\\
 1&-2&-2&-8\\
 2&-5&0&5\\
 \end{pmatrix}
$
の行列式を定理\ref{determinant}を用いて計算すると次の通りになる.
\begin{align*}
 &\begin{vmatrix}
 2&-4&-5&3\\
 -6&13&14&1\\
 1&-2&-2&-8\\
 2&-5&0&5\\
 \end{vmatrix}
 \overset{\text{定理\ref{determinant}.(8)} } {=}
 (-1)
  \begin{vmatrix}
   1&-2&-2&-8\\
 -6&13&14&1\\
 2&-4&-5&3\\
 2&-5&0&5\\
 \end{vmatrix}
  \overset{\text{定理\ref{determinant}.(9)} }  {=}
 (-1)
  \begin{vmatrix}
   1&-2&-2&-8\\
 0&1 &2  &-47\\
 0& 0&-1&19\\
 0&-1&4&21\\
 \end{vmatrix}
\\ %%%
& \overset{\text{定理\ref{determinant}.(4)} } {=}
 (-1)
  \begin{vmatrix}
1 &2  &-47\\
 0&-1&19\\
-1&4&21\\
 \end{vmatrix}
  \overset{\text{定理\ref{determinant}.(9)} } {=}
   (-1)
  \begin{vmatrix}
1 &2  &-47\\
 0&-1&19\\
 0&6&-26\\
 \end{vmatrix}
 \overset{\text{定理\ref{determinant}.(4)} } {=}
  (-1)
    \begin{vmatrix}
-1&19\\
6&-26\\
 \end{vmatrix}
 \\ %%
 & \overset{\text{例 \ref{2jidet}} } {=}
 (-1)\left\{(-1)\times (-26) - 6\times 19\right\} = 88.
\end{align*}

 \end{exa}

\section{演習問題}
演習問題の解答は授業動画にあります.

1. 行列式
$
\begin{vmatrix}
 0& -3& -6 &15 \\
 -2& 5& 14 &4 \\
 1& -3& -2 &5 \\
 15 & 10& 10 &-5 \\
 \end{vmatrix} 
 $
 を計算せよ.


 

\end{document}
