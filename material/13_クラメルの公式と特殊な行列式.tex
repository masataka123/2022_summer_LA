\documentclass[dvipdfmx,a4paper,11pt]{article}
\usepackage[utf8]{inputenc}
%\usepackage[dvipdfmx]{hyperref} %リンクを有効にする
\usepackage{url} %同上
\usepackage{amsmath,amssymb} %もちろん
\usepackage{amsfonts,amsthm,mathtools} %もちろん
\usepackage{braket,physics} %あると便利なやつ
\usepackage{bm} %ラプラシアンで使った
\usepackage[top=30truemm,bottom=30truemm,left=25truemm,right=25truemm]{geometry} %余白設定
\usepackage{latexsym} %ごくたまに必要になる
\renewcommand{\kanjifamilydefault}{\gtdefault}
\usepackage{otf} %宗教上の理由でmin10が嫌いなので


\usepackage[all]{xy}
\usepackage{amsthm,amsmath,amssymb,comment}
\usepackage{amsmath}    % \UTF{00E6}\UTF{0095}°\UTF{00E5}\UTF{00AD}\UTF{00A6}\UTF{00E7}\UTF{0094}¨
\usepackage{amssymb}  
\usepackage{color}
\usepackage{amscd}
\usepackage{amsthm}  
\usepackage{wrapfig}
\usepackage{comment}	
\usepackage{graphicx}
\usepackage{setspace}
\usepackage{pxrubrica}
\setstretch{1.2}


\newcommand{\R}{\mathbb{R}}
\newcommand{\Z}{\mathbb{Z}}
\newcommand{\Q}{\mathbb{Q}} 
\newcommand{\N}{\mathbb{N}}
\newcommand{\C}{\mathbb{C}} 
\newcommand{\Sin}{\text{Sin}^{-1}} 
\newcommand{\Cos}{\text{Cos}^{-1}} 
\newcommand{\Tan}{\text{Tan}^{-1}} 
\newcommand{\invsin}{\text{Sin}^{-1}} 
\newcommand{\invcos}{\text{Cos}^{-1}} 
\newcommand{\invtan}{\text{Tan}^{-1}} 
\newcommand{\Area}{\text{Area}}
\newcommand{\vol}{\text{Vol}}
\newcommand{\maru}[1]{\raise0.2ex\hbox{\textcircled{\tiny{#1}}}}
\newcommand{\sgn}{{\rm sgn}}
%\newcommand{\rank}{{\rm rank}}



   %当然のようにやる.
\allowdisplaybreaks[4]
   %もちろん.
%\title{第1回. 多変数の連続写像 (岩井雅崇, 2020/10/06)}
%\author{岩井雅崇}
%\date{2020/10/06}
%ここまで今回の記事関係ない
\usepackage{tcolorbox}
\tcbuselibrary{breakable, skins, theorems}

\theoremstyle{definition}
\newtheorem{thm}{定理}
\newtheorem{lem}[thm]{補題}
\newtheorem{prop}[thm]{命題}
\newtheorem{cor}[thm]{系}
\newtheorem{claim}[thm]{主張}
\newtheorem{dfn}[thm]{定義}
\newtheorem{rem}[thm]{注意}
\newtheorem{exa}[thm]{例}
\newtheorem{conj}[thm]{予想}
\newtheorem{prob}[thm]{問題}
\newtheorem{rema}[thm]{補足}

\DeclareMathOperator{\Ric}{Ric}
\DeclareMathOperator{\Vol}{Vol}
 \newcommand{\pdrv}[2]{\frac{\partial #1}{\partial #2}}
 \newcommand{\drv}[2]{\frac{d #1}{d#2}}
  \newcommand{\ppdrv}[3]{\frac{\partial #1}{\partial #2 \partial #3}}


%ここから本文.
\begin{document}
%\maketitle
\begin{center}
{\Large 第13回. クラメルの公式と特殊な行列式 (三宅先生の本, 3.4, 3.5の内容)} 
\end{center}

\begin{flushright}
 岩井雅崇 2022/07/14
\end{flushright}

この授業で行う内容は理解しなくても構いません(結構マニアックな話題を扱います). 
また覚える必要もございません. 
\section{クラメルの公式}

\begin{tcolorbox}[
    colback = white,
    colframe = green!35!black,
    fonttitle = \bfseries,
    breakable = true]
    \begin{thm}
$A$を正則な$n$次正方行列とし, 列ベクトル$a_1, \ldots, a_{n}$を用いて
$
A = 
\begin{pmatrix}
a_1 & \cdots & a_{n}
\end{pmatrix}
$
と表されているとする.
このとき連立1次方程式$A \bm{x} =\bm{b}$の解は次のようになる.
$$
\bm{x}= \begin{pmatrix}
x_1 \\ \vdots \\ x_{n}
\end{pmatrix}, 
x_i = \frac{\det
\begin{pmatrix}
a_1 & \cdots& \bm{b}&\cdots & a_{n}
\end{pmatrix}
}{\det A}.
$$
    \end{thm}
 \end{tcolorbox}
\begin{exa}
$
A = 
\begin{pmatrix}
5 &1\\
3&2 \\ 
\end{pmatrix}
$, $
\bm{b} = 
\begin{pmatrix}
3\\
2 \\ 
\end{pmatrix}
$
とする. 
連立1次方程式$A \bm{x} =\bm{b}$の解を
$
\bm{x}= \begin{pmatrix}
x_1 \\x_2
\end{pmatrix} 
$
とすると,
$$
x_1 = \frac{\det
\begin{pmatrix}
 \bm{b}& a_{2}
\end{pmatrix}
}{\det A}
= 
\frac{ 
\begin{vmatrix}
3&1\\
2&2 \\
\end{vmatrix}
}
{
\begin{vmatrix}
5&1\\
3&2 \\
\end{vmatrix}
}
=\frac{4}{7} 
\text{, }
x_2 = \frac{\det
\begin{pmatrix}
a_{1}& \bm{b}
\end{pmatrix}
}{\det A}
= 
\frac{ 
\begin{vmatrix}
5&3\\
3&2 \\
\end{vmatrix}
}
{
\begin{vmatrix}
5&1\\
3&2 \\
\end{vmatrix}
}
=\frac{1}{7}
\text{となる.}
$$
\end{exa}

\section{特殊な行列式}
\begin{tcolorbox}[
    colback = white,
    colframe = green!35!black,
    fonttitle = \bfseries,
    breakable = true]
    \begin{thm}
  \begin{enumerate}
\item (ヴァンデルモンドの行列式)
\begin{align*}
 &
 \begin{vmatrix}
1   &   1     & \cdots &  1  \\
x_1& x_2 & \cdots &x_n \\
{x_1}^{2}& {x_2}^{2} & \cdots &{x_n}^{2} \\
\vdots & \vdots    &  &\vdots \\
{x_1}^{n-1}& {x_2}^{n-1} & \cdots &{x_n}^{n-1} \\
\end{vmatrix}
= \prod_{1 \leqq i < j \leqq n} (x_j - x_i).
\\%%
%&=(x_2 - x_1)  (x_3 - x_1)  \cdots  (x_n - x_1)  (x_3- x_2)    \cdots  (x_n - x_2)  (x_4 - x_3)\cdots (x_n - x_{n-1})
%\begin{matrix}=(x_2 - x_1)  (x_3 - x_1)  \cdots  (x_n - x_1) \\\times (x_3- x_2)  (x_4 - x_2)  \cdots  (x_n - x_2) \\\times \cdots \\\times (x_n - x_{n-1}) \\ \end{matrix}
%\begin{matrix}(x_2 - x_1) & \times&  (x_3 - x_1) & \times&  \cdots &\times & (x_n - x_1) \\\end{matrix}
\end{align*}
\item (ヴァンデルモンドの行列式の応用) $b_1, \ldots, b_n, c_1, \ldots, c_n$を実数とし, $b_1, \ldots, b_n$は相異なると仮定する.
 このとき実数係数の$n$次式$f(x) = x^{n} +  a_1 x^{n-1} +  \cdots +  a_{n-1} x + a_{n}$があって, 任意の$i = 1, \ldots, n$について$f(b_i) =c_i $となる.

  \end{enumerate}
    \end{thm}
 \end{tcolorbox}
$\prod_{1 \leqq i < j \leqq n}$は積の記号で, $\prod_{1 \leqq i < j \leqq n} (x_j - x_i)$は「$1 \leqq i < j \leqq n$を満たす$(i,j)$について$ (x_j - x_i)$を全てかけた数」を表している.



\begin{tcolorbox}[
    colback = white,
    colframe = green!35!black,
    fonttitle = \bfseries,
    breakable = true]
    \begin{thm}
$$
 \begin{vmatrix}
a_0   &   -1   &0 &0 & \cdots &  0 \\
a_1   &   x    &-1  &0 & \cdots &  0 \\
a_2   &   0    & x   &-1 & \cdots &  0 \\
\vdots  &    \vdots   &   &\ddots & \ddots &  0 \\
a_{n-1}&   0    & 0   &                  & x&  -1\\
a_n   &   0   & 0   &0 & \cdots &  x\\
\end{vmatrix}
=a_0 x^{n} +  a_1 x^{n-1} +  \cdots +  a_{n-1} x + a_{n}.
$$
    \end{thm}
 \end{tcolorbox}
 
\section{終結式と判別式}
以下の内容は「永田雅宜著 理系のための線型代数の基礎 (紀伊國屋書店)」の第3章に基づく.

\begin{tcolorbox}[
    colback = white,
    colframe = green!35!black,
    fonttitle = \bfseries,
    breakable = true]
    \begin{dfn}
$a_0, a_1, \ldots, a_n$を複素数とし, $f(x) = a_0 x^{n} +  a_1 x^{n-1} +  \cdots +  a_{n-1} x + a_{n}$とする(ただし$a_0 \neq 0$とする). $f(x)=0$の解を$\alpha_{1}, \ldots, \alpha_{n}$とするとき, 
$$
D = a_{0}^{2n -2} \prod_{1 \leqq i < j \leqq n} (\alpha_j - \alpha_i)^2
\text{ を\underline{$f(x)$の判別式}という.}
$$
    \end{dfn}
 \end{tcolorbox}
 簡単にわかることとして, 「$D \neq 0$ $\Leftrightarrow$ $f(x)=0$の解が相異なる」である.
\begin{exa}
$f(x) = a_0 x^2 + a_1 x + a_2$の判別式$D$を求める(ただし$a_0 \neq 0$とする). 
$\alpha_1, \alpha_2$を$f(x)=0$の解とすると, 解と係数の関係から
$$
a_0 x^2 + a_1 x + a_2 = a_0 (x - \alpha_1) (x- \alpha_2)
$$
であるので, $-a_ 1 = a_0 (\alpha_1 + \alpha_2), a_2 = a_0\alpha_1 \alpha_2$となる. よって
$$
D = a_{0}^{2} (\alpha_2 - \alpha_1)^2 = a_{0}^{2}\{ (\alpha_1 + \alpha_2)^2 - 4 \alpha_1 \alpha_2 \} = a_{1}^{2} - 4 a_0 a_{2}.
 $$
\end{exa}

\begin{tcolorbox}[
    colback = white,
    colframe = green!35!black,
    fonttitle = \bfseries,
    breakable = true]
    \begin{dfn}
複素係数多項式$f(x) = a_0 x^{n} +  a_1 x^{n-1} +  \cdots +  a_{n-1} x + a_{n}$, 
$g(x) = b_0 x^{m} +  b_1 x^{m-1} +  \cdots +  b_{m-1} x + b_{m}$
(ただし$a_0 \neq 0, b_0 \neq 0$)について, $m+n$次正方行列を次で定める.
$$
 \begin{pmatrix}
 a_0 	& a_1& \cdots &a_n 	&0		&0 	     & \cdots&0 \\
0   		& a_0 & a_1		& \cdots &a_n 	&0 	      & \cdots&0 \\
 0 		& 0      & a_0 		&    a_1	& \cdots &a_n  & \cdots&0 \\
\vdots  & \vdots  &      \ddots 	&     \ddots &  \ddots     &   & \ddots&\vdots \\
   0		 & 0         &	 \cdots	&    0	& a_0 &a_1 & \cdots &a_n \\
 b_0 	& b_1& b_2 &  \cdots  	& \cdots		&b_m	     & \cdots&0 \\
%0		& b_0 	& b_1& b_2 &  \cdots  	&b_m 	    &      &0 \\
\vdots 	&    \ddots 	&   \ddots    &   \ddots     & 		&   	    & \ddots&\vdots \\
0		&     \cdots 	&      b_0 & b_1 &  b_2  	&  \cdots& \cdots&b_m \\
 \end{pmatrix}
$$
この行列の行列式を\underline{$f,g$の終結式と言い, $R(f,g)$と表す.}
    \end{dfn}
 \end{tcolorbox}


\begin{tcolorbox}[
    colback = white,
    colframe = green!35!black,
    fonttitle = \bfseries,
    breakable = true]
    \begin{thm}
    \begin{enumerate}
\item $f(x) =0$の解を$\alpha_1, \ldots, \alpha_n$とし, $g(x) =0$の解を$\beta_1, \ldots, \beta_m$とすると
$$
R(f,g) = a_{0}^{m}b_{0}^{n} \prod_{1 \leqq i \leqq n, 1\leqq j \leqq m} (\alpha_i - \beta_j)
=a_{0}^{m} \prod_{1 \leqq i \leqq n} g(\alpha_{i})\text{である.}
$$
特に$R(f,g)=0$は$f(x)=g(x)=0$が共通解を持つことと同値である.
\item $f'$を$f$の微分とすると, 
$$
R(f, f') = (-1)^{\frac{n(n-1)}{2}} a_0 D.
$$
特に$f(x)$の判別式$D$は$a_0, \ldots, a_n$の式でかける. また$R(f,f')=0$は$f(x)=0$が重根を持つことと同値である.
    \end{enumerate}

    \end{thm}
 \end{tcolorbox}
 
 
 \end{document}
