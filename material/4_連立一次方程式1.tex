\documentclass[dvipdfmx,a4paper,11pt]{article}
\usepackage[utf8]{inputenc}
%\usepackage[dvipdfmx]{hyperref} %リンクを有効にする
\usepackage{url} %同上
\usepackage{amsmath,amssymb} %もちろん
\usepackage{amsfonts,amsthm,mathtools} %もちろん
\usepackage{braket,physics} %あると便利なやつ
\usepackage{bm} %ラプラシアンで使った
\usepackage[top=30truemm,bottom=30truemm,left=25truemm,right=25truemm]{geometry} %余白設定
\usepackage{latexsym} %ごくたまに必要になる
\renewcommand{\kanjifamilydefault}{\gtdefault}
\usepackage{otf} %宗教上の理由でmin10が嫌いなので


\usepackage[all]{xy}
\usepackage{amsthm,amsmath,amssymb,comment}
\usepackage{amsmath}    % \UTF{00E6}\UTF{0095}°\UTF{00E5}\UTF{00AD}\UTF{00A6}\UTF{00E7}\UTF{0094}¨
\usepackage{amssymb}  
\usepackage{color}
\usepackage{amscd}
\usepackage{amsthm}  
\usepackage{wrapfig}
\usepackage{comment}	
\usepackage{graphicx}
\usepackage{setspace}
\usepackage{pxrubrica}
\setstretch{1.2}


\newcommand{\R}{\mathbb{R}}
\newcommand{\Z}{\mathbb{Z}}
\newcommand{\Q}{\mathbb{Q}} 
\newcommand{\N}{\mathbb{N}}
\newcommand{\C}{\mathbb{C}} 
\newcommand{\Sin}{\text{Sin}^{-1}} 
\newcommand{\Cos}{\text{Cos}^{-1}} 
\newcommand{\Tan}{\text{Tan}^{-1}} 
\newcommand{\invsin}{\text{Sin}^{-1}} 
\newcommand{\invcos}{\text{Cos}^{-1}} 
\newcommand{\invtan}{\text{Tan}^{-1}} 
\newcommand{\Area}{\text{Area}}
\newcommand{\vol}{\text{Vol}}
\newcommand{\maru}[1]{\raise0.2ex\hbox{\textcircled{\tiny{#1}}}}



   %当然のようにやる.
\allowdisplaybreaks[4]
   %もちろん.
%\title{第1回. 多変数の連続写像 (岩井雅崇, 2020/10/06)}
%\author{岩井雅崇}
%\date{2020/10/06}
%ここまで今回の記事関係ない
\usepackage{tcolorbox}
\tcbuselibrary{breakable, skins, theorems}

\theoremstyle{definition}
\newtheorem{thm}{定理}
\newtheorem{lem}[thm]{補題}
\newtheorem{prop}[thm]{命題}
\newtheorem{cor}[thm]{系}
\newtheorem{claim}[thm]{主張}
\newtheorem{dfn}[thm]{定義}
\newtheorem{rem}[thm]{注意}
\newtheorem{exa}[thm]{例}
\newtheorem{conj}[thm]{予想}
\newtheorem{prob}[thm]{問題}
\newtheorem{rema}[thm]{補足}

\DeclareMathOperator{\Ric}{Ric}
\DeclareMathOperator{\Vol}{Vol}
 \newcommand{\pdrv}[2]{\frac{\partial #1}{\partial #2}}
 \newcommand{\drv}[2]{\frac{d #1}{d#2}}
  \newcommand{\ppdrv}[3]{\frac{\partial #1}{\partial #2 \partial #3}}


%ここから本文.
\begin{document}
%\maketitle
\begin{center}
{\Large 第4回. 連立1次方程式1-基本変形- (三宅先生の本, 1.4, 2.1の内容)}
\end{center}

\begin{flushright}
 岩井雅崇 2022/05/12
\end{flushright}

\section{連立1次方程式}

 \begin{tcolorbox}[
    colback = white,
    colframe = green!35!black,
    fonttitle = \bfseries,
    breakable = true]
    \begin{dfn}[係数行列, 拡大係数行列]
$m$個の式からなる$n$変数連立1次方程式
\begin{equation*}
%\label{equation}
\left\{ 
\begin{matrix}
a_{11}x_1&+& a_{12} x_2& +&\cdots &+&a_{1n}x_n &= &b_1 \\
a_{21}x_1&+& a_{22} x_2& +&\cdots &+&a_{2n}x_n &= &b_2 \\
\vdots		&& 	\vdots				 && 		& &\vdots&&\vdots	\\
a_{m1}x_1&+& a_{m2} x_2& +&\cdots &+&a_{mn}x_n &= &b_m \\
\end{matrix}
\right.
\text{に対して}
\end{equation*}
$$
A=\begin{pmatrix}
a_{11}& a_{12} & \cdots &a_{1n} \\
a_{21}& a_{22} & \cdots &a_{2n} \\
\vdots& \vdots	&	\ddots   &	\vdots \\
a_{m1}& a_{m2} & \cdots &a_{mn} \\
\end{pmatrix}
\bm{x} =\begin{pmatrix}
x_1\\x_2\\\vdots\\x_n
\end{pmatrix}
\bm{b} =\begin{pmatrix}
b_1\\b_2\\\vdots\\b_m
\end{pmatrix}
\text{とおく.}
$$
行列$A$を連立1次方程式の\underline{係数行列}といい, 
$$
[A : \bm{b}] = \begin{pmatrix}
a_{11}& a_{12} & \cdots &a_{1n} & b_1\\
a_{21}& a_{22} & \cdots &a_{2n} &b_2\\
\vdots& \vdots	&	\ddots   &	\vdots&\vdots \\
a_{m1}& a_{m2} & \cdots &a_{mn}&b_m \\
\end{pmatrix}
\text{を連立1次方程式の\underline{拡大係数行列}という.}
$$
  \end{dfn}
 \end{tcolorbox}
 これにより上の連立1次方程式は$A\bm{x}=\bm{b}$とかける.

 \begin{exa}
 連立1次方程式
 $
 \left\{ 
\begin{matrix}
2x&+&3y& = &7 \\
x&-&4y& = &9 \\
\end{matrix}
\right.
 $
 について, 係数行列は
 $A = 
 \begin{pmatrix}
 2 & 3 \\
 1 & -4
 \end{pmatrix}
 $
 で, 拡大係数行列は
 $[A : \bm{b}] = 
  \begin{pmatrix}
 2 & 3  &7\\
 1 & -4 &9
 \end{pmatrix}
 $
 である.
 \end{exa}
 
  \begin{exa}
 連立1次方程式
 $
 \left\{ 
\begin{matrix}
3x_1&-&2x_2& +& x_3 &+& 4x_4 &=& 7 \\
x_1 &  & 	   & -& 3x_3 &+& x_4 &=& 5 \\
2x_1&-& x_2& +& 9x_3 & & 	 &=& 0 \\
\end{matrix}
\right.
 $
 について, \\
 係数行列は
 $A = 
 \begin{pmatrix}
 3 & -2  & 1&4\\
 1 & 0   & -3&1\\
2 & -1  & 9&0\\
 \end{pmatrix}
 $
 で, 拡大係数行列は
 $[A : \bm{b}] = 
 \begin{pmatrix}
 3 & -2  & 1&4 & 7\\
 1 & 0   & -3&1 &5\\
2 & -1  & 9&0 & 0\\
 \end{pmatrix}
 $
 である.
 \end{exa}
 
 \section{行列の基本変形}
  \begin{tcolorbox}[
    colback = white,
    colframe = green!35!black,
    fonttitle = \bfseries,
    breakable = true]
    \begin{dfn}[行列の基本変形]
 行列の次の3つの変形を(行)基本変形という.
 \begin{enumerate}
 \item 1つの行を何倍か($\neq 0$倍)する.
 \item 2つの行を入れ替える.
 \item1つの行に他の行の何倍かを加える.
 \end{enumerate}
  \end{dfn}
  
 \end{tcolorbox} 

拡大係数行列の(行)基本変形を行うことで連立1次方程式が解ける(連立方程式の解き方に関しては, 第6回資料を見てください).

\begin{exa}
 連立1次方程式
 $
 \left\{ 
\begin{matrix}
2x&+&3y& -&z &= &-3 \\
-x&+&2y& +&2z &= &1 \\
x&+&y& -&z &= &-2 \\
\end{matrix}
\right.
 $
 を考える. 
 これを拡大係数行列の基本変形と式変形で解いてみて, その対応を表すと下の通りとなる.\footnote{途中で現れる「$\maru{1}+\maru{3}\times(-2)$」は「行列の1行目に3行目の$(-2)$倍を加える」あるいは「1行目の式に3行目の式の$(-2)$倍を加える」を意味している(一応教科書に従った記法である).}

  %まずこれを式変形で解くと次のとおりである.\footnote{途中で現れる「$\maru{1}+\maru{3}\times(-2)$」は「1行目の式に3行目の式の(-2)倍を加える」を意味している(一応教科書に従った記法である).}
%これに対する拡大係数行列の基本変形は以下の通りである. \footnote{途中で現れる「$\maru{1} + \maru{3}\times(-2)$」は「行列の1行目に3行目の(-2)倍を加える」を意味している.}
 
  \begin{align*}
 & &\begin{pmatrix}
 2& 3  & -1&-3\\
-1 & 2 & 2&1\\
1& 1  & -1&-2\\
 \end{pmatrix}
 & \overset{\text{対応}}{\longleftrightarrow}& &\left\{ 
\begin{matrix}
2x&+&3y& -&z &= &-3 \\
-x&+&2y& +&2z &= &1 \\
x&+&y& -&z &= &-2 \\
\end{matrix}
\right.
 \\ %%
 &\overset{\text{$\maru{1} + \maru{3}\times(-2)$}}{\longrightarrow} 
 & \begin{pmatrix}
 0& 1  & 1&1\\
-1 & 2 & 2&1\\
1& 1  & -1&-2\\
 \end{pmatrix}
&\overset{\text{対応}}{\longleftrightarrow} 
&\overset{\text{$\maru{1} + \maru{3}\times(-2)$}}{\longrightarrow} 
 & \left\{ 
\begin{matrix}
 & & y& + &z &= &1 \\
-x&+&2y& +&2z &= &1 \\
x&+&y& -&z &= &-2 \\
\end{matrix}
\right.
 \\ %%
  &\overset{\text{$\maru{2}+\maru{3}\times 1$ }}{\longrightarrow} 
 &\begin{pmatrix}
 0& 1  & 1&1\\
0 & 3& 1&-1\\
1& 1  & -1&-2\\
 \end{pmatrix}
&\overset{\text{対応}}{\longleftrightarrow} 
&\overset{\text{$\maru{2}+\maru{3}\times 1$ }}{\longrightarrow} 
 & \left\{ 
\begin{matrix}
 & & y& + &z &= &1\\
 & &3y& +& z &= &-1 \\
x&+&y& -&z &= &-2 \\
\end{matrix}
\right.
 \\ %%
   &\overset{\text{\maru{3}と\maru{1}を入れ替え}}{\longrightarrow} 
 &\begin{pmatrix}
1& 1  & -1&-2\\
0 & 3& 1&-1\\
 0& 1  & 1&1\\
 \end{pmatrix}
&\overset{\text{対応}}{\longleftrightarrow} 
&\overset{\text{\maru{3}と\maru{1}を入れ替え}}{\longrightarrow} 
 & \left\{ 
\begin{matrix}
x&+&y& -&z &= &-2 \\
 & &3y& +& z &= &-1 \\
 & & y& + &z &= &1\\
\end{matrix}
\right.
 \\ %%
    &\overset{\text{\maru{2}と\maru{3}を入れ替え}}{\longrightarrow} 
 &\begin{pmatrix}
1& 1  & -1&-2\\
 0& 1  & 1&1\\
 0 & 3& 1&-1\\
 \end{pmatrix}
&\overset{\text{対応}}{\longleftrightarrow} 
&\overset{\text{\maru{2}と\maru{3}を入れ替え}}{\longrightarrow} 
 & \left\{ 
\begin{matrix}
x&+&y& -&z &= &-2 \\
 & & y& + &z &= &1\\
  & &3y& +& z &= &-1 \\
\end{matrix}
\right.
 \\ %%
    &\overset{\maru{1}  + \maru{2}\times(-1)}{\underset{\maru{3} +  \maru{2}\times(-3)}{\longrightarrow}}
 &\begin{pmatrix}
1& 0  & -2&-3\\
 0& 1  & 1&1\\
 0 & 0& -2&-4\\
 \end{pmatrix}
&\overset{\text{対応}}{\longleftrightarrow} 
&\overset{\maru{1}  + \maru{2}\times(-1)}{\underset{\maru{3} +  \maru{2}\times(-3)}{\longrightarrow}}
 & \left\{ 
\begin{matrix}
x& & & -&2z &= &-3 \\
 & & y& + &z &= &1\\
  & & & -& 2z &= &-4 \\
\end{matrix}
\right.
 \\ %%
    &\overset{\text{\maru{3}$\times (-\frac{1}{2})$}}{\longrightarrow}
 &\begin{pmatrix}
1& 0  & -2&-3\\
 0& 1  & 1&1\\
 0 & 0& 1&2\\
 \end{pmatrix}
&\overset{\text{対応}}{\longleftrightarrow} 
&\overset{\text{\maru{3}$\times (-\frac{1}{2})$}}{\longrightarrow}
 & \left\{ 
\begin{matrix}
x& & & -&2z &= &-3 \\
 & & y& + &z &= &1\\
  & & &  & z &= &2 \\
\end{matrix}
\right.
 \\ %%
&\overset{\maru{1} + \maru{3}\times 2}{\underset{\maru{2} + \maru{3}\times(-1) }{\longrightarrow}}
 &\begin{pmatrix}
1& 0  & 0&1\\
 0& 1  & 0&-1\\
 0 & 0& 1&2\\
 \end{pmatrix}
&\overset{\text{対応}}{\longleftrightarrow} 
&\overset{\maru{1} + \maru{3}\times 2}{\underset{\maru{2} + \maru{3}\times(-1) }{\longrightarrow}}
 & \left\{ 
\begin{matrix}
x& & &  &  &= &1 \\
 & & y&   & &= &-1\\
  & & &  & z &= &2 \\
\end{matrix}
\right.
 \\ %%
 \end{align*}
 以上より解は$x=1, y=-1, z=2$である.
\end{exa}

%%%%%%%%%%%%%%%%%%%%%%%%%%%%%%%%%%%%%%%%%%%%%%%%%%%%%%%%%
 
 \begin{comment}
 ボツネタ
 
  
 %\overset{\text{$ \maru{1}  + \maru{2}\times(-1)$ と$\maru{3} +  \maru{2}\times(-3) $}}{\longrightarrow} 
 $\overset{\maru{1}  + \maru{2}\times(-1)}{\underset{\maru{3} +  \maru{2}\times(-3)}{\longrightarrow}}$
 
 \begin{align*}
 &\begin{pmatrix}
 2& 3  & -1&-3\\
-1 & 2 & 2&1\\
1& 1  & -1&-2\\
 \end{pmatrix}
\overset{\text{$\maru{1} + \maru{3}\times(-2)$}}{\longrightarrow} 
 \begin{pmatrix}
 0& 1  & 1&1\\
-1 & 2 & 2&1\\
1& 1  & -1&-2\\
 \end{pmatrix}
\overset{\text{$\maru{2}+\maru{3}\times 1$ }}{\longrightarrow} 
 \begin{pmatrix}
 0& 1  & 1&1\\
0 & 3& 1&-1\\
1& 1  & -1&-2\\
 \end{pmatrix}
 \\
 &\overset{\text{\maru{3}と\maru{1}を入れ替え}}{\longrightarrow} 
\begin{pmatrix}
1& 1  & -1&-2\\
0 & 3& 1&-1\\
 0& 1  & 1&1\\
 \end{pmatrix}
  \overset{\text{\maru{2}と\maru{3}を入れ替え}}{\longrightarrow} 
\begin{pmatrix}
1& 1  & -1&-2\\
 0& 1  & 1&1\\
 0 & 3& 1&-1\\
 \end{pmatrix}
 \\
 &  \overset{\text{
$ \maru{1}  + \maru{2}\times(-1)$ と$\maru{3} +  \maru{2}\times(-3) $
 }}{\longrightarrow} 
 \begin{pmatrix}
1& 0  & -2&-3\\
 0& 1  & 1&1\\
 0 & 0& -2&-4\\
 \end{pmatrix}
   \overset{\text{\maru{3}$\times (-\frac{1}{2})$}}{\longrightarrow}
\begin{pmatrix}
1& 0  & -2&-3\\
 0& 1  & 1&1\\
 0 & 0& 1&2\\
 \end{pmatrix} 
 \\
 &\overset{\text{
$\maru{1} + \maru{3}\times 2 $ と $\maru{2} + \maru{3}\times(-1) $
  }}{\longrightarrow} 
  \begin{pmatrix}
1& 0  & 0&-1\\
 0& 1  & 0&-1\\
 0 & 0& 1&2\\
 \end{pmatrix} 
\end{align*}
 
 
 
 
 \begin{align*}
&\begin{pmatrix}
 2& 3  & -1&-3\\
-1 & 2 & 2&1\\
1& 1  & -1&-2\\
 \end{pmatrix}
 &\text{対応}&
  \left\{ 
\begin{matrix}
2x&+&3y& -&z &= &-3 \\
-x&+&2y& +&2z &= &1 \\
x&+&y& -&z &= &-2 \\
\end{matrix}
\right.
 \\
&\overset{\text{$\maru{1} + \maru{3}\times(-2)$}}{\longrightarrow} 
 \begin{pmatrix}
 0& 1  & 1&1\\
-1 & 2 & 2&1\\
1& 1  & -1&-2\\
 \end{pmatrix}
 \\
&\overset{\text{$\maru{2}+\maru{3}\times 1$ }}{\longrightarrow} 
 \begin{pmatrix}
 0& 1  & 1&1\\
0 & 3& 1&-1\\
1& 1  & -1&-2\\
 \end{pmatrix}
 \\
 &\overset{\text{\maru{3}と\maru{1}を入れ替え}}{\longrightarrow} 
\begin{pmatrix}
1& 1  & -1&-2\\
0 & 3& 1&-1\\
 0& 1  & 1&1\\
 \end{pmatrix}
  \overset{\text{\maru{2}と\maru{3}を入れ替え}}{\longrightarrow} 
\begin{pmatrix}
1& 1  & -1&-2\\
 0& 1  & 1&1\\
 0 & 3& 1&-1\\
 \end{pmatrix}
 \\
 &  \overset{\text{
$ \maru{1}  + \maru{2}\times(-1)$ と$\maru{3} +  \maru{2}\times(-3) $
 }}{\longrightarrow} 
 \begin{pmatrix}
1& 0  & -2&-3\\
 0& 1  & 1&1\\
 0 & 0& -2&-4\\
 \end{pmatrix}
   \overset{\text{\maru{3}$\times (-\frac{1}{2})$}}{\longrightarrow}
\begin{pmatrix}
1& 0  & -2&-3\\
 0& 1  & 1&1\\
 0 & 0& 1&2\\
 \end{pmatrix} 
 \\
 &\overset{\text{
$\maru{1} + \maru{3}\times 2 $ と $\maru{2} + \maru{3}\times(-1) $
  }}{\longrightarrow} 
  \begin{pmatrix}
1& 0  & 0&-1\\
 0& 1  & 0&-1\\
 0 & 0& 1&2\\
 \end{pmatrix} 
\end{align*}

\begin{align*}
 \begin{pmatrix}
 2& 3  & -1&-3\\
-1 & 2 & 2&1\\
1& 1  & -1&-2\\
 \end{pmatrix}
&\overset{\text{対応}}{\leftrightarrow} 
 \left\{ 
\begin{matrix}
2x&+&3y& -&z &= &-3 \\
-x&+&2y& +&2z &= &1 \\
x&+&y& -&z &= &-2 \\
\end{matrix}
\right.
\\
\end{align*}


  \end{comment}

 
 
\section{演習問題}
演習問題の解答は授業動画にあります.

1. 連立1次方程式
 $
 \left\{ 
\begin{matrix}
x_1&+&x_2& -&x_3 &= & 1\\
2x_1&+&x_2& +&3x_3&= &4 \\
-x_1&+&2x_2& -&4x_3 &= &-2 \\
\end{matrix}
\right.
 $
 を解け.


 

\end{document}
